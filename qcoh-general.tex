\documentclass[./main.tex]{subfiles}
\begin{document}

We define quasi-coherent sheaves on a scheme $X$
to be such that it is equivalent to
having modules over an affine open cover plus
identification on pairwise intersections.
In practice, Zariski affine covers of schemes 
will be such that 
the intersection of two affine opens will always be affine again.

\begin{dfn}[Quasi-coherent sheaves w.r.t. Zariski affine cover]

  Let $X$ be a scheme and $\cU = \coprod_{i \in I} \SPEC A_i$ 
  a Zariski affine cover
  where the pairwise intersections are affine.
  Consider the following diagram : 
  \begin{cd}
    {\mathcal{U} \times_X \mathcal{U} \times_X \mathcal{U}} & {\mathcal{U} \times_X \mathcal{U}} & {\mathcal{U}}
    \arrow["{p_0}"{description}, shift left=2, from=1-2, to=1-3]
    \arrow["{p_{12}}"{description}, from=1-1, to=1-2]
    \arrow["{p_{02}}"{description}, shift right=4, from=1-1, to=1-2]
    \arrow["{p_{01}}"{description}, shift left=4, from=1-1, to=1-2]
    \arrow["{p_1}"{description}, shift right=2, from=1-2, to=1-3]
  \end{cd}
  where we have the projections \begin{itemize}
    \item $p_i : \cU \times_X \cU \to \cU, (x_0 , x_1) \mapsto x_i$
    \item $p_{ij} : \cU \times_X \cU \times_X \cU \to \cU \times_X \cU, 
    (x_0 , x_1, x_2) \mapsto (x_i , x_j)$
  \end{itemize}
  By our assumption on $X$,
  all schemes are disjoint unions of affine opens of $X$.
  Momentarily, let \[
    \QCOH \cU := \prod_{i \in I} A_i\MOD
  \]
  and similarly for $\QCOH(\cU \times \cU) , \QCOH(\cU \times_X \cU \times_X \cU)$.
  % Suppose we already know what quasi-coherent sheaves look like on
  % $\cU , \cU \times_X \cU , \cU \times_X \cU \times_X \cU$.
  % For example, $\QCOH \cU \simeq \prod_{U \in \cU} \cO(U)\MOD$.
  % Note that pullback along these projections correspond to
  % restricting to double and triple intersections.
  For $p : \SPEC B \to \SPEC A$ any of the above projections,
  we use $p^*$ to denote the functor $B \otimes_A \_$.
  Then define the \emph{category $\QCOH (X , \cU)$ 
  of quasi-coherent sheaves on $X$ w.r.t. $\cU$} as follows : 
  \begin{itemize}
    \item An object of $\QCOH (X , \cU)$ is 
    a set of modules $\cF \in \QCOH(\cU)$
    equipped with a \emph{transition map} 
    $\phi : p_0^* \cF \cong p_1^* \cF$ 
    in $\QCOH( \cU \times_X \cU)$ such that
    \begin{enumerate}
      \item $\phi = \id$ when restricted to the diagonal 
      $\De : \cU \to \cU \times_X \cU$
      \item the following diagram commute : 
      \begin{cd}
        & {p_{01}^*p_1^* \mathcal{F}} \\
        {p_{01}^*p_0^*\mathcal{F}} && {p_{12}^*p_0^*\mathcal{F}} \\
        {p_{02}^*p_0^*\mathcal{F}} && {p_{12}^*p_1^*\mathcal{F}} \\
        & {p_{02}^*p_1^* \mathcal{F}}
        \arrow["{p_{01}^*(\phi)}", from=2-1, to=1-2]
        \arrow["{p_{12}^*(\phi)}", from=2-3, to=3-3]
        \arrow["{p_{02}^*(\phi)}"', from=3-1, to=4-2]
        \arrow["\sim"', from=2-1, to=3-1]
        \arrow["\sim", from=1-2, to=2-3]
        \arrow["\sim", from=3-3, to=4-2]
      \end{cd}
      % \item for $(U_i , U_j , U_k) \in \cU \times \cU \times \cU$,
      % we have ``$\phi_{ik} = \phi_{jk} \phi_{ij}$''.
      % More rigorously, the following diagram commutes :
      % \begin{cd}[sep = small]
      %   {\mathcal{F}(U_i) \otimes_{\mathcal{O}(U_i)} \mathcal{O}(U_{ij}) \otimes_{\mathcal{O}(U_{ij})} \mathcal{O}(U_{ijk})} & {\mathcal{F}(U_i) \otimes_{\mathcal{O}(U_i)} \mathcal{O}(U_{ijk})} \\
      %   {\mathcal{F}(U_j) \otimes_{\mathcal{O}(U_j)} \mathcal{O}(U_{ij}) \otimes_{\mathcal{O}(U_{ij})} \mathcal{O}(U_{ijk})} & {\mathcal{F}(U_i) \otimes_{\mathcal{O}(U_i)} \mathcal{O}(U_{ik}) \otimes_{\mathcal{O}(U_{ik})} \mathcal{O}(U_{ijk})} \\
      %   {\mathcal{F}(U_j) \otimes_{\mathcal{O}(U_j)} \mathcal{O}(U_{ijk})} \\
      %   {\mathcal{F}(U_j) \otimes_{\mathcal{O}(U_j)} \mathcal{O}(U_{jk}) \otimes_{\mathcal{O}(U_{jk})} \mathcal{O}(U_{ijk})} & {\mathcal{F}(U_k) \otimes_{\mathcal{O}(U_k)} \mathcal{O}(U_{ik}) \otimes_{\mathcal{O}(U_{ik})} \mathcal{O}(U_{ijk})} \\
      %   {\mathcal{F}(U_k) \otimes_{\mathcal{O}(U_k)} \mathcal{O}(U_{jk}) \otimes_{\mathcal{O}(U_{jk})} \mathcal{O}(U_{ijk})} & {\mathcal{F}(U_k) \otimes_{\mathcal{O}(U_k)} \mathcal{O}(U_{ijk})}
      %   \arrow["\sim", from=1-1, to=1-2]
      %   \arrow["\sim"', from=5-1, to=5-2]
      %   \arrow["\sim", from=4-2, to=5-2]
      %   \arrow["\sim"', from=2-2, to=1-2]
      %   \arrow["{\phi_{ik} \otimes \mathrm{id}}", from=2-2, to=4-2]
      %   \arrow["{\phi_{jk} \otimes \mathrm{id}}"', from=4-1, to=5-1]
      %   \arrow["\sim", from=4-1, to=3-1]
      %   \arrow["\sim"', from=2-1, to=3-1]
      %   \arrow["{\phi_{ij} \otimes \mathrm{id}}"', from=1-1, to=2-1]
      % \end{cd}
      % https://q.uiver.app/?q=WzAsOSxbMCwwLCJcXG1hdGhjYWx7Rn0oVV9pKSBcXG90aW1lc197XFxtYXRoY2Fse099KFVfaSl9IFxcbWF0aGNhbHtPfShVX3tpan0pIFxcb3RpbWVzX3tcXG1hdGhjYWx7T30oVV97aWp9KX0gXFxtYXRoY2Fse099KFVfe2lqa30pIl0sWzEsMCwiXFxtYXRoY2Fse0Z9KFVfaSkgXFxvdGltZXNfe1xcbWF0aGNhbHtPfShVX2kpfSBcXG1hdGhjYWx7T30oVV97aWprfSkiXSxbMCw0LCJcXG1hdGhjYWx7Rn0oVV9rKSBcXG90aW1lc197XFxtYXRoY2Fse099KFVfayl9IFxcbWF0aGNhbHtPfShVX3tqa30pIFxcb3RpbWVzX3tcXG1hdGhjYWx7T30oVV97amt9KX0gXFxtYXRoY2Fse099KFVfe2lqa30pIl0sWzEsNCwiXFxtYXRoY2Fse0Z9KFVfaykgXFxvdGltZXNfe1xcbWF0aGNhbHtPfShVX2spfSBcXG1hdGhjYWx7T30oVV97aWprfSkiXSxbMSwzLCJcXG1hdGhjYWx7Rn0oVV9rKSBcXG90aW1lc197XFxtYXRoY2Fse099KFVfayl9IFxcbWF0aGNhbHtPfShVX3tpa30pIFxcb3RpbWVzX3tcXG1hdGhjYWx7T30oVV97aWt9KX0gXFxtYXRoY2Fse099KFVfe2lqa30pIl0sWzEsMSwiXFxtYXRoY2Fse0Z9KFVfaSkgXFxvdGltZXNfe1xcbWF0aGNhbHtPfShVX2kpfSBcXG1hdGhjYWx7T30oVV97aWt9KSBcXG90aW1lc197XFxtYXRoY2Fse099KFVfe2lrfSl9IFxcbWF0aGNhbHtPfShVX3tpamt9KSJdLFswLDEsIlxcbWF0aGNhbHtGfShVX2opIFxcb3RpbWVzX3tcXG1hdGhjYWx7T30oVV9qKX0gXFxtYXRoY2Fse099KFVfe2lqfSkgXFxvdGltZXNfe1xcbWF0aGNhbHtPfShVX3tpan0pfSBcXG1hdGhjYWx7T30oVV97aWprfSkiXSxbMCwyLCJcXG1hdGhjYWx7Rn0oVV9qKSBcXG90aW1lc197XFxtYXRoY2Fse099KFVfail9IFxcbWF0aGNhbHtPfShVX3tpamt9KSJdLFswLDMsIlxcbWF0aGNhbHtGfShVX2opIFxcb3RpbWVzX3tcXG1hdGhjYWx7T30oVV9qKX0gXFxtYXRoY2Fse099KFVfe2prfSkgXFxvdGltZXNfe1xcbWF0aGNhbHtPfShVX3tqa30pfSBcXG1hdGhjYWx7T30oVV97aWprfSkiXSxbMCwxLCJcXHNpbSJdLFsyLDMsIlxcc2ltIiwyXSxbNCwzLCJcXHNpbSJdLFs1LDEsIlxcc2ltIiwyXSxbNSw0LCJcXHBoaV97aWt9IFxcb3RpbWVzIFxcbWF0aHJte2lkfSJdLFs4LDIsIlxccGhpX3tqa30gXFxvdGltZXMgXFxtYXRocm17aWR9IiwyXSxbOCw3LCJcXHNpbSJdLFs2LDcsIlxcc2ltIiwyXSxbMCw2LCJcXHBoaV97aWp9IFxcb3RpbWVzIFxcbWF0aHJte2lkfSIsMl1d
      
    \end{enumerate}
    \item a morphism $\eta : (\cF , \phi) \to (\cG , \psi)$ is
    a morphism $\eta : \cF \to \cG$ in $\QCOH \cU$ such that
    the following commutes : 
    \footnote{
      This may remind you of 
      the definition of equivariant maps between
      set endowed with an action of a group $G$.
      This is not a coincidence :
      $\QCOH (X , \cU)$ is effectively
      quasi-coherent sheaves on $\cU$ equipped with
      an action of the \emph{groupoid} $\cU \times_X \cU \rightrightarrows \cU$.
    }
    \begin{cd}
      {p_0^* \cF} & {p_1^* \cF} \\
      {p_0^* \cG} & {p_1^* \cG}
      \arrow["{p_0^*(\eta)}"', from=1-1, to=2-1]
      \arrow["{p_1^*(\eta)}", from=1-2, to=2-2]
      \arrow["\psi"', from=2-1, to=2-2]
      \arrow["\phi", from=1-1, to=1-2]
    \end{cd}
  \end{itemize}
\end{dfn}

\begin{eg}
  
  Let $U_0 := \SPEC k[t] , U_1 = \SPEC k[t^{-1}]$
  be the standard affine opens of $\bP^1_k$.
  Then $U_{01} := U_0 \times_{\bP^1_k} U_1 \simeq \SPEC k[t , t^{-1}]$.
\end{eg}

The above definition is how one goes about giving examples of
quasi-coherent sheaves on non-affine schemes in practice.
For theoretical purposes, we need to have
a definition that is independent of the choice of Zariski affine cover
and then prove that it is equivalent to the above.

\begin{dfn}[Quasi-coherent sheaves on a functor]
  
  Let $X$ be a scheme.
  Then a quasi-coherent sheaf on $X$ 
  consists of the following data : 
  \begin{itemize}
    \item for each $x : \SPEC A \to X$,
    an $A$-module $\cF_x$.
    % \footnote{
    %   I don't want to let $x$ range over all affines mapping into $X$
    %   because this is not a set.
    %   The collection of affine opens of $X$ on the other hand
    %   is a set because it's bounded by
    %   the cardinality of a specific Zariski affine cover of $X$.
    % }
    \item (transition map) 
    for each $f : \SPEC B \to \SPEC A$ and $x : \SPEC A \to X$,
    a morphism of $A$-modules 
    \[
      \cF_x \to \cF_{x f}
    \]
    that is identity when $f$ is.
    We call the map $\cF_x \to \cF_{xf}$ 
    the \emph{transition map associated to $f$}.
    \item (transitivity) We require the maps given in the previous point 
    to satisfy that for any commuting triangle on the left,
    \begin{cd}
      {\SPEC C} && \rightsquigarrow & {\cF_{xfg}} \\
      {\SPEC B} & {\SPEC A} && {\cF_{xf}} & {\cF_x}
      \arrow["g"', from=1-1, to=2-1]
      \arrow["f"', from=2-1, to=2-2]
      \arrow[from=1-1, to=2-2]
      \arrow[from=2-5, to=2-4]
      \arrow[from=2-4, to=1-4]
      \arrow[from=2-5, to=1-4]
    \end{cd}
    we get a commuting triangle on the right of $A$-modules.
    \item (quasi-coherence)
    The transition map associated to $f : \SPEC B \to \SPEC A$
    for any point $x : \SPEC A \to X$
    induces an isomorphism $\cF_x \otimes_A B \simeq \cF_{x f}$.
  \end{itemize}
  Let $\cF , \cG$ be two quasi-coherent sheaves on $X$.
  Then a morphism $\varphi : \cF \to \cG$ consists of the data : 
  \begin{itemize}
    \item for each $x : \SPEC A \to X$,
    a morphism of $A$-modules 
    \[
      \varphi_x : \cF_x \to \cG_x  
    \]
    \item for each $f : \SPEC B \to \SPEC A$ and $x : \SPEC A \to X$,
    we have the commutative diagram in $A$-modules 
    \begin{cd}
      {\mathcal{F}_{xf}} & {\mathcal{G}_{xf}} \\
      {\mathcal{F}_x} & {\mathcal{G}_x}
      \arrow[from=2-1, to=1-1]
      \arrow[from=2-2, to=1-2]
      \arrow["{\varphi_x}"', from=2-1, to=2-2]
      \arrow["{\varphi_{xf}}", from=1-1, to=1-2]
    \end{cd}
    where the vertical morphisms are
    the transition maps of $\cF$ and $\cG$
    associated to $f$.
  \end{itemize}
  We write $\QCOH X$ for the category of quasi-coherent sheaves on $X$.
\end{dfn}

% A priori, a quasi-coherent sheaf consists of
% infinite amount of data.
% We nonetheless have the following for affine schemes.

% \begin{prop}
  
%   Let $X = \SPEC R$.
%   We define a functor $R\MOD \to \QCOH X$.
%   \begin{enumerate}
%     \item (Objects) Let $M \in R\MOD$.
%     \begin{itemize}
%       \item (fibers) 
%       For every algebra map $R \to A$,
%       we have an $A$-module $A \otimes_R M$.
%       \item (transition maps) For $R \to A \to B$,
%       the adjunction $B \otimes_A \_ \dashv \text{ forget } : 
%       A\MOD \leftrightarrow B\MOD$ gives the morphism
%       \[
%         A \otimes_R M \to B\otimes_R M ,
%         a \otimes x \mapsto \tilde{a} \otimes x
%       \]
%       where $\tilde{a}$ is the image of $a$ in $B$.
%       If $A \to B$ is the identity of $A$
%       then the above map is indeed the identity.
%       \item (transitivity) This is clear.
%       \item (quasi-coherence)
%       The map $A \otimes_R M \to B\otimes_R M$
%       indeed induces
%       \[
%         B \otimes_A (A \otimes_R M) \map{\sim}{} B\otimes_R M
%       \]
%     \end{itemize}
%     \item (Morphisms) Let $\al : M \to N$ be a morphism
%     of $R$-modules.
%     Then functoriality of $A \otimes_R \_$
%     across all $R \to A$ defines
%     a morphism between the quasi-coherent sheaves associated with
%     $M$ and $N$.
%     Functoriality is readily checked.
%   \end{enumerate}
%   We have a functor $\QCOH X \to R\MOD$ given by
%   taking fiber at $\id : \SPEC R \to \SPEC R$.
%   Then this gives an equivalence \[
%     R\MOD \simeq \QCOH X  
%   \]
% \end{prop}
% The proof has no geometric content and is included
% purely for pedantry.
% Indeed, this definition of quasi-coherent sheaves
% is designed so that this proposition holds more-or-less by definition.
% \begin{proof}
%   It is clear that the composition
%   $R\MOD \to \QCOH X \to R\MOD$ is on-the-dot
%   the identity functor.
%   We show the other composition
%   $\QCOH X \to R\MOD \to \QCOH X$ is isomorphic to the identity functor.
%   Let $\cF \in \QCOH X$, $M = \cF_{\id_R}$ and $\tilde{M}$
%   the image of $M$ under $R\MOD \to \QCOH X$.
%   Then for each algebra map $x : \SPEC A \to \SPEC R$,
%   quasi-coherence of $\cF$ implies that
%   the equality $M = \cF_{\id_R}$ extends to
%   an isomorphism
%   \[
%     \varphi_x^\cF : \tilde{M}_x = A \otimes_R M \map{\sim}{} \cF_x
%   \]
%   Now we show that for $f : \SPEC B \to \SPEC A$,
%   the above square in the following diagram commutes :
%   \begin{cd}
%     {\tilde{M}_{xf}} & {\mathcal{F}_{xf}} \\
%     {\tilde{M}_x} & {\mathcal{F}_x} \\
%     M & {\mathcal{F}_{\id_R}}
%     \arrow[from=2-1, to=1-1]
%     \arrow[from=2-2, to=1-2]
%     \arrow["{\varphi_x^\cF}"', from=2-1, to=2-2]
%     \arrow["{\varphi_{xf}^\cF}", from=1-1, to=1-2]
%     \arrow[from=3-1, to=2-1]
%     \arrow["{=}"', from=3-1, to=3-2]
%     \arrow[from=3-2, to=2-2]
%   \end{cd}
%   By the universal property of $\tilde{M}_x = A \otimes_R M$,
%   it suffices to show that from $M$,
%   ``up-up-right = up-right-up''.
%   Indeed \begin{align*}
%     \text{ ``up-up-right'' }
%     &= \text{ ``right-up-up'' }\,\,\,\text{by def of $\varphi_{xf}$} \\
%     &= \text{ ``up-right-up'' }\,\,\,\text{by def of $\varphi_x$}
%   \end{align*}
%   We have thus defined an isomorphism morphism 
%   $\varphi^\cF : \tilde{M} \simeq \cF$.
%   Now let $\al : \cF \to \cG$ in $\QCOH X$
%   and $\varphi^\cG : \tilde{N} \simeq \cG$ be the corresponding
%   reconstruction morphism for $\cG$.
%   Let $\tilde{\al} : \tilde{M} \to \tilde{N}$ be the
%   reconstructed morphism from $\al$.
%   We wish to check commutativity of
%   \begin{cd}
%     {\tilde{M}} & {\mathcal{F}} \\
%     {\tilde{N}} & {\mathcal{G}}
%     \arrow["{\tilde{\alpha}}"', from=1-1, to=2-1]
%     \arrow["{\varphi^\mathcal{G}}"', from=2-1, to=2-2]
%     \arrow["\alpha", from=1-2, to=2-2]
%     \arrow["{\varphi^\mathcal{F}}", from=1-1, to=1-2]
%   \end{cd}
%   We consider how we defined $\varphi_x^\cF, \varphi_x^\cG$ in the first place : 
%   \begin{cd}
%     M & {\mathcal{F}_{\id_R}} \\
%     N & {\mathcal{G}_{\id_R}} & {\tilde{M}_x} & {\mathcal{F}_x} \\
%     && {\tilde{N}_x} & {\mathcal{G}_x}
%     \arrow["{\tilde{\alpha}_x}"', from=2-3, to=3-3]
%     \arrow["{\varphi^\mathcal{G}_x}"', from=3-3, to=3-4]
%     \arrow["{\alpha_x}", from=2-4, to=3-4]
%     \arrow["{\varphi^\mathcal{F}_x}", from=2-3, to=2-4]
%     \arrow["{=}", from=2-1, to=2-2]
%     \arrow["{=}", from=1-1, to=1-2]
%     \arrow["{\alpha_{\id_R}}"', from=1-1, to=2-1]
%     \arrow["{\alpha_{\id_R}}", from=1-2, to=2-2]
%     \arrow[from=2-1, to=3-3]
%     \arrow[from=1-1, to=2-3]
%     \arrow[from=2-2, to=3-4]
%     \arrow[from=1-2, to=2-4]
%   \end{cd}
%   By the universal property of $\tilde{M}_x = A\otimes_R M$,
%   it suffices to show that starting from $M$
%   we have ``diagonal-right-down = diagonal-down-right''.
%   Indeed \begin{align*}
%     \text{ ``diag-right-down'' }
%     &= \text{ ``right-diag-down' }\,\,\,\text{by def of $\varphi_{x}^\cF$} \\
%     &= \text{ ``right-down-diag'' }\,\,\,\text{by def of $\al$} \\
%     &= \text{ ``down-right-diag'' }\,\,\, \text{clear}\\
%     &= \text{ ``down-diag-right'' }\,\,\,\text{by def of $\varphi_{x}^\cG$} \\
%     &= \text{ ``diag-down-right'' }\,\,\,\text{by def of $\al_{x}$} \\
%   \end{align*}
% \end{proof}
Note that a priori, a quasi-coherent sheaf consists of infinite amount of data.
\footnote{
  For those worried about set-theoretic issues,
  we do indeed have a problem here.
  For any functor $X$,
  the category $\AFF / X$ is too large to be a set.
  So as of now, $\QCOH X$ has a proper class of objects
  and even for $\cF , \cG \in \QCOH X$,
  the collection $(\QCOH X)(\cF , \cG)$ is also a proper class.
}
We first note :
\begin{prop}
  For $X = \SPEC A$,
  \[
    \QCOH X \simeq A\MOD  
  \]
\end{prop}
\begin{proof}
  Exercise.
\end{proof}
We now justify our motivation that 
we can compute quasi-coherent sheaves from affine open covers.

% \textbf{Hmmmm finiteness issues. 
% A scheme is allowed to be covered by infinitely many affine opens,
% i.e. allowed to be not quasi-compact.
% So even if we assume affine diagonal,
% the pullback of a Zariski cover to an affine may not be
% a standard Zariski cover of the affine.
% However, a Zariski cover of an affine
% has a canonically refinement by an infinite standard Zariski cover.
% So we only need descent w.r.t. infinite standard Zariski covers.
% Hmmm but it's not true that infinite product of flat is flat.
% Does this mean I can only defined quasi-coherent sheaves for
% quasi-compact schemes?
% Maybe this is okay for first passing of algebraic geometry.
% Even in general,
% Picard schemes, quot schemes etc are only not quasi-compact because
% they are infinite disjoint union of quasi-compact schemes.
% I guess I go with qcoh sheaves only on qc schemes
% and show fpqc descent beforehand.
% }

% \begin{prop}[Faithfully flat descent of modules]
  
%   \cite[\href{https://stacks.math.columbia.edu/tag/023N}{Tag 023N}]{stacks-project}
%   \footnote{
%     Note that an infinite product of flat algebras need not stay flat!
%   }
% \end{prop}


\begin{prop}

  Let $X$ be a scheme. 
  Consider $\QCOH_\AFF X$ which is defined as
  the same as $\QCOH X$ except we only look at $x : \SPEC A \to X$
  which is an open of $X$.
  Also let $\cU$ be a Zariski affine cover of $X$
  where pairwise intersections are affine.
  Then the restriction functors
  are equivalences of categories.\footnote{
    In particular, all set-theoretic issues are resolved.
    $\QCOH_\AFF X$ has a set of objects
    and for $\cF , \cG \in \QCOH_\AFF X$,
    the collection $(\QCOH_\AFF X)(\cF , \cG)$ is a set.
  }
  \begin{cd}
    {\QCOH X} & {\QCOH_\AFF X} \\
    & {\QCOH(X , \cU)}
    \arrow["\sim", from=1-1, to=1-2]
    \arrow["\sim"', from=1-1, to=2-2]
    \arrow["\sim", from=1-2, to=2-2]
  \end{cd}
\end{prop}
\begin{proof}
  We only construct a quasi-inverse to the diagonal arrow.
  A quasi-inverse to the horizontal map is similar
  using a choice of arbitrary Zariski affine cover of $X$.
  
  (Objects)
  Let $(\cF , \phi) \in \QCOH(X , \cU)$.
  Let $x : \SPEC A \to X$.
  By assumption of $\cU \to X$ being a Zariski epimorphism,
  there exists a standard Zariski cover $\cU_A$ of $\SPEC A$ and
  a commuting diagram : \begin{cd}
    {\cU_A} & \cU \\
    {\SPEC A} & X
    \arrow[from=1-1, to=2-1]
    \arrow[from=2-1, to=2-2]
    \arrow[from=1-2, to=2-2]
    \arrow["\exists", dashed, from=1-1, to=1-2]
  \end{cd}
  For such a $\cU_A$,
  we have an explicit inverse functor given by Zariski descent of modules : 
  \begin{cd}
    {\mathrm{QCoh}(\mathrm{Spec}\,A , \cU_A)} & A\MOD \\
    {(\res{\mathcal{F}}{\mathcal{\cU_A}} , \res{\phi}{\mathcal{\cU_A}})} & 
    {\mathrm{Eq}(\prod_{V \in \cU_A}\mathcal{F}(V) \rightrightarrows 
    \prod_{V_0, V_1 \in \cU_A}\mathcal{F}(V_0 \cap V_1))}
    \arrow["{\mathrm{Glue}_{\cU_A}}", from=1-1, to=1-2]
    \arrow[maps to, from=2-1, to=2-2]
  \end{cd}
  And hence gives
  \[
    \mathrm{Glue}_A := \mathrm{Glue}_{\cU_A} : \QCOH(X , \cU) \to A\MOD
  \]
  However, in order to get a quasi-coherent sheaf on $X$
  we also need for each inclusion of affine opens
  $\SPEC B \to \SPEC A \to X$ a morphism of $A$-modules
  \[
    \mathrm{Glue}_A(\cF , \phi) \to \mathrm{Glue}_B(\cF , \phi)
  \]
  satisfying identity, transitivity, and quasi-coherence.
  So we cannot depend on the choice of $\cU_A$.
  We thus redefine $\mathrm{Glue}_A$ by using all of them at once : 
  \[
    \mathrm{Glue}_A : \QCOH(X , \cU) \to A\MOD , (\cF , \phi) \mapsto 
    \COLIM_{\cV} \mathrm{Glue}_{\cU_A}(\res{\cF}{\cU_A} , \res{\phi}{\cU_A})  
  \]
  where the colimit is over the filtered diagram of
  standard Zariski covers $\cU_A$ of $\SPEC A$ factoring through $\cU$.
  \footnote{
    The lesson here is that
    when you do not want to make a choice and 
    all choices are equivalent by induced maps,
    just take the limit or colimit of the choices.
    This gives something isomorphic to any choice but is canonical.
  }
  Now we get induced transition maps
  which satisfies identity and transitivity.
  It remains to check quasi-coherence.
  The point is that the given projection
  \[
    \mathrm{Glue}_A(\cF , \phi) \map{\sim}{} 
    \mathrm{Glue}_{\cU_A}(\res{\cF}{\cU_A} , \res{\phi}{\cU_A})  
  \]
  is still an isomorphism for any standard Zariski cover $\cU_A$
  factoring through $\cU$ because all the transition maps
  of the filtered diagram are isomorphisms.
  Furthermore, for inclusions of $\SPEC B \to \SPEC A$ of affine opens of $X$,
  the base change $\cU_B$ of $\cU_A$ to $\SPEC B$ is another standard Zariski cover.
  So it follows that 
  \begin{cd}
    {\mathrm{Glue}_A(\mathcal{F} , \phi)} & 
    {\mathrm{Glue}_B(\mathcal{F} , \phi)} & 
    \rightsquigarrow & 
    {B \otimes_A \mathrm{Glue}_A(\mathcal{F} , \phi)} & 
    {\mathrm{Glue}_B(\mathcal{F} , \phi)} \\
    {\mathrm{Glue}_\cU(\res{\cF}{\cU_A} , \res{\phi}{\cU_A})} & 
    {\mathrm{Glue}_\cU(\res{\cF}{\cU_B} , \res{\phi}{\cU_B})} &
    & 
    {B \otimes_A \mathrm{Glue}_\cU(\res{\cF}{\cU_A} , \res{\phi}{\cU_A})} & 
    {\mathrm{Glue}_\cU(\res{\cF}{\cU_B} , \res{\phi}{\cU_B})}
    \arrow["\sim"', from=1-1, to=2-1]
    \arrow["\sim"', from=1-2, to=2-2]
    \arrow[from=1-1, to=1-2]
    \arrow[from=2-1, to=2-2]
    \arrow["\sim"', from=2-4, to=2-5]
    \arrow["\sim"', from=1-4, to=2-4]
    \arrow["\sim", from=1-5, to=2-5]
    \arrow[from=1-4, to=1-5]
  \end{cd}
  So $\mathrm{Glue}_A(\cF , \phi)$ assembles across
  all affine opens of $X$
  to give a quasi-coherent sheaf $\mathrm{Glue}(\cF , \phi)$ on $X$.

  (Morphisms) 
  Clear from the construction of $\mathrm{Glue}$.

  ($\QCOH(X , \cU) \to \QCOH(X , \cU)$ isomorphic to identity)
  Clear by construction.

  ($\QCOH X \to \QCOH X$ isomorphic to identity)
  Also clear from construction.
\end{proof}

% \begin{proof}[Alternative proof using fpqc descent.]
%   Let $R$ denote the restriction functor.
%   % We are doing an example of fpqc descent.

%   (Faithful)
%   For faithfullness,
%   reduce to requiring for $\cF \in \QCOH X$, 
%   $R \cF = 0$ implies $\cF = 0$.
%   Idea : 
%   for any $x : \SPEC A \to X$,
%   the Zariski affine cover $\cU$ of X and X affine diagonal gives 
%   Zariski affine cover $\cU_A$ of $\SPEC A$.
%   The issue is that without fpqc descent of modules,
%   we can only descent modules on \emph{standard} Zariski covers of $\SPEC A$.
%   Technically,
%   we can refine $\cU_A$ to a standard Zariski cover
%   and then
%   the point now is that $\cF_x \in A\MOD$ is zero
%   iff it is zero when restricted to a standard Zariski cover.
%   This works for faithfulness.
%   However as we will now see for fullness,
%   this will not work because taking a finite refinement is not functorial.

%   (Fullness)
%   Let $\cF, \cG \in \QCOH X$ and $\phi : R \cF \to R \cG$.
%   We seek to construct $\cF \to \cG$ which restricts to $\phi$.
%   Let $x : \SPEC A \to X$.
%   Goal : construct $\cF_x \to \cG_x$ in $A\MOD$.
%   Again,
%   the finite Zariski cover $\cU$ of $X$
%   pulls back to a finite Zariski cover $\cU_A$ of $\SPEC A$ as a Zariski sheaf.
%   We want to use descent w.r.t $\cU_A$
%   to construct $\cF_x \to \cG_x$.
%   However, we further need this to be functorial w.r.t $x$.
%   So we \emph{cannot} just choose a refinement of $\cU_A$
%   to a standard Zariski cover because this will not be functorial.
%   Fortunately, due to finiteness of $\cU_A$ 
%   it is an fpqc cover of $\SPEC A$
%   so we have fpqc descent.
%   For each $U \in \cU$, 
%   denote $U_A := \SPEC A \times_X U$ and
%   the pullback of $\cF$ to $U_A$ as $\cF(U_A)$.
%   The point of fpqc descent is that
%   ``$\cF_x$ is equivalent to gluing data on a cover of $\SPEC A$''. 
%   More concretely, 
%   $\cF_x$ can be realised by the equalizer diagram : 
%   \[
%     0 \to \cF_x \to \prod_{U \in \cU} \cF(U_A) \rightrightarrows
%     \prod_{U, V \in \cU} \cF(U_A \cap V_A)
%   \]
%   The data of $\phi$ then gives a morphism $\cF_x \to \cG_x$
%   functorially in $x$.
%   One can check that this does nothing when $\SPEC A$ is
%   one of the affine opens in the cover $\cU$ as desired.

%   (Essential Surjectivity)
%   Take $\cF \in \QCOH(X , \cU)$.
%   Goal : construct $\cF_x$ for every $x : \SPEC A \to X$ functorially
%   and quasi-coherent.

%   Sub-goal : fix $x : \SPEC A \to X$. 
%   Construct $\cF_x \in A\MOD$ first.
%   This time, we \emph{define} $\cF_x$ by the equalizer diagram in $A\MOD$
%   \[
%     0 \to \cF_x \to \prod_{U \in \cU} \cF(U_A) \rightrightarrows
%     \prod_{U, V \in \cU} \cF(U_A \cap V_A)
%   \]

%   Sub-goal : now functoriality.
%   We are given 
%   \[
%     \SPEC B \overset{f}{\to} \SPEC A \overset{x}{\to} X  
%   \]
%   We need to give a morphism $\cF_x \to \cF_{x f}$ in $A\MOD$.
%   The forgetful functor $B\MOD \to A\MOD$ preserves limits,
%   so the UP of $\cF_{x f}$ induces the dashed morphism in the
%   commutative diagram : 
%   \begin{cd}
%     0 & {\mathcal{F}_{xf}} & {\prod_{U \in \mathcal{U}} \mathcal{F}(U_B)} & {\prod_{U , V \in \mathcal{U}} \mathcal{F}(U_B \cap V_B)} \\
%     0 & {\mathcal{F}_x} & {\prod_{U \in \mathcal{U}} \mathcal{F}(U_A)} & {\prod_{U , V \in \mathcal{U}} \mathcal{F}(U_A \cap V_A)}
%     \arrow[from=2-1, to=2-2]
%     \arrow[from=2-2, to=2-3]
%     \arrow[shift right=1, from=2-3, to=2-4]
%     \arrow[shift left=1, from=2-3, to=2-4]
%     \arrow[shift left=1, from=1-3, to=1-4]
%     \arrow[shift right=1, from=1-3, to=1-4]
%     \arrow[from=1-2, to=1-3]
%     \arrow[from=1-1, to=1-2]
%     \arrow[from=2-4, to=1-4]
%     \arrow[from=2-3, to=1-3]
%     \arrow[dashed, from=2-2, to=1-2]
%   \end{cd}
%   The compatibility of compositions $\SPEC C \to \SPEC B \to \SPEC A \to X$ 
%   comes from the UP of equalizers.

%   Sub-goal : Quasi-coherence.
%   We need to show that the morphisms $\cF_x \to \cF_{xf}$ induces
%   $\cF_x \otimes_A B \iso \cF_{x f}$ in $B\MOD$.
%   To prove this morphisms induces $\cF_x \otimes_A B \iso \cF_{xf}$,
%   it suffices to check it is an isomorphism on each open $U_B$ in
%   the cover of $\SPEC B$.
%   Tensoring $\cF_x \otimes_A B \to \cF_{xf}$ by $\cO(U_B)$,
%   we obtain the commutative diagram in $\QCOH U_B$ :
%   \begin{cd}
%     {\mathcal{F}_{xf} \otimes_B \mathcal{O}(U_B)} & {\mathcal{F}(U_B)} \\
%     {\mathcal{F}_x \otimes_A B \otimes_B \mathcal{O}(U_B)} & {\mathcal{F}(U_A) \otimes_{\mathcal{O}(U_A)} \mathcal{O}(U_B)}
%     \arrow[from=2-1, to=1-1]
%     \arrow["\cong", from=1-1, to=1-2]
%     \arrow["\cong"', from=2-1, to=2-2]
%     \arrow["\cong"', from=2-2, to=1-2]
%   \end{cd}

% \end{proof}

% One make the technical leap to infinite covers.
% The proof is abstract non-sense.
% \begin{prop}
%   Let $X$ be a scheme with affine diagonal
%   and $\cU$ a Zariski affine cover of $X$.
%   Then the restriction functor gives an equivalence \[
%     \QCOH X \map{\sim}{} \QCOH(X , \cU)  
%   \]
% \end{prop}
% \begin{proof}
  
% \end{proof}


\begin{prop}
  
  Let $X$ be a scheme.
  Then 
  \begin{enumerate}
    \item $\QCOH X$ is an abelian category with
    small colimits.
    \item All small colimits and finite limits are computed
    on affine opens.
    \item fiberwise tensor product endows $\QCOH X$
    with a symmetric monoidal structure.
  \end{enumerate}
\end{prop}
\begin{proof}
  Small colimits are computed fiberwise.
  This only relies on the fact that tensor product
  preserves small colimits.
  For kernels to exist,
  one must compute on affine opens.
  Then the key fact is that 
  for $\SPEC B \to \SPEC A$ an open embedding,
  the functor $B \otimes_A \_$ is flat.
\end{proof}

\end{document}