\documentclass[./main.tex]{subfiles}
\begin{document}

Although schemes are built out of commutative rings
and pictures can be drawn without reference to topological spaces,
it can sometimes be useful to have a topological space.
In this section, we show how to associate a topological space $\abs{X}$
to a scheme $X$.
The reader is warned that the procedure
$X \rightsquigarrow \abs{X}$ loses \emph{a lot} of information.
One \emph{should not} conflate $X$ with $\abs{X}$.

\begin{dfn}
  
  For $A \in \CRING$, 
  define \[
    \abs{\SPEC A} := \set{A \to K \st K \text{ field}} / \sim
  \]
  where the equivalence relation is generated by 
  $(A \to K) \sim (A \to L)$ when the first factors through the second.
  This forms a functor to $\AFF \to \SET$.
  For $f \in A$, because $A \to A_f$ is an epimorphism
  it follows that $\abs{D(f)} \to \abs{\SPEC A}$ is injective,
  which we view as a subset.
  Let the opens in $\abs{\SPEC A}$
  be generated by subsets of the form $\abs{D(f)}$ for $f \in A$.
  Then this forms a functor \[
    \abs{\_} : \AFF \to \TOP
  \]
\end{dfn}

\begin{rmk}[Worries]

  For those who have already done some modern algebraic geometry, 
  you may be wondering where the prime ideals are,
  since $\abs{\SPEC A}$ is more standardly defined as 
  the set of prime ideals of $A$.
  A perhaps more serious question is whether under our definition,
  $\abs{\SPEC A}$ is even a set at all?
  
  Both these questions are answered by the following lemma 
  which realizes our idea that for $p \in \abs{\SPEC A}$,
  $\ev_p : A \to \ka(p)$ is the minimal representative of $p$.
  
\end{rmk}

\begin{lem}[Evaluation]
  
  Let $A \in \RING$, $p \in \abs{\SPEC A}$ be a point.
  Then there exists $\ev_p : A \to \ka(p)$
  where $\ka(p)$ is a field that is minimal in 
  the equivalence class $p$,
  i.e. every other $A \to K$ representing $p$ 
  factors uniquely through $\ev_p$.
  $\ev_p$ is thus unique up to unique isomorphism in the class $p$.
  In fact, $\ev_p$ is an epimorphism of rings. 
  Its codomain $\ka(p)$ is called the \emph{residue field at $p$}.

  Hence, 
  $\abs{\SPEC A}$ bijects with the set of prime ideals of $A$.
  
\end{lem}
\begin{proof}
  For any $\ev_K : A \to K$ representing $p$,
  the ideal $I = \ker\ev_K$ is independent of $K$ by 
  the equivalence relation and ring morphisms from fields being injective. 
  Define $\ev_p : A \to \ka(p)$ with $\ka(p) := \text{Frac }A/I$.
  The the UP of quotients and fields of fractions 
  implies the minimality of $\ev_p$ as a representative of $p$.
  $\ev_p$ is indeed epi and $I = \ker\ev_p$.

  Prime ideals inject into $\abs{\SPEC A}$ via 
  $\f(p) \mapsto \text{Frac }A/\f(p)$.
  Now for $p \in \abs{\SPEC A}$,
  we already saw that $\ka(p) = \text{Frac }A/\ker\ev_p$.
  $\ker\ev_p$ is thus a prime ideal mapping to $p$, proving surjectivity. 
\end{proof}

\begin{dfn}
  
  Let $X$ be a scheme.
  Then define 
  \[
    \abs{X} := \COLIM_{U} \abs{U}
  \]
  where $U$ runs through all affine opens of $X$.
  This upgrades our previous functor to
  \[
    \abs{\_} : \SCH \to \TOP  
  \]
\end{dfn}

\begin{prop}
  
  Let $\cU$ be a Zariski affine cover of a scheme $X$.
  Then we have a coequalizer diagram \[
      \abs{\cU \times_X \cU} \rightrightarrows
      \abs{\cU} \to \abs{X}
  \]
\end{prop}

% \begin{prop}[Topological $\spec$]
  
%   Let $A \in \RING$.
%   For any $f \in A$, 
%   define the \emph{support of $f$} to be \[
%     D(f) := \set{p \in \abs{\SPEC A} \st \ev_p(f) \neq 0}
%   \]
%   The subsets of $\abs{\SPEC A}$ of the above form are called 
%   \emph{basic opens}.
%   We will use $D(A)$ to denote the set of basic opens. 

%   Then the following are true : 
%   \begin{itemize}
%     \item (Pretopology)
%     For all basic opens $U, V \subs \abs{\SPEC A}$,
%     there exists a basic open $W \subs \abs{\SPEC A}$ such that $W \subs U \cap V$.
%     In particular, for $f, g \in A$, $D(f) \cap D(g) = D(fg)$.

%     Hence $D(A)$ is a pretopology on $\abs{\SPEC A}$.
%     The topology it generates is called the \emph{Zariski topology},
%     the topology initial with respect to basic opens being open. 

%     \item (Morphisms)
%     Let $\ph^\bot \in \RING(A,B)$ and 
%     $\ph^\tau \in \SET(\spec B, \abs{\SPEC A})$ the corresponding map. 
%     For $f \in A$, 
%     we have \[
%       {\ph^\tau}\inv D(f) = D(\ph^\bot(f))
%     \]
%     Hence, by giving every spectrum the Zariski topology, 
%     we have the functor : 
%     \begin{cd}
%       \RING\op \ar[r,"\spec"] & \TOP 
%     \end{cd}
%   \end{itemize}

% \end{prop}
% \begin{proof}
%   \textit{(Morphism)}
%   For $f \in A$ and $p \in \spec B$,
%   $p \in \ph\inv D(f) \iff \ev_{\ph(p)}(f) \neq 0 
%   \iff \ev_p(\ph^\bot(f)) \neq 0
%   \iff p \in D(\ph^\bot(f))$.
% \end{proof}

\end{document}