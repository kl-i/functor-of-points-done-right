\documentclass[./main.tex]{subfiles}
\begin{document}

Let $\SPEC A$ be a physical system
and $a_0$ a single state in it.
Suppose $a_0$ actually lies in a line of states $a$,
\begin{cd}
  \PT & {\SPEC A} \\
	{\bA^1}
	\arrow["a"', from=2-1, to=1-2]
	\arrow["0"', from=1-1, to=2-1]
	\arrow["{a_0}", from=1-1, to=1-2]
\end{cd}
Then given a measuring device $f$ on $\SPEC A$,
we get 
\[
  f(a) = f_0 + f_1 t + f_2 t^2 + \cdots \in \cO(\bA^1)
\]
where $t$ is the standard meausring device on $\bA^1$.
Note that $f_0 = f(a_0)$.
This says the way the value of $f$ changes over
the family of states $a$ is described as
a polynomical expression in the values of $t$.

The idea of a \emph{tangent vector at the state $a_0$ in the direction of $a$}
is we want only the \emph{first order change of $f$ along $a$}.
With highschool intuition, we write
\[
  \brkt{\frac{\partial}{\partial a}}_{a_0} f
  = \frac{f(a) - f(a_0)}{t} \text{ ignore $t^2$ and above}
\]
But we see that this is actually formal! This is the coefficient $f_1$.
This gives the first definition of a tangent vector.

\begin{dfn}[Deformation perspective of tangent vectors]

  A tangent vector at $a_0$ is an extension 
  \begin{cd}
    \PT & {\SPEC A} \\
    {\SPEC k[t]/(t^2)}
    \arrow["a"', from=2-1, to=1-2, dashed]
    \arrow["0"', from=1-1, to=2-1]
    \arrow["{a_0}", from=1-1, to=1-2]
  \end{cd}

  For such an extension,
  we will write 
  \[
    f(a) = f(a_0) + \brkt{\brkt{\frac{\partial}{\partial a}}_{a_0} f} t
  \]

  More generally,
  a vector field along a map $a_0 : \SPEC B \to \SPEC A$
  is an extension
  \begin{cd}
    {\SPEC B} & {\SPEC A} \\
    {\SPEC B[t]/(t^2)}
    \arrow["a"', from=2-1, to=1-2, dashed]
    \arrow["0"', from=1-1, to=2-1]
    \arrow["{a_0}", from=1-1, to=1-2]
  \end{cd}
\end{dfn}

Cons of this perspective : 
\begin{itemize}
  \item the $k$-vector space structure of tangent spaces
  is not clear.
  It is only clearly pointed.
  \footnote{
    Actually, a $k$-vector space structure is present,
  but it is non-trivial to check.
  The key is 
  \[
    \frac{k[x]}{(x^2)} \times_k \frac{k[y]}{(y^2)}
    \simeq \frac{k[x,y]}{(x,y)^2}
  \]
  Then the $k$-vector space structure on $(t)$ inside $k[t]/(t^2)$
  makes $k[t]/(t^2)$ into a
  \emph{``$k$-vector space object in $k$-algebras''}.
  \footnote{
    i.e. it is an abelian group object in $k$-algebras and
    additionally we have a monoid morphism $k \to \END_{k\dash\ALG} k[t]/(t^2)$.
  }
  Making this into spaces,
  we want $\SPEC \frac{k[x,y]}{(x,y)^2}$ to be the pushout of
  $\SPEC k[x]/(x^2) \leftarrow \SPEC k \rightarrow k[y]/(y^2)$
  at least in the category of $k$-schemes.
  One of showing this is to see that any map
  into a $k$-scheme must factor through an affine open
  and thus reduces to the affine case where it is true.
  This endows the $k$-valued points of the tangent bundle with
  a $k$-vector space structure.
  To get this for $B$-valued points where $B$ is any $k$-algebra,
  one must repeat the above for $B[t]/(t^2)$.
  The non-trivial part is showing $\SPEC B[x,y]/(x,y)^2$ 
  is the desired pushout in the category of schemes.
  This is \cite[\href{https://stacks.math.columbia.edu/tag/07RS}{Lemma 37.14.1}]{stacks}.
  }
  At this point, this is too much work for something that should be trivial.
  This \emph{is} however useful for defining tangent bundle of stacks
  because it is less obvious to generalise algebraic derivations.
\end{itemize}

Pros of this perspective : 
\begin{itemize}
  \item it is closest to highschool intuition :
  that of first order change.
  \item the definition of the tangent bundle of $\SPEC A$ is intuitive.
  It is the space $T\SPEC A$ such that
  \[
    \AFF(\SPEC \_ , T\SPEC A) \simeq 
    \AFF(\SPEC (\_ [t] / (t^2)) , \SPEC A)
  \]
  This exists a priori in $\PSH \AFF$.
  \item it is clear that tangent vectors pushforward;
  this is a direct consequence of composing maps.
  \item the relative definition is also clear.
  Consider the example of $\pi : \bA^2 \to \bA^1, (x , y) \mapsto x$
  and finding a tangent vectors in $\bA^2$ relative to $\pi$
  at points.
  You will discover that the tangent vectors are forced to be
  \emph{vertical},
  i.e. lie within the fiber of $\pi$.
  This leads to the general definition :
  Given a map $X \to S$ of affine schemes,
  a vector field along a map $T \to X$ is a solution
  to the lifting problem : 
  \begin{cd}
    T & X \\
    {T \times_k k[t]/(t)^2} & S
    \arrow[from=1-1, to=2-1]
    \arrow[from=1-1, to=1-2]
    \arrow[from=2-1, to=2-2]
    \arrow[from=1-2, to=2-2]
    \arrow[dashed, from=2-1, to=1-2]
  \end{cd}
  
  \item Given a sequence of affine schemes $X \to Y \to S$,
  we get $T(X / S) \to T(Y / S)$ where $T(\_ / S)$ denotes
  tangent bundle relative to $S$.
  There is also $T(X / Y)$ consisting of tangent vectors in $X$
  which lie in the fibers of the map $X \to Y$.
  Since fibers of $X \to Y$ lie in fibers of $X \to S$,
  we obtain 
  \[
    T(X / Y) \subs T(X / S) \to T(Y / S)
  \]
  Looking at the fibers as pointed sets,
  one can see that $T(X / Y)$ is
  in the ``kernel'' of $T(X / S) \to T(Y / S)$.
  Conversely, any tangent vector of $X$ relative $S$
  which dies under projection to $Y$
  must tautologically lie in $T(X / Y)$.
  So the above is ``left exact'',
  if only we are able to put these three spaces
  inside some abelian category.
  This is one of the pros of the next definition of tangent vectors.

\end{itemize}

One can see that the data of $a$ is equivalent to
specifying the linear operator 
$\brkt{\frac{\partial}{\partial a}}_{a_0} \in k\MOD(A , k)$.
Following one's nose, 
we see that such linear operators $\partial$ satisfy the Leibniz rule.
\[
  \partial(f g) = \partial(f) g(a_0) + f(a_0) \partial(g)
\]
These are called \emph{$k$-derivations}.
Let $\mathrm{Der}_k(A , k) \subs k\MOD(A , k)$ denote the
subset of $k$-derivations.
One can show $\mathrm{Der}_k(A , k)$ bijects
with the set tangent vectors in our first definition,
giving a second definition of tangent vectors.

\begin{dfn}[Pragmatic perspective on tangent vectors]
  
  A tangent vector at state $a_0$ is a 
  $k$-derivation from $A$ to $k_{a_0}$.

  More generally, a vector field along $a_0 : \SPEC B \to \SPEC A$
  is a $k$-derivation $A \to B$ where $B$ is an $A$-module via
  its $A$-algebra structure.
\end{dfn}

Pros of the this perspective : 
\begin{itemize}
  \item Computable!
  E.g. show that the tangent space of any point in $\bA^n$ 
  is an $n$-dimensional vector space.
  \item Makes obvious the $k$-vector space structure of tangent spaces.
  We can do even better.
  Given $a : \SPEC B \to \SPEC A$,
  \begin{align*}
    \AFF_A (\SPEC B , T\SPEC A)
    \simeq \mathrm{Der}_k(A , B)
  \end{align*}
  shows that local sections of the tangent bundle $T\SPEC A$
  have an $A$-module structure.

  \item Let $X \to Y \to S$ be maps of affine schemes and
  $R \to A \to B$ be the corresponding maps of algebras of functions.
  Then the sequence of spaces 
  \[
    T(X / Y) \subs T(X / S) \to T(Y / S)
  \]
  corresponds to the sequence 
  \[
    0 \to \mathrm{Der}_A(B , \_) \overset{(1))}{\to}
    \mathrm{Der}_R(B , \_) \overset{(2)}{\to} \mathrm{Der}_R(A , \_)
  \]
  \begin{enumerate}
    \item ``fibers of $X \to Y$ lie in fibers of $X \to S$ so
    relative tangents of $X$ to $Y$ are also relative tangents of $X$ to $S$''
    is precisely the injection (1).
    \item ``relative tangents of $X \to Y$ die when pushforward to $Y$''
    is precisely the fact that at (2)
    the image lies in the kernel.
    \item ``relative tangents of $X \to S$ whose pushforward to $Y$ is zero
    are relative tangents of $X \to Y$''
    precisely says that at (2) the kernel lies in the image.
  \end{enumerate}
  
\end{itemize}

So far, we have seen how to
differentiate $f \in A$ with respect to some state $a \in \SPEC A$
using a tangent vector.
In other words,
we have the inclusion of $A$-modules 
\[
  \mathrm{Der}_k(A , \ka(a)) \to k\MOD(A , k)
\]
But the duality of space and function via evaluation
gives us \begin{align*}
  A &\to k\MOD(\mathrm{Der}_k(A , \kappa(a)) , k) \\
  f &\mapsto 
  \brkt{
    \res{\frac{\partial}{\partial v}}{a} \mapsto 
    \res{\frac{\partial}{\partial v}}{a} f
  }
  =: (df)_a
\end{align*}
In the example of $A = k[x]$, one sees that
\[
  (df)_a(X) = X(f) = f^\prime(a) X(x)
  = f^\prime(a) (dx)_a(X)
\]
in other words
\[
  (df)_a = f^\prime(a) (dx)_a
\]
\begin{dfn}[Algebraic one forms as dual of vector fields]
  
  Let $R \to A$ be a map of algebras.
  Then a \emph{module of differentials of $A$ relative $R$}
  is an $A$-module $\Om_{A / R}$ representing the 
  functor on $A$-modules $\mathrm{Der}_R(A , \_)$.
  In other words, $\Om_{A / R}$ is equipped with
  an isomorphism 
  \[
    A\MOD(\Om_{A / R} , \_) \simeq \mathrm{Der}_R(A , \_)
  \]
\end{dfn}
This definition can be seen as
the algebraic formulation of the idea that
one forms are dual of tangent vectors.
This is somewhat unsatisfying because 
it would be nice to have a way of thinking about one forms
independent of tangent vectors.

Given a point $a : A \to k$, 
we have the following interpretation.
\[
  (df)_a := f - f(a) \text{ mod }I(a)^2
\]
To get one forms with domains larger than a point,
we need to generalise the above formula to $a : A \to B$.
This is an issue because there is no canonical copy of $B$ inside $A$
for general $B$ unlike $B = k$.

% We rewrite the formula by considering the map 
% \[
%   a \otimes \id : A \otimes A \to A , f \otimes g \mapsto f(a) g
% \]

Let $I_\De$ be the kernel of $A \otimes A \to A , f \otimes g \mapsto f g$.
Consider $I_\De$ as an $A$-module via the left component of $A$.
If $f_0 \otimes g_0 + f_1 \otimes g_1$ is in $I_\De$,
then 
\[
  f_0 g_0 + f_1 g_1 = 0
\]
It follows that
\[
  f_0 \otimes g_0 + f_1 \otimes g_1
  = f_0 (1 \otimes g_0 - g_0 \otimes 1)
 + f_1 (1 \otimes g_1 - g_1 \otimes 1)
\]
This generalises to $f_0 g_0 + \cdots + f_n g_n$ in $I_\De$
and hows that $I_\De$ is generated as a left $A$-module
by elements of the form
\[
  1 \otimes f - f \otimes 1
\]
For the example of $A = k[t]$ and $A \otimes A = k[x , y]$,
elements like this look like 
\[
  f(y) - f(x)
\]
This inspires the choice of notation 
\[
  \De f := 1 \otimes f - f \otimes 1
\]
which in turn inspires the definition 
\[
  df := \De f \text{ mod } I_\De^2
\]
Let's check this is $k$-linear.
For $\la \in k$ and $f \in A$,
we have
\begin{align*}
  d(\la f) = 1 \otimes \la f - \la f \otimes 1
  = \la (1 \otimes f - f \otimes 1) = \la (d f)
\end{align*}
So $k$-linearity of $d$ comes from
the fact that $k$ is allowed to pass between the two sides in $A \otimes A$.

\begin{dfn}[Algebraic one forms as linear change]
  
  For a map of algebras $R \to A$,
  let $I_\De$ be the kernel of multiplication $A \otimes_R A \to A$.
  Then define 
  \[
    \Om_{A / R} := I_\De / I_\De^2
  \]
  This is a left $A$-module via $A \otimes 1 \subs A \otimes A$.
  Define \[
    d : A \to \Om_{A / R} , f \mapsto (1 \otimes f - f \otimes 1) 
    \text{ mod } I_\De^2 
  \]
  This map is only $R$-linear.
\end{dfn}

Our computation earlier shows that
$\Om_{A / R}$ is generated as an $A$-module by
the image of $d$.

Relation between the two definitions?
Something something universal square zero extension.

\end{document}