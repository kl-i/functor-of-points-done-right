\documentclass[./main.tex]{subfiles}
\begin{document}

We give intuition for fiber products of affine schemes.
To do this, we first give intuition for fiber products of sets.
A fiber product generalises the following notions : 
\begin{enumerate}
  \item Given a map $f : X \to Y$ of sets,
  then subsets $V$ of $Y$ pullback to subsets $f^{-1}V$ of $X$.
  \item Given two sets $X , Y$ ,
  we can form the product $X \times Y$.
\end{enumerate}
To describe fiber products,
it is convenient to realise the following tautology : 
a map of sets $X \to S$ can equivalently be thought of as
a family of sets $X_s$ parameterised by points $s \in S$
where $X_s$ are the fibers of the map above $s$.
This is called the \emph{relative point of view}
and it was first realized by Grothendieck.
To emphasize that when we are thinking of $X \to S$
as a family of sets over $S$,
one often draws the arrow \emph{vertically}.
\begin{cd}
  X \\
  S 
  \arrow[from = 1-1 , to = 2-1]
\end{cd}
With this, we are ready for the fiber product.
\begin{dfn}[Fiber product of sets]
  
  Let $S$ be a set and $X \to S , Y \to S$ be families of sets over $S$.
  We define their \emph{fiber product} as
  \[
    X \times_S Y := \coprod_{s \in S} X_s \times Y_s
  \]
  This has a natural projection to $S$
  making it a family of sets over $S$ with
  fiber over $s$ giving $X_s \times Y_s$.
\end{dfn}

\begin{ex}
  Show that we can alternatively define 
  \[
    X \times_S Y := 
    \set{ (x , y) \in X \times Y \st x = y \text{ when projected to } S}  
  \]
  Projection into the $X$ and $Y$ factors gives
  maps \[
    X \leftarrow X \times_S \rightarrow Y  
  \]
  By definition, the two are equal when composed with
  $X \to S , Y \to S$ respectively
  and thus defines a morphism $X \times_S Y \to S$.
\end{ex}

\begin{prop}[Universal property of fiber product of sets]
  
  Let $S$ be a set and $X \to S , Y \to S$ be families of sets over $S$.
  Define \[
    X \times_S Y \to X
  \] by doing
  the projection $X_s \times Y_s \to X_s$ for each fiber above $s \in S$.
  We can similarly define $X \times_S Y \to Y$.
  Then we have a commutative square \begin{cd}
    X & {X \times_S Y} \\
    S & Y
    \arrow[from=1-1, to=2-1]
    \arrow[from=1-2, to=1-1]
    \arrow[from=1-2, to=2-2]
    \arrow[from=2-2, to=2-1]
  \end{cd}
  and furthermore,
  for any $T \to S$ together with maps
  $x : T \to X$ and $y : T \to Y$ over $S$,
  there exists a unique map $(x , y) : T \to X \times_S Y$
  commuting with the two projections to $X, Y$.
  \begin{cd}
    & & T \\
    X & {X \times_S Y} \\
    S & Y
    \arrow[from=2-1, to=3-1]
    \arrow[from=2-2, to=2-1]
    \arrow[from=2-2, to=3-2]
    \arrow[from=3-2, to=3-1]
    \arrow["x"', bend right = 30, from=1-3, to=2-1]
    \arrow["y", bend left = 30, from=1-3, to=3-2]
    \arrow["{(x , y)}"{description}, dashed, from=1-3, to=2-2]
  \end{cd}

\end{prop}
\begin{proof}
  An easy exercise.
\end{proof}

Given a commutative square \begin{cd}
  X & T \\
	S & Y
	\arrow[from=1-1, to=2-1]
	\arrow[from=1-2, to=1-1]
	\arrow[from=1-2, to=2-2]
	\arrow[from=2-2, to=2-1]
\end{cd}
such that the induced map $T \to X \times_S Y$ is an isomorphism,
we denote \begin{cd}
  X & T \\
	S & Y
	\arrow[from=1-1, to=2-1]
	\arrow[from=1-2, to=1-1]
	\arrow[from=1-2, to=2-2]
	\arrow[from=2-2, to=2-1]
	\arrow["\lrcorner"{anchor=center, pos=0.125, rotate=-90}, draw=none, from=1-2, to=2-1]
\end{cd}
and say that we have a \emph{cartesian square}.

Using the universal property,
it is easy to verify the following.
(It is also easy to directly provide an isomorphism.)

\begin{eg}[Pullback of subsets as fiber products]
  
  Let $f : X \to Y$ be a map of sets.
  For the inclusion of a subset $V \subs Y$,
  we have an isomorphism of sets over $X$ \[
    f^{-1}V \simeq X \times_Y V  
  \]
\end{eg}

\begin{eg}[Products as fiber products]
  
  For $X \to S$ and $Y \to S$ where $S$ is singleton,
  we have \[
    X \times_S X \simeq X \times Y  
  \]
\end{eg}

To deal with fiber products in a general category such as $\AFF_k$,
one reverses the above and defines
fiber products by the universal property.
First, we introduce terminology for the relative point of view.

\begin{dfn}
  
  Let $S$ be an affine scheme over $k$.
  Then \emph{an affine scheme over $S$} is defined as
  a morphism $T \to S$ in $\AFF_k$.
\end{dfn}

Meta remark : 
The ``logically correct'' thing to do that we could have done
at the beginning of our theory is to start by defining
the category of affine schemes $\AFF$ as
opposite to the category of commutative rings,
not over any base field in particular.
Then for every field $k$,
the category of affine schemes over $\SPEC k$
is equivalent to what we have called $\AFF_k$.
Of course, this was inconvenient for our
narrative of commutative algebras as formalising physical systems
which is why we started with $\AFF_k$.
We hope the reader will forgive us.
We will henceforth work with $\AFF$ as our basic category of affine schemes.
All the results of closed embeddings and basic opens still hold
without a base field by the same arguments.
Affine $n$-space $\bA^n$ is defined instead as $\bZ[x_1 , \dots , x_n]$
and its universal property is similarly proved.

\begin{dfn}[Fiber products in $\AFF$]

  Let $X \to S , Y \to S$ be affine schemes over an affine scheme $S$.
  A \emph{fiber product of $X , Y$ over $S$} is
  an affine scheme $X \times_S Y$ together with
  morphisms $X \times_S Y \to X , Y$ such that we have
  a commutative square \begin{cd}
    X & {X \times_S Y} \\
    S & Y
    \arrow[from=1-1, to=2-1]
    \arrow[from=1-2, to=1-1]
    \arrow[from=1-2, to=2-2]
    \arrow[from=2-2, to=2-1]
  \end{cd}
  and furthermore,
  for any $T \to S$ together with maps
  $x : T \to X$ and $y : T \to Y$ over $S$,
  there exists a unique map $(x , y) : T \to X \times_S Y$
  commuting with the two projections to $X, Y$.
  \begin{cd}
    & & T \\
    X & {X \times_S Y} \\
    S & Y
    \arrow[from=2-1, to=3-1]
    \arrow[from=2-2, to=2-1]
    \arrow[from=2-2, to=3-2]
    \arrow[from=3-2, to=3-1]
    \arrow["x"', bend right = 30, from=1-3, to=2-1]
    \arrow["y", bend left = 30, from=1-3, to=3-2]
    \arrow["{(x , y)}"{description}, dashed, from=1-3, to=2-2]
  \end{cd}
\end{dfn}

We give examples of fiber products.

\begin{eg}[Zero locus is the preimage of zero]
  
  Let $i$ denote the closed embedding \[
    \set{0} \to \bA^1
  \]
  corresponding to the quotient map $\bZ[t] \to \bZ[t] / (t)$.
  Let $f \in A$ where $A$ is a ring.
  Reinterpreting $f$ as a map $\SPEC A \to \bA^1$,
  we can consider the fiber product : 
  \begin{cd}
    {\mathrm{Spec}\,A} & {?} \\
    {\mathbb{A}^1} & {\{0\}}
    \arrow["f"', from=1-1, to=2-1]
    \arrow[from=1-2, to=1-1]
    \arrow[from=1-2, to=2-2]
    \arrow["i", from=2-2, to=2-1]
    \arrow["\lrcorner"{anchor=center, pos=0.125, rotate=-90}, draw=none, from=1-2, to=2-1]
  \end{cd}
  Unraveling the definition of fiber products,
  we are trying to find a ring $B$ together with a commutative diagram
  \begin{cd}
    A & B \\
    {\mathbb{Z}[t]} & {\mathbb{Z}[t] / (t)}
    \arrow["{t \mapsto f}", from=2-1, to=1-1]
    \arrow["{t \mapsto t}"', from=2-1, to=2-2]
    \arrow[from=1-1, to=1-2]
    \arrow[from=2-2, to=1-2]
  \end{cd}
  such that 
  for all algebras $C$ with maps
  $A \to C$ and $\bZ[t]/(t) \to C$ such that
  both agree when restricted to $\bZ[t]$,
  there exists a unique map $B \to C$ making the following diagram commute :
  \begin{cd}
    && C \\
    A & B \\
    {\mathbb{Z}[t]} & {\mathbb{Z}[t] / (t)}
    \arrow["{t \mapsto f}", from=3-1, to=2-1]
    \arrow["{t \mapsto t}"', from=3-1, to=3-2]
    \arrow[from=2-1, to=2-2]
    \arrow[from=3-2, to=2-2]
    \arrow[from=2-1, to=1-3, bend left = 20]
    \arrow[from=3-2, to=1-3, bend right = 20]
    \arrow["{\exists !}"{description}, dashed, from=2-2, to=1-3]
  \end{cd}
  The fact that the square for $B$ commutes
  is equivalent to the fact that under $A \to B$,
  the element $f$ becomes zero.
  The map $\bZ[t] / (t) \to B$ is completely determined
  by the map $\bZ[t] \to A \to B$.

  Similarly, the map $\bZ[t] / (t) \to C$ is uniquely determined by the map
  $\bZ[t] \to A \to C$ and exists iff
  $\bZ[t] \to A \to C$ sends $t$ to zero in $C$,
  which happens iff $f \in A$ is sent to zero in $C$.
  This shows that the universal property of $B$ as a fiber product
  is equivalent to the universal property of $A \to B$,
  which is plainly the universal property of the quotient map 
  $A \to A / (f)$.
  This shows that we have a cartesian square   
  \begin{cd}
    {\mathrm{Spec}\,A} & {\SPEC A / (f)} \\
    {\mathbb{A}^1} & {\{0\}}
    \arrow["f"', from=1-1, to=2-1]
    \arrow[from=1-2, to=1-1]
    \arrow[from=1-2, to=2-2]
    \arrow["i", from=2-2, to=2-1]
    \arrow["\lrcorner"{anchor=center, pos=0.125, rotate=-90}, draw=none, from=1-2, to=2-1]
  \end{cd}
  In other words, 
  \emph{the zero locus of $f$ is the preimage of
  0 in $\bA^1$ under the reinterpretation of $f$ as a map
  $\SPEC A \to \bA^1$}.

\end{eg}

\begin{eg}[Basic open is the preimage of $\bA^1 \setminus \set{0}$]
  
  Let $j$ denote the inclusion of the basic open \[
    \bA^1\setminus\set{0} := D(t) \to \bA^1 
  \]
  corresponding to the ring map $\bZ[t] \to \bZ[t , t^{-1}]$.
  Let $f \in A$ where $A$ is a ring.
  Reinterpreting $f$ as a map $\SPEC A \to \bA^1$,
  one can similarly show that we have a cartesian square :
  \begin{cd}
    {\mathrm{Spec}\,A} & {D(f)} \\
    {\mathbb{A}^1} & {D(t)}
    \arrow["f"', from=1-1, to=2-1]
    \arrow[from=1-2, to=1-1]
    \arrow[from=1-2, to=2-2]
    \arrow["j", from=2-2, to=2-1]
    \arrow["\lrcorner"{anchor=center, pos=0.125, rotate=-90}, draw=none, from=1-2, to=2-1]
  \end{cd}
\end{eg}

\begin{eg}
  
  An easy exercise is to show that $\bA^2$ with its two projections
  to $\bA^1$ is a fiber product of $\bA^1 , \bA^1$ over
  $\SPEC \bZ$.
\end{eg}

\begin{eg}[A family of pairs of points]
  
  For this example, let us work over an algebraically closed field $k$,
  which means we are working in $\AFF_k$.
  We will further assume $k$ has characteristic not equal to two.
  Consider the following morphism
  \[
    \varphi : \bA^1_k \to \bA^1_k , t \mapsto t^2
  \]
  corresponding to the $k$-algebra map
  $k[t] \to k[t] , t \mapsto t^2$.
  We have an identitification : 
  \[
    \bA^1_k(k) \simeq k , \la \mapsto t(\la)  
  \]
  For each $\la \in k$ let $\la$ also denote
  the morphism $\SPEC k \to \bA^1_k$ of affine schemes over $k$.
  Let us investigate the fibers of $\varphi$ over $\la$,
  or in other words,
  give a fiber product \begin{cd}
    {\bA^1_k} & {\SPEC A_\la} \\
    {\bA^1_k} & {\SPEC k}
    \arrow["{t^2}"', from=1-1, to=2-1]
    \arrow[from=1-2, to=1-1]
    \arrow[from=1-2, to=2-2]
    \arrow["{\la}", from=2-2, to=2-1]
    \arrow["\lrcorner"{anchor=center, pos=0.125, rotate=-90}, draw=none, from=1-2, to=2-1]
  \end{cd}
  Unraveling the definition of fiber products,
  one can see that $A_\la$ is supposed to have
  an element $f$ with $f^2 - \la = 0$ and
  $A_\la$ is supposed to be the universal such $k$-algebra.
  $A_\la = k[t] / (t^2 - \la)$ does the trick.
  Indeed, we have a commutative square \begin{cd}
    {k[t]} & {k[t] / (t^2 - \lambda)} \\
    {k[t]} & k
    \arrow["{t \mapsto t^2}", from=2-1, to=1-1]
    \arrow["{t \mapsto \lambda}"', from=2-1, to=2-2]
    \arrow[from=1-1, to=1-2]
    \arrow[from=2-2, to=1-2]
  \end{cd}
  and the universal property is equivalent to
  the universal property of the quotient map
  $k[t] \to k[t] / (t^2 - \la)$.

  The interesting thing here is that for $\la \neq 0$,
  since $k$ is algebraically closed and characteristic not two,
  we have a factoring \[
    t^2 - \la = (t - a) (t + a)
  \]
  with $a \neq - a \in k$.
  By the chinese remainder theorem, we then have \[
    k[t] / (t^2 - \la) \simeq k[t] / (t - a) \times k[t] / (t + a)  
  \]
  From this it follows that $\SPEC A_\la$ has
  exactly two $k$-points corresponding to $\pm a$.

  When $\la = 0$ we do not have such a factoring.
  We have \[
    \SPEC A_\la = \SPEC k[t] / (t^2)  
  \]
  which only has a single $k$-point corresponding to $t = 0$.
  So the map \begin{cd}
    {\bA^1_k} \\
    {\bA^1_k}
    \arrow[from = 1-1, to = 2-1 , "{t^2}"]
  \end{cd}
  behaves almost like a family of two points at the level of $k$-points.
  It only fails at the origin.
  \textbf{However},
  if instead of counting the number of $k$-points we counted
  the \emph{dimension} of $A_\la$ as a vector space over $k$,
  then we would get a constant number two.
  Indeed, we will see much later that
  this is a consequence of the fact
  \[
    k[t] = k[t^2] \oplus t k[t^2] 
  \]
  as modules over $k[t^2]$ and hence $\varphi : k[t] \to k[t]$
  exhibits the latter as a free rank 2 module over $k[t]$.
  So in this sense,
  we do have a family of two points;
  it's just that sometimes the two points are ``on top of each other''.
  In algebraic geometry, this is called a \emph{double point}.
  This shows that \emph{nilpotents} are useful
  for keeping track of ``degenerate cases''
  such as $\la = 0$ in this example.
\end{eg}

For fiber products in general, we have the following :

\begin{prop}[Fiber products exist in affine schemes.]
  
  Let $A \to B , A \to C$ be algebra maps.
  Then a fiber product of $\SPEC B , \SPEC C$ over $\SPEC A$
  is given by $\SPEC (B \otimes_A C)$
  where $B \otimes_A C$ is the tensor product of
  $B$ and $C$ over $A$ as rings.
\end{prop}
\begin{proof}
  This is a restatement of universal property of the tensor product of rings.
\end{proof}

We have secretly danced around an important question : 
\[
  \text{\emph{Given two fiber products of the same diagram,
  are they isomorphic?}}  
\]
More generally, we have seen many examples of objects with
universal properties by now.
\[
  \text{\emph{Given two objects with the same universal properties,
  are they isomorphic?}}  
\]
The answer is given by \emph{Yoneda's lemma}
but we will delay the discussion to a later section
so that we can give more examples of ``algebra capturing geometry''.

\end{document}