\documentclass[./main.tex]{subfiles}
\begin{document}

There are two examples of spaces we want which are non-affine.
\begin{enumerate}
  \item $\bA^2 \setminus \set{0}$, the complement of 
  the closed embedding $i : \set{0} \to \bA^2$
  corresponding to the quotient map
  $\bZ[x , y] \to \bZ[x , y] / (x , y)$.
  \item Projective space $\bP^1$.
  This should be obtained as the coequalizer 
  \[
    D(t) \rightrightarrows \bA^1 + \bA^1 \to \bP^1  
  \]
  where we have $D(t) \to \bA^1$ via $t \mapsto t$ and $t \mapsto 1 / t$.
\end{enumerate}
Let us discuss (1) first.
We will discuss (2) in the next section.
The punchline is that we have a Goldielocks situation : 
\begin{enumerate}
  \item $\bA^2 \setminus \set{0}$ should be
  the union of $D(x)$ and $D(y)$ inside $\bA^2$.
  This computation \emph{does not work} in $\AFF$ so we move to $\PSH \AFF$.
  \item The union in $\PSH \AFF$ does not give what we want either.
  The reason is that the union in presheaves \emph{``does not see'' all 
  Zariski covers}.
  Restricting to the ones that do brings us to \emph{Zariski sheaves}.
  This is more than half of the definition of \emph{schemes},
  the manifolds of algebraic geometry.
\end{enumerate}
\begin{dfn}[Complements]
  
  Let $i : Z \to X$ be a morphism in $\AFF$.
  Define the \emph{complement of $i$} as the subpresheaf of $X$
  \[
    (X \setminus Z) (T) := 
    \set{x \in X(T) \st T \times_X Z = \nothing}
  \]
  We say a map $j : U \to X$ in $\AFF$
  is an affine complement of $i$ when 
  $j$ induces an isomorphism \[
    U \simeq X \setminus Z  
  \]
  We say $i$ has an complement representable by an affine scheme.
\end{dfn}
From the previous section,
it is clear that if an affine complement exists, 
it is unique up to isomorphism.
What is not clear is that affine complements exist.
In fact, it is not true.
\begin{prop}[Union computed in $\AFF$ is no good]
  
  The closed embedding $i : \set{0} \to \bA^2$ has no affine complement.
\end{prop}
\begin{proof}
  
  Suppose $j : \SPEC C \to \bA^2$ is an affine complement of $i$.
  Under Yoneda, $j$ is given by the data of $(f , g) \in C \times C$
  such that $C / (f , g) = 0$.
  We then have the following : 
  \begin{cd}
    {\mathrm{Spec}\,C} & {\mathrm{Spec}\,C[1 / f] + \mathrm{Spec}\,C[1 / g]} & {\mathrm{Spec}\,C[1 / fg]} \\
    {\mathbb{A}^2} & {D(x) + D(y)} & {D(x y)}
    \arrow[from=1-1, to=2-1]
    \arrow[from=1-2, to=1-1]
    \arrow[from=1-2, to=2-2]
    \arrow[shift right, from=1-3, to=1-2]
    \arrow[shift left, from=1-3, to=1-2]
    \arrow[from=1-3, to=2-3]
    \arrow[from=2-2, to=2-1]
    \arrow[shift right, from=2-3, to=2-2]
    \arrow[shift left, from=2-3, to=2-2]
  \end{cd}
  where the top row is obtained by pulling back the bottow row across $j$.
  Now something interesting happens : 
  we can show that $\SPEC C[1 / f] \to D(x)$ is actually an isomorphism.
  To see this, note that $x$ is invertible in $k[x , y , 1 / x]$ so
  the inclusion $D(x) \to \bA^2$ factors through $j$
  by virtue of $\SPEC C = \bA^2 \setminus \set{0}$.
  \begin{cd}
    {\mathrm{Spec}\,C} && \rightsquigarrow & {\mathrm{Spec}\,C} & {D(x)} \\
    {\mathbb{A}^2} & {D(x)} && {\mathbb{A}^2} & {D(x)}
    \arrow["j"', from=1-1, to=2-1]
    \arrow[from=2-2, to=2-1]
    \arrow["s"', dashed, from=2-2, to=1-1]
    \arrow["j"', from=1-4, to=2-4]
    \arrow["s"', from=1-5, to=1-4]
    \arrow["{\mathrm{id}}", from=1-5, to=2-5]
    \arrow[from=2-5, to=2-4]
    \arrow["\lrcorner"{anchor=center, pos=0.125, rotate=-90}, draw=none, from=1-5, to=2-4]
  \end{cd} 
  The universal property of $j : \SPEC C \to \bA^2$
  implies that $\SPEC C \to \bA^2$ is a monomorphism.
  Using the final proposition of the previous section,
  we obtain $\SPEC C[1/f] = \SPEC C \times_{\bA^2} D(x) \simeq D(x)$.
  The same goes for $D(y)$ and $D(xy)$.
 
  So the middle and right vertical morphisms are isomorphisms.
  By assumption $C / (f , g) = 0$
  i.e. $(f , g) = (1)$ so
  $D(f) + D(g)$ gives a {Zariski cover} of $\SPEC C$.
  Putting it all together with the
  fundamental lemma of basic Zariski covers, 
  we have the following commutative diagram of rings : 
  \begin{cd}
    C & {C[1/f] \times_{C[1/fg]} C[1/g]} \\
    {k[x,y]} & {k[x,y , 1 / x] \times_{k[x,y , 1 / x , 1 / y]} k[x,y,1.y]}
    \arrow["{x,y\mapsto f,g}", from=2-1, to=1-1]
    \arrow["{\sim}",from=2-1, to=2-2]
    \arrow["\sim", from=1-1, to=1-2]
    \arrow["\sim"', from=2-2, to=1-2]
  \end{cd}
  The right morphism is an isomorphism by the previous paragraph
  and a simple computation shows 
  the bottom morphism is also an isomorphism.
  It follows that $j$ is an isomorphism.
  But this is a contradiction since taking any field $k$
  \begin{cd}
    {(\SPEC \, C) (k)} & {k^2\setminus\set{0}} \\
    {\bA^2(k)} & {k^2}
    \arrow["j"', from=1-1, to=2-1]
    \arrow["\sim", from=1-1, to=2-1]
    \arrow["\sim", from=2-2, to=2-1]
    \arrow["\sim"', from=1-2, to=1-1]
    \arrow["\subseteq"', from=1-2, to=2-2]
    \arrow["\nsim", from=1-2, to=2-2]
  \end{cd}
\end{proof}
This shows that if we want \emph{open embeddings} to make sense
beyond basic opens,
we need to allow ``non-affine spaces'' i.e. presheaves.
More generally,
\begin{dfn}[Open embedding with affine target]

  Let $I \subs A$ be an ideal.
  Then the complement of $\SPEC A / I \to \SPEC A$
  is defined as the subfunctor of the Yoneda embedding of $\SPEC A$
  consisting of maps $\SPEC B \to \SPEC A$
  such that $B / I B = 0$.

  A map $\SPEC B \to \SPEC A$ is called an \emph{open embedding}
  when it is the complement of a closed embedding into $\SPEC A$.
\end{dfn}
We give some more properties of the presheaf category over $\AFF$.
\begin{prop}[How to compute limits/colimits in presheaves]
  
  The following are true : 
  \begin{enumerate}
    \item $\PSH \AFF$ has all limits and it is computed
    ``pointwise'' i.e.
    for a diagram $X : I \to \PSH \AFF$ we have
    functorially in $S \in \AFF$
    \[
      (\LIM_{i \in I} X_i) (S) \map{\sim}{} \LIM_{i \in I} X_i(S)
    \]
    The same is true for colimits.
    \item All epimorphisms are effective epimorphisms,
    i.e. ``a surjection is the quotient of its source by 
    the equivalence relation of being in same fibers''.
    \item Limits and colimits of subobjects are computed pointwise
    by intersection and union.
    \item The Yoneda embedding preserves limits.
    However it does \emph{not} preserve colimits.
    % \item $\PSH \AFF$ is cartesian closed i.e.
    % we can consider ``mapping spaces''.
  \end{enumerate}
\end{prop}
None of the above are hard to prove,
but let us say something about the Yoneda embedding not preserving colimits.
At first, this may appear to be a bad thing
however this is in fact desireable.
Indeed, throughout the proof that $\bA^2 \setminus \set{0}$ 
cannot be affine, we proved that we have an coequalizer diagram
\[
  D(x y) \rightrightarrows D(x) + D(y) \to \bA^2  
\]
inside the category $\AFF$.
This is bad because this coequalizer should be $\bA^2 \setminus\set{0}$
and we do not want the inclusion $\bA^2 \setminus \set{0} \subs \bA^2$
to be an isomorphism!

Let us investigate further.
Note that the inclusions $D(x) , D(y) \subs \SPEC A$
factor through $\bA^2\setminus\set{0}$
so it makes sense to ask whether the following diagram
in $\PSH \AFF$ is a coequalizer diagram
\[
  D(x y) \rightrightarrows D(x) + D(y) \to \bA^2 \setminus \set{0}
\]
\begin{prop}[Union in presheaves also no good]
  
  Let $U$ denote the coequalizer of $D(x y) \rightrightarrows D(x) + D(y)$
  inside $\PSH \AFF$.
  This is a subobject of $\bA^2$ but it is \emph{not}
  isomorphic to $\bA^2\setminus\set{0}$ as subobjects of $\bA^2$.
\end{prop}
\begin{proof}
  Because $D(xy) , D(x) , D(y)$ are all subfunctors of $\bA^2$,
  the coequalizer is simply the union.
  In other words, for $\SPEC A \in \AFF$ 
  \[
    U(A) = \set{(f , g) \in A^2 \st f \in A^\times \text{ or } g \in A^\times}
  \]
  Since $D(x) , D(y) \subs \bA^2\setminus\set{0}$
  we have an induced inclusion $U \subs \bA^2\setminus\set{0}$.
  However this is not an equality.
  Indeed for $A = k[t] , f = t , g = 1 - t$
  we have that neither $t , 1 - t$ are units but
  $(1) = (t , 1 - t)$.
\end{proof}
One way of understanding the above phenomenon is that
given $(f , g) : \SPEC A \to \bA^2\setminus\set{0}$ we have 
a basic Zariski cover
\[
  D(f g) \rightrightarrows D(f) + D(g) \to \SPEC A  
\]
which is a coequalizer diagram in $\AFF$,
however this is \emph{not} a coequalizer diagram in $\PSH \AFF$.
Specifically, $U$ can fail to recognise $\SPEC A$ as the coequalizer.
Indeed, with the example in the proof,
we have a map $D(f) + D(g) \to U$ which agrees on the
intersection $D(fg)$ but fails to descend 
to a map $\SPEC A \to U$.
% More precisely, let $S_{\set{f , g}}$ be the coequalizer
% of $D(f g) \rightrightarrows D(f) + D(g)$ inside $\PSH \AFF$.
% This induces a map
% \[
%   S_{\set{f , g}} \to \SPEC A  
% \]
% and $U$ fails to recognise this as an isomorphism.
It turns out,
trying to force $U$ to recognise these coequalizers 
turns it into $\bA^2\setminus\set{0}$.
Making this idea rigorous leads us to the notion of a \emph{Zariski sheaf}.
% We first abstract the above example.
% \begin{prop}[Zariski sieves]
  
%   Let $X = \SPEC A \in \PSH \AFF$
%   and let $I \subs A$ be finite with $(I) = A$.
%   Define the Zariski sieve associated $I$ as
%   the subfunctor $S_I \subs \SPEC A$
%   of points $x : \SPEC B \to \SPEC A$
%   such that there exists a factoring for some $f \in I$
%   \[
%     \SPEC B \to \SPEC A[1 / f] \to \SPEC A
%   \]
%   Then we have a coequalizer diagram of presheaves
%   \[
%     \cU_I \times_X \cU_I \rightrightarrows \cU_I \to S_I
%   \]
%   where $\cU_I := \coprod_{f \in I} \SPEC A[1 / f]$.
% \end{prop}
% \begin{proof}
%   Colimits of presheaves are computed fiberwise.
%   We are thus reduced to showing
%   \[
%     \cU_I(B) \times_{X(B)} \cU_I(B) \rightrightarrows \cU_I(B) \to S_I(B)
%   \]
%   is a coequalizer diagram for each $\SPEC B \in \AFF$.
%   This follows because $S_I(B)$ is precisely the image of
%   $U_I(B)$ and for sets,
%   the image of a map is the coequalizer of its kernel pair.
% \end{proof}
\begin{prop}[Zariski sheaves]

  Define $\SH_\ZAR \AFF$ to be the full subcategory of $\PSH \AFF$
  of presheaves $X$ such that for all 
  basic Zariski covers $\SPEC \prod_{i = 1}^n A[1 / f_i] \to \SPEC A$
  of affines,
  $X$ sees the cover as an effective epimorphism,
  or in other words,
  restriction along the cover gives an equalizer of sets
  \[
    X(A) \to \prod_{i = 1}^n X(A[1 / f_i]) \rightrightarrows
    \prod_{i, j = 1}^n X(A[1 / f_i f_j])
  \]
  where we have used Yoneda's lemma.
  Such presheaves are called \emph{Zariski sheaves}.
  \footnote{
    Our definition matches
    \cite[\href{https://stacks.math.columbia.edu/tag/020W}{Tag 020W}]{stacks-project}.
  }
  Then 
  \begin{enumerate}
    \item We have an adjunction 
    \begin{cd}
      {\mathrm{Sh}_{\mathrm{Zar}} \mathrm{Aff}} & {\mathrm{PSh} \mathrm{Aff}}
      \arrow["L"', shift right=2, from=1-2, to=1-1]
      \arrow["\subseteq"', shift right=2, from=1-1, to=1-2]
      \arrow["\bot"{description}, draw=none, from=1-1, to=1-2]
    \end{cd}
    \item the left adjoint $L$ is commutes with finite limits,
    equivalently sends the final object to the final object and
    preserves (finite) fiber products.
    \item for all presheaves $X$ and $x : S \to LX$ where $S$ is affine,
    there exists a basic Zariski cover $\cU$ of $S$ with a map
    $\tilde{x} : \cU \to X$ such that we have a commuting square 
    \begin{cd}
      {\mathcal{U}} & X \\
      S & LX
      \arrow[from=1-2, to=2-2]
      \arrow[from=1-1, to=2-1]
      \arrow["x"', from=2-1, to=2-2]
      \arrow["{\tilde{x}}", from=1-1, to=1-2]
    \end{cd}
  \end{enumerate}
\end{prop}
\begin{proof}
  We blackbox this result because the proof,
  though perhaps long, is straightful and does not
  use any geometric ideas.
\end{proof}
The functor $L$ is called (Zariski) sheafification.
Note that the lemma of Zariski covers shows that
all affine schemes are Zariski sheaves.
To complete the example of $\bA^2\setminus\set{0}$,
we have
\begin{prop}[Finally, union in Zariski sheaves is good]
  
  The inclusion $j : U \subs \bA^2\setminus\set{0}$ exhibits
  the latter as the Zariski sheafification of $U$.
\end{prop}
\begin{proof}
  Let $Y \in \SH\AFF$.
  We need to show that $\bA^2 \setminus \set{0}$ is a Zariski sheaf and 
  $j$ induces
  \[
    \SH_\ZAR \AFF (\bA^2\setminus\set{0} , Y)
    \map{\sim}{} \PSH\AFF(U , Y)
  \]
  (injectivity)
  Let $\varphi , \psi : \bA^2\setminus\set{0} \to Y$
  such that $\res{\varphi}{U} = \res{\psi}{U}$.
  To show $\varphi = \psi$,
  the uniqueness part of the density theorem says it suffices to show that for all
  $(f , g) : \SPEC A \to \bA^2\setminus\set{0}$
  we have $\varphi(f , g) = \psi(f , g)$.
  We have the commutative diagram
  \begin{cd}
    {\mathrm{Spec}\,A} & {D(f) + D(g)} & {D(fg)} \\
    Y & U
    \arrow[from=1-2, to=1-1]
    \arrow[shift right, from=1-3, to=1-2]
    \arrow[shift left, from=1-3, to=1-2]
    \arrow["{\varphi(f ,g)}"', from=1-1, to=2-1]
    \arrow[from=1-2, to=2-2]
    \arrow[from=2-2, to=2-1]
  \end{cd}
  Since $Y$ is a Zariski sheaf,
  it sees $\SPEC A$ as the coequalizer of
  $D(f g) \rightrightarrows D(f) + D(g)$.
  In particular,
  $\varphi(f , g)$ is determined by its restrictions to
  $D(f) , D(g)$.
  But $\res{\varphi(f , g)}{D(f)} = 
  \res{\brkt{\res{\varphi}{U}}}{D(f)}$
  and similarly for $D(g)$.
  The same goes for $\psi(f , g)$
  and thus $\varphi = \psi$.

  (surjectivity)
  Assume $\psi : U \to Y$ with the goal of showing
  this extends to a map $\bA^2\setminus\set{0} \to Y$.
  Again by the existence part of the density theorem, 
  the data of the such a map is equivalent to the following
  \begin{enumerate}
    \item for each point $(f , g) : \SPEC A \to \bA^2\setminus\set{0}$
    we give a point $\tilde{\psi}(f , g) : \SPEC A \to Y$
    \item for each $\varphi : \SPEC B \to \SPEC A$
    we have \[
      \tilde{\psi}(f , g)\varphi = \tilde{\psi}(f \varphi , g \varphi)
    \]
    \item When $(f , g)$ lies in $U$,
    we want $\tilde{\psi}(f , g) = \psi(f , g)$.
  \end{enumerate}
  Let $(f , g)$ be an $A$-point of $\bA^2\setminus\set{0}$.
  Again since $Y$ is a Zariski sheaf,
  it recognises $\SPEC A$ as the coequalizer of 
  $D(fg) \rightrightarrows D(x) + D(y)$.
  So to define $\tilde{\psi}(f , g)$ is equivalent to
  defining an element of $Y(A[1/f]) \times_{Y(A[1/fg])} Y(A[1/g])$.
  Again note that $D(f) , D(g) \to \bA^2\setminus\set{0}$ factor through $U$
  and agree on $D(f g)$.
  So we use the element 
  \[
    \brkt{\psi\brkt{\res{(f,g)}{D(f)}} , \psi\brkt{\res{(f,g)}{D(g)}}}  
  \]
  We leave (2) and (3) as an exercise.

  (Zariski sheaf) 
  A map $\SPEC A \to \bA^2\setminus\set{0}$
  is the same as $(f , g) \in A$ such that $(f , g) = (1)$.
  This is the same as an $A$-module map $A^2 \to A$ which is a surjection.
  Given a basic Zariski cover $U$ of $\SPEC A$,
  Zariski descent says the set of maps $A^2 \to A$
  bijects by restriction to the set of maps $\cO(U)^2 \to \cO(U)$
  compatible with descent data.
  Since Zariski descent is an equivalence of abelian categories,
  and cokernel on the cover is computed in each open,
  this shows that surjectivity of an $A$-module map
  is \emph{Zariski local}, which completes the proof.
\end{proof}

The above proof that $\bA^2\setminus\set{0}$ is a Zariski sheaf
is an example of computation with \emph{quasi-coherent sheaves}.
We will come back to this in chapter 3.

\begin{eg}
  Let $\SPEC A + \SPEC B$ be presheaf coproduct.
  Then this is \emph{not} a Zariski sheaf.
  (Exercise : find a counterexample.)
  However, its sheafification is given by $\SPEC (A \times B)$.
\end{eg}

% Throughout the above proof,
% we used the crucial observation that
% a map $\SPEC A \to LX$ lifts across $X \to LX$
% for a basic Zariski cover of $\SPEC A$.
% This generalises to the result below and 
% is useful for dealing with sections of sheafifications, 
% however we will omit the proof
% since it is not very interesting.
% \begin{prop}
  
%   Let $X \in \PSH \AFF$ and $\SPEC A \in \AFF$.
%   \begin{enumerate}
%     \item Given a point $x \in (LX) A$,
%     there exists a Zariski covering $\cU_1 \to \SPEC A$
%     with a map $\tilde{x} : \cU_1 \to X$
%     such that the following diagram commutes
%     \begin{cd}
%       {\mathcal{U}_2} & {\mathcal{U}_1 \times_{\mathrm{Spec} A} \mathcal{U}_1} & {\mathcal{U}_1} & {\mathrm{Spec}\,A} \\
%       && X & LX
%       \arrow[from=2-3, to=2-4]
%       \arrow["{\tilde{x}}", from=1-3, to=2-3]
%       \arrow["x", from=1-4, to=2-4]
%       \arrow[from=1-3, to=1-4]
%       \arrow[shift right=2, from=1-2, to=1-3]
%       \arrow[from=1-1, to=1-2]
%       \arrow[from=1-1, to=2-3]
%       \arrow[shift left=2, from=1-2, to=1-3]
%     \end{cd}
%     for some Zariski cover $\cU_2$ of $\cU_1 \times_{\SPEC A} \cU_1$.
%     \footnote{
%       Note that we do \emph{not} assume $\tilde{x}$
%       coequalizes $\cU_1 \times_{\SPEC A} \cU_1 \rightrightarrows \cU_1$.
%     }
%     \item Given a Zariski cover $\cU_1 \to \SPEC A$
%     and $\tilde{x} : \cU_1 \to X$ satisfying the above condition,
%     $\tilde{x}$ descends to a unique point $x : \SPEC A \to X$.
%   \end{enumerate}
% \end{prop}

Since $L$ is a left adjoint,
it follows that colimits of Zariski sheaves are
given by $L$ applied to the corresponding presheaf colimit.
However, Zariski sheafification is hard to compute,
blackboxed or not.
We have seen that effective epimorphisms
are very important kinds of colimits because they generalise
surjections from the quotients point of view.
We end this section with
a useful criterion for recognising
effective epimorphisms between Zariski sheaves.
In the category of sets,
a morphism is surjective iff it admits a section.
This is a form of the axiom of choice. 
The following is a version of this 
for effective epimorphisms between general Zariski sheaves.
\begin{prop}[Local axiom of choice]
  
  Let $p : X \to Y$ be an map in $\SH_\ZAR \AFF$.
  Then $p$ is an epimorphism iff it is an effective epimorphism 
  iff ``it locally has sections''
  i.e. for all points $y : \SPEC A \to Y$
  there exists a basic Zariski cover $\cU$ of $\SPEC A$ such that
  there exists a commutative diagram
  \begin{cd}
    {\mathcal{U}} & X \\
    {\SPEC\,A} & Y
    \arrow["p", from=1-2, to=2-2]
    \arrow["y"', from=2-1, to=2-2]
    \arrow[from=1-1, to=2-1]
    \arrow["{\exists \,x}", dashed, from=1-1, to=1-2]
  \end{cd}
\end{prop}
\begin{proof}
  
  Let $\im p$ denote the subfunctor of $Y$
  consisting of points $y \in Y(A)$ which admit local lifts across $p$.
  Whether $p$ is an epimorphism or not
  is about maps out of $Y$.
  Therefore we need a way of detecting whether $\im p = Y$
  using a map out of $Y$.
  The idea is to use indicator functions A.K.A.
  the existence of a subobject classifier.
  We omit the proof since it requires no geometric intuition.

  % Let $\PT$ denote the final object of $\PSH \AFF$.
  % It is represented by $\SPEC \bZ$
  % and it is also the final object of $\SH_\ZAR \AFF$.
  % Define $\PROP := \PT + \PT$ where the coproduct is taken in $\SH_\ZAR \AFF$.
  % By definition of Zariski sheaves, this is represented by 
  % $\SPEC \bZ \times \bZ \simeq \SPEC \bZ[t]/(t(1 - t))$
  % so a map $\SPEC A \to \PROP$ consist of giving an idempotent of $A$.
  % We use $\top$ to denote the inclusion $\PT \to \PROP$
  % corresponding the idempotent $1$ inside $\bZ$.

  % We now define the indicator function of $\im p$.
  % Define $\chi : Y \to \PROP$ by using the density theorem.
  % Again this means we need : 
  % \begin{enumerate}
  %   \item for each point $y : \SPEC A \to Y$
  %   to give an idempotent $\chi(y)$ of $A$
  %   \item proof that for $\varphi : \SPEC B \to \SPEC A$
  %   and $y : \SPEC A \to Y$ we have
  %   \[
  %    \chi(y)\varphi = \chi(y \varphi)  
  %   \]
  % \end{enumerate}



\end{proof}
% The issue is if we compute this in $\AFF$, we would get
% \[
%   k[t , 1 / t] \leftleftarrows k [t] \times k [1/t] \leftarrow 
%   k \simeq \cO(\bP^1)
% \]
% But certainly we do not want $\bP^1 \simeq \SPEC k$ !

% Let us write down some data we would want from $\bP^1$.
% \begin{enumerate}
%   \item For every $S \in \AFF_k$,
%   we want to able to consider maps $S \to \bP^1$.
%   \item The two maps $j_i : \bA^1 \to \bP^1$ should be \emph{open immersions}.
%   In particular, for all $S \in \AFF_k$,
%   we should be able to form the cartesian square
%   and have $U_i \to S$
% \end{enumerate}



\end{document}