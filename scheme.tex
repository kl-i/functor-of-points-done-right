\documentclass[./main.tex]{subfiles}
\begin{document}

Slogan : 
\emph{a scheme is the quotient of affines schemes 
along open immersions}.
For this, we need to define open immersions between Zariski sheaves.

% To get the idea of the rigorous definition,
% the question one needs to ask is : 
% \emph{what should a open cover $\cU \to X$ do?}

% Let $\cU$ be a set of affines and $X$ a presheaf.
% Defining what it means for $U \in \cU$ to have
% an open embedding $j_U : U \to X$ is easy enough :
% the pullback to any affine should be a subset
% which is the complement of a closed embedding.

% Let us abuse notation and denote $\coprod_{U \in \cU} U$ 
% again with $\cU$.
% The question to ask is : 
% how does giving a map $S \to X$ from an affine $S$
% relate to maps into $\cU$?
% The answer comes from considering the diagram 
% \begin{cd}
%   {\coprod_{i , j} S[1/f_i f_j]} & {\mathcal{V} \times_S \mathcal{V}} & {\mathcal{U} \times_X \mathcal{U}} \\
% 	{\coprod_{i} S[1/f_i]} & {\mathcal{V}} & {\mathcal{U}} \\
% 	S & S & X
% 	\arrow[from=3-2, to=3-3]
% 	\arrow[from=2-3, to=3-3]
% 	\arrow[from=2-2, to=3-2]
% 	\arrow[from=2-2, to=2-3]
% 	\arrow["\lrcorner"{anchor=center, pos=0.125}, draw=none, from=2-2, to=3-3]
% 	\arrow[from=3-1, to=3-2]
% 	\arrow[from=2-1, to=3-1]
% 	\arrow[shift left=2, from=1-1, to=2-1]
% 	\arrow[shift right=2, from=1-1, to=2-1]
% 	\arrow[from=2-1, to=2-2]
% 	\arrow[shift right=2, from=1-3, to=2-3]
% 	\arrow[shift left=2, from=1-3, to=2-3]
% 	\arrow[shift left=2, from=1-2, to=2-2]
% 	\arrow[shift right=2, from=1-2, to=2-2]
% 	\arrow[from=1-1, to=1-2]
% 	\arrow[from=1-2, to=1-3]
% \end{cd}
% The $\cU$ should pullback to $\cV$
% which we should be able to refine with standard Zariski opens.
% Now condition (2) should say
% that these standard Zariski opens should cover $S$,
% i.e. we have a partition of unity $1 = \sum_{i} \la_i f_i$.
% Furthermore, 
% the data of the map $S \to X$ should be equivalent to
% the map from the standard Zariski cover agree on intersections,
% i.e. $X$ \emph{should recognise $S$ as the coequaliser of
% the standard Zariski cover}.
% We arrive at a definition of schemes.

\begin{dfn}
  
  Let $j : X \to Y$ be a map of presheaves.
  Then $j$ is called an open immersion when
  for all points $y : \SPEC A \to Y$,
  the pullback $\SPEC A \times_Y X \to \SPEC A$
  is an open immersion.
\end{dfn}

We can now define schemes.
\begin{dfn}

  Let $X$ be a Zariski sheaf.
  An affine Zariski cover / atlas of $X$ is defined as
  a set of open immersions $\set{j_i : \SPEC A_i \to X}_{i \in I}$
  such that the Zariski disjoint union gives a Zariski epimorphism
  \[
    \coprod_{i \in I} \SPEC A_i \twoheadrightarrow X
  \]
  A scheme $X$ is a Zariski sheaf that admits
  an affine Zariski atlas.
\end{dfn}

The following justifies the slogan of this section.

\begin{prop}[Scheme = quotient of affines along open immersions]
  
  Let $X$ be a Zariski sheaf and let $\set{j_i : \SPEC A_i \to X}_{i \in I}$
  be a set of maps.
  Let $\cU := \coprod_{i \in I} \SPEC A_i$ be the Zariski disjoint union.
  Then the following are equivalent : 
  \begin{enumerate}
    \item $\set{j_i : \SPEC A_i \to X}_{i \in I}$ is a Zariski affine atlas
    for $X$.
    \item There is an equivalence relation $\cR$ on $\cU$ such that
    \begin{itemize}
      \item the projections $\cR \rightrightarrows \cU$ are open embeddings
      \item The map $\cU \to X$ exhibits $X$ as 
      the Zariski coequalizer of $\cR \rightrightarrows \cU$.
    \end{itemize}
  \end{enumerate}
\end{prop}

The proof requires some basic knowledge of the category of
Zariski sheaves $\SH_\ZAR \AFF \subs \PSH \AFF$.

\begin{proof}
  (1) implies (2) is clear.

  Let $\cU, \cR$ be as in the situation (2).
  It suffices to show for each affine scheme summand $U = \SPEC A_i$ of $\cU$ 
  the map $j_U : U \to \cU \to X$ is an open embedding.
  Let $\tilde{X}$ be the coequaliser of $\cR \rightrightarrows \cU$
  in the category of presheaves.
  Note that the Zariski sheafification of the coequaliser sequence
  for $\tilde{X}$ is the coequaliser sequence for $X$
  because Zariski sheafification is a left exact left adjoint.
  We need the following.
  \begin{lem}
    Let $f : X \to Y$ be a morphism of presheaves which is an open immersion.
    Then
    \begin{enumerate}
      \item if $Y$ is a Zariski sheaf, so is $X$.
      \item if $f$ is an open embedding then so is its Zariski sheafification.
    \end{enumerate}
    \begin{proof1}
      (1) Exercise.

      (2) Suppose we have $S \to L(Y)$ with $S$ affine.
      We need to show $S \times_{L(Y)} L(X) \to S$ is an open embedding.
      It suffices to do so after passing to some Zariski cover
      $\tilde{S} \to S$.
      One of the properties of Zariski sheafification is that
      we can find a Zariski cover $\tilde{S} \to S$
      such that $\tilde{S} \to S \to L(Y)$ factors as
      $\tilde{S} \to Y \to L(Y)$.
      The rest follows from (1) and the fact that
      Zariski sheafification $L$ is
      left exact.
      \begin{cd}
        {L(\tilde{S} \times_{Y}X)} & {\tilde{S} \times_{Y}X} & X \\
        {\tilde{S} \times_{L(Y)}L(X)} & {\tilde{S}} & Y & {L(X)} \\
        && S & {L(Y)}
        \arrow[from=2-4, to=3-4]
        \arrow[from=2-3, to=3-4]
        \arrow[from=1-3, to=2-3]
        \arrow[from=1-3, to=2-4]
        \arrow[from=3-3, to=3-4]
        \arrow[dashed, from=2-2, to=3-3]
        \arrow["\lrcorner"{anchor=center, pos=0.125}, draw=none, from=1-2, to=2-3]
        \arrow[from=1-2, to=2-2]
        \arrow[from=1-2, to=1-3]
        \arrow[dashed, from=2-2, to=2-3]
        \arrow["\simeq"{description}, draw=none, from=1-1, to=1-2]
        \arrow["\simeq"{description}, draw=none, from=1-1, to=2-1]
      \end{cd}
    \end{proof1}
  \end{lem}
  So it suffices to show $U \to \tilde{X}$ is an open embedding.
  Let $T \to \tilde{X}$ be a general point where $T$ is affine.
  By the way colimits of presheaves are computed,
  we can lift $T \to \tilde{X}$ to a map $T \to \cU$.
  We then have the cartesian squares : 
  \begin{cd}
    {T \times_{\tilde{X}} U} & {\mathcal{U} \times_{\tilde{X}} U} & U \\
    & {\cR} & {\mathcal{U}} \\
    T & {\mathcal{U}} & {\tilde{X}}
    \arrow[from=2-3, to=3-3]
    \arrow[from=3-2, to=3-3]
    \arrow[from=2-2, to=3-2]
    \arrow[from=2-2, to=2-3]
    \arrow["\lrcorner"{anchor=center, pos=0.125}, draw=none, from=2-2, to=3-3]
    \arrow[from=1-2, to=2-2]
    \arrow[from=1-3, to=2-3]
    \arrow[from=1-2, to=1-3]
    \arrow["\lrcorner"{anchor=center, pos=0.125}, draw=none, from=1-2, to=2-3]
    \arrow[from=3-1, to=3-2]
    \arrow[from=1-1, to=3-1]
    \arrow[from=1-1, to=1-2]
    \arrow["\lrcorner"{anchor=center, pos=0.125}, draw=none, from=1-1, to=3-3]
  \end{cd}
  For the left vertical morphism to be an open embedding,
  it suffices that the middle two vertical morphisms are open embeddings.
  By assumption, $\cR \to \cU$ is an open embedding.
  It remains to prove that the inclusion $U \to \cU$ is an open embedding.
  Using the lemma above again, it suffices to do so for
  the coproduct as presheaves, where the result is then clear.

\end{proof}

\begin{eg}
  Our example $\bA^2\setminus\set{0}$ which motivated
  Zariski sheaves is a scheme with
  atlas given by $D(x) + D(y)$.
\end{eg}

\begin{eg}

  Projective 1-space $\bP^1$ is defined by the equivalence relation
  \begin{cd}
    {\bA^1\setminus 0 } & {\bA^1 \coprod \bA^1} & {\bP^1}
    \arrow["t", shift left=2, from=1-1, to=1-2]
    \arrow["{1/t}"', shift right=2, from=1-1, to=1-2]
    \arrow[from=1-2, to=1-3]
  \end{cd}
\end{eg}
After we see how to interpret modules from algebra as vector bundles in geometry,
we will have another definition of $\bP^1$ and $\bP^n$ in general.

\end{document}