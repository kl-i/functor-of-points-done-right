\documentclass[./main.tex]{subfiles}
\begin{document}

In this section,
we prove the concrete description of pushforward
of quasi-coherent sheaves along a qcqs morphism $f : X \to Y$ between schemes.

Let us first consider the case when $Y = \SPEC A$ an affine.
\begin{prop}[Pushforward with affine target]
  
  The following three functors $\QCOH X \to \QCOH(\SPEC A)$ are isomorphic : 
  \begin{enumerate}
    \item (Abstract pushforward) the pushforward $f_*$
    \item (Global sections) For each $\cF \in \QCOH X$,
    take the $A$-module \[
      \cF(X) := (\QCOH X)(\cO_X , \cF)
    \]
    \item (Practical) For an affine Zariski cover $(U_i)_{i \in I}$ of $X$
    with $U_{ij} := U_i \cap U_j$ affine again,
    take the $A$-module \[
      \mathrm{Eq}\brkt{
        \prod_{i \in I} \cF(U_i) \rightrightarrows
        \prod_{i , j \in I} \cF(U_{i j})
      }  
    \]
  \end{enumerate}
\end{prop}
\begin{proof}
  (1 isomorphic 2)
  \[
    f_* \cF \simeq
    A\MOD(A , f_* \cF) \simeq 
    (\QCOH X)(f^* A , \cF) \simeq
    (\QCOH X)(\cO_X , \cF) =: \cF(X)
  \]

  (2 isomorphic 3)
  Let $\cU := \coprod_{i \in I} U_i$.
  Then we use \[
    \QCOH X \simeq \QCOH (X , \cU)  
  \]
  Under this equivalence,
  $\cO_X \in \QCOH X$ is sent to
  the descent data $(\cO(U_i))_{i \in I}$
  with the natural cocycle data.
  Then the $A$-module $(\QCOH X)(\cO_X , \cF)$
  is isomorphic to
  the $A$-module of morphisms of cocycle data \[
    (\cO(U_i))_{i \in I} \to (\cF(U_i))_{i \in I}  
  \]
  Such a map is equivalent to
  choosing a compatible system of sections of $\cF$
  on the opens $U_i$,
  i.e. an element of the desired equalizer.
\end{proof}
Most practical situations fall in the case of (3).
For theoretical purposes,
we can relax the condition of $U_i \cap U_i$ being affine 
by choosing an affine Zariski cover $(U_{i j}^k)_{k \in I_{i,j}}$ 
of $U_i \cap U_j$.
With that, one can define $f_*\cF$ by the equaliser diagram in
$\QCOH A$, 
\[
  f_* \cF \to \prod_{i \in I} \cF(U_i) \rightrightarrows
  \prod_{i , j \in I} \prod_{k \in I_{i, j}} \cF(U_{ij}^k)
\]
Let us call the data of $((U_i)_{i \in I} , (U_{ij}^k)_{k \in I_{ij}})$
a \emph{Zariski presentation} of $X$ because it ``presents''
$X$ as a coequalizer of affines \[
  \coprod_{ i ,j \in I} \coprod_{k \in I_{ij}} U_{ij}^k \rightrightarrows
  \coprod_{i \in I} U_i \to X
\]
Let us be more ambitious now and consider 
$f : X \to Y$ a morphism between schemes.
By the definition of $\QCOH Y$,
we must produce for each $y : \SPEC A \to Y$
a module $y^*f_* \cF \in \QCOH A$
equipped with quasi-coherent pullback when changing $y$.
Given $y : \SPEC A \to Y$, one can form the pullback square : 
\begin{cd}
  {X_y} & X \\
	{\mathrm{Spec}\,A} & Y
	\arrow["y"', from=2-1, to=2-2]
	\arrow["f", from=1-2, to=2-2]
	\arrow["{\tilde{y}}", from=1-1, to=1-2]
	\arrow["{\tilde{f}}"', from=1-1, to=2-1]
	\arrow["\lrcorner"{anchor=center, pos=0.125}, draw=none, from=1-1, to=2-2]
\end{cd}
which suggests a way of forming $y^* f_* \cF$ is
by using $\tilde{f}_* \tilde{y}^* \cF$ since 
we know how to deal with $\tilde{f}_*$
by the proposition we just proved.
The main issue is the \emph{quasi-coherence}.
\begin{enumerate}
  \item With the abstract definition of $\tilde{f}_*$ as the right adjoint
  of $\tilde{f}^*$ it is hard to say anything because
  when we have a change in source $\SPEC B \to \SPEC A$,
  we need to $\_ \otimes_A B$ which is a left adjoint.
  Categorically we cannot say much about how left and right adjoints
  interact.
  \item With the global sections definition,
  it is again not clear how this interacts with $\_ \otimes_A B$.
  \item The definition with \[
    \tilde{f}_*\tilde{y}^*\cF \to 
    \prod_{i \in I}(\tilde{y}^*\cF)(U_i) \rightrightarrows
    \prod_{i , j \in I} \prod_{k \in I_{ij}} (\tilde{y}^*\cF)(U_{ij}^k)
  \]
  where $((U_i)_{i \in I} , (U_{ij}^k)_{k \in I_{ij}})$ 
  is Zariski presentation of $X_y$
  has two orthogonal issues : 
  \begin{enumerate}
    \item Requires choosing a Zariski presentation of $X_y$
    and thus will not be functorial in $y$.
    \item $\_ \otimes_A B$ may not preserve equalisers.
    \item $\_ \otimes_A B$ may not commute with the
    arbitrary products defining the equaliser.
  \end{enumerate}
\end{enumerate}
(a) can be solved by taking filtered colimit over
all Zariski presentations ordered by refinement.
(b) can be solved by using the equivalence \[
  \QCOH Y \simeq \QCOH_\AFF Y  
\]
because for $\SPEC B \subs \SPEC A$ an inclusion of affine opens of $Y$
the algebra map $A \to B$ is flat.
(c) is a serious issue because
$\_ \otimes_A B$ \emph{never} commutes with taking infinite products.
For this to work,
we need $I$ to be finite and each $I_{ij}$ also to be finite
so that products are equivalent to direct sums.
This leads us to the notion of \emph{quasi-compact quasi-separatedness}.

\begin{dfn}
  
  Let $X$ be a scheme and $f : X \to Y$ be a morphism of schemes.
  Then $X$ is called a quasi-compact when
  it admits a \emph{finite} Zariski affine atlas.
  Relatively, $f$ is called quasi-compact when for all $y : \SPEC A \to Y$,
  $f^{-1}(y)$ is qc.

  $X$ is called quasi-separated when
  $\De : X \to X \times X$ is quasi-compact.
  Relatively, $f$ is called quasi-separated when $\De_f : X \to X \times_Y X$ is qc.

  One often asummes quasi-compact and quasi-separated together,
  which we will abbreviate to qcqs.
\end{dfn}

The rough analogy is 
\begin{cd}
  {M \,A\text{-module}} & {X \text{ scheme}} \\
	{M \text{ f.g.}} & {X \text{ q.c.}} \\
	{M \text{ coherent}} & {X \text{ q.c.q.s}}
\end{cd}

\begin{prop}
  
	Let $f : X \to Y$ be a morphism of schemes.
  Then $f$ is qcqs iff
  for all $y : S \to Y$ with $S$ affine,
  the scheme $X_y$ is qcqs.
\end{prop}
\begin{proof}
	(Forward) $T := X \times_Y S$ is qc by definition.
  To show qs, note that we have a commutative diagram
  \begin{cd}
    T & {T \times T} & {T \times T} \\
    X & {X \times_Y X} & {X \times X}
    \arrow[hook, from=2-2, to=2-3]
    \arrow[from=2-1, to=2-2]
    \arrow[hook, from=1-3, to=2-3]
    \arrow[from=1-1, to=2-1]
    \arrow[from=1-1, to=1-2]
    \arrow[from=1-2, to=1-3]
    \arrow[from=1-2, to=2-2]
    \arrow["\lrcorner"{anchor=center, pos=0.125}, draw=none, from=1-2, to=2-3]
    \arrow["\lrcorner"{anchor=center, pos=0.125}, draw=none, from=1-1, to=2-2]
  \end{cd}
  where \begin{enumerate}
    \item The right square is cartesian because
    $X \times_Y X \to X \times X$ is a monomorphism of presheaves
    and $T \times T \to X \times T$ factors through $X \times_Y X$.
    \item The outter square is cartesian because $T \to X$ is a monomorphism.
    \item Hence the left square is cartesian.
  \end{enumerate}
  Since $X \to X \times_Y X$ is qc,
  we have $T \to T \times T$ is qc as desired.

  (Reverse)
  Again, it is qs that is non-trivial.
  Goal : 
  for $S \to X \times_Y X$ with $S$ affine,
  show $U := X \times_{X \times_Y X} S$ is qc.
  Given $S \to X \times_Y X$,
  we have single map $S \to Y$.
  Let $T := X \times_S Y$.
  Then we have \begin{cd}
    X & T & U \\
    {X \times_Y X} & {T \times T} & S
    \arrow[from=1-1, to=2-1]
    \arrow[from=2-2, to=2-1]
    \arrow[from=2-3, to=2-2]
    \arrow[from=1-2, to=1-1]
    \arrow[from=1-2, to=2-2]
    \arrow["\lrcorner"{anchor=center, pos=0.125, rotate=-90}, draw=none, from=1-2, to=2-1]
    \arrow[from=1-3, to=1-2]
    \arrow[from=1-3, to=2-3]
    \arrow["\lrcorner"{anchor=center, pos=0.125, rotate=-90}, draw=none, from=1-3, to=2-2]
  \end{cd}
  where \begin{enumerate}
    \item the left square is cartesian as in the proof of the forward direction
    \item the bottom row comes from
    $S \to X \times_S X$ factoring through $T \times T$
    \item the outter square is cartesian by definition of $U$.
  \end{enumerate}
  It follows that the right square is cartesian.
  By assumption, $T \to T \times T$ is qc so
  $U \to S$ is qc and hence $U$ is qc.
\end{proof}

% \begin{dfn}
  
%   Let $\cF \in \QCOH X$ where $X$ is a scheme.
%   Then the global sections of $\cF$ is defined as
%   \[
%     \cF(X) := (\QCOH X)(\cO_X , \cF)
%   \]
%   This is a $\cO_X(X)$-module.

%   In general, for an open embedding $j : U \to X$,
%   define the sections of $\cF$ on $U$ to be
%   \[
%     \cF(U) := (j^{-1}\cF)(U)
%   \]
% \end{dfn}

% \begin{prop}
  
%   Let $f : X \to \SPEC A$ where $X$ is a scheme.
%   We also assume $f$ is qcqs, equivalently that $X$ is qcqs.
%   Then $f_* \_ \simeq \_ (X)$.
% \end{prop}
% \begin{proof}
%   \textbf{Not sure if need qcqs.}
%   \[
%     f_* \cF \simeq A\MOD(A , f_* \cF)
%     \simeq (\QCOH X)(f^* A , \cF)
%     \simeq (\QCOH X)(\cO_X , \cF)
%     = \cF(X)
%   \]
% \end{proof}

% \begin{prop}
  
%   Let $f : X \to Y$ be a qc morphism of schemes,
%   $\cF \in \QCOH X$.
% \end{prop}
% \begin{proof}
  
% \end{proof}
% Let's keep things simple by assuming more than quasi-separated-ness : 
% we assume $X$ has affine diagonal so that finite intersection of
% affine opens is still affine.
% At this point, we've effectively rediscovered 
% Cech cohomology and are trying to prove the independence of
% Cech cohomology w.r.t. an affine Zariski cover. 
% Great. I finally self-motivated the colimit definition of Cech cohomology.
% TLDR of Cech cohomology : 
% \begin{enumerate}
%   \item define $\check{C}^\bullet (\cU , \cF)$ for Zariski affine covers $\cU$.
%   \item show if $\cV$ refines $\cU$, then 
%   $\check{C}^\bullet (\cU , \cF) \to \check{C}^\bullet (\cV , \cF)$
%   is quasi-isomorphism
%   \item define $\check{H}^\bullet(X , \cF) := 
%   \COLIM_{\cU} \check{H}^\bullet (\cU , \cF)$
% \end{enumerate}

\begin{prop}[Base change for affine opens]
  
  Let $f : X \to Y$ be a qcqs morphism between schemes.
  Let $j : U \subs Y$ be an affine open and $\cF \in \QCOH X$.
  Then \[
    (f_* \cF)(U) \simeq \cF(f^{-1}(U))  
  \]
\end{prop}
\begin{proof}
  As we discussed.
  $f$ qcqs implies that
  for each $y : S \to Y$ with $S$ affine,
  the fiber $X_y$ admits a \emph{finite} Zariski presentation.
  Then the pushforward $X_y \to S$ can be computed
  by taking global sections w.r.t.
  any finite Zariski presentation of $X$.
  To make this functorial in $y$,
  we take the filtered colimit across all finite Zariski presentations
  ordered by refinement.
  Then the fact that the products in the equalizer
  defining global sections are finite
  and the fact that restricting along affine opens are flat
  implies that we have defined a quasi-coherent sheaf on $Y$.
  The fact that this is a right adjoint to $f^*$
  follows from our computation of the pushforward
  in the case of having an affine target.
\end{proof}
\end{document}