\documentclass[./main.tex]{subfiles}
\begin{document}

To motivate coverings,
let us first review what we know about surjections in sets!

\begin{prop}[Surjections in sets]
  
  Let $q : \cU \to X$ of sets.
  Then the following are equivalent : 
  \begin{enumerate}
    \item $q$ is surjective
    \item (Effective epimorphism) Consider $\cU \times_X \cU$ whose points are
    pairs of points $(a , b) \in \cU \times \cU$ such that
    $q a = q b$.
    \[
      \cU \times_X \cU \rightrightarrows \cU \to X
    \]
    is the coequalizer diagram.
    In concrete terms,
    this says ``a map out of $X$ is the same as
    a map out of $\cU$ which is constant along fibers of $\cU$''.
  \end{enumerate}
\end{prop}

We can play the same game in $\AFF$,
or dually in $\CRING$.
Specifically, given a commutative ring $A$
and a finite set of elements $f_1 , \dots , f_n \in A$,
we can form \[
  \cU := \coprod_{i = 1}^{n} D(f_i) 
  = \coprod_{i = 1}^{n} \SPEC A[1 / f_i]
  = \SPEC \prod_{i = 1}^n A[1 / f_i]
\]
Letting $X := \SPEC A$, one can show that 
a fiber product of $\cU$ with itself over $X$ is given by 
\[
  \cU \times_X \cU := \SPEC \prod_{i , j = 1}^n A[1 / f_i f_j]  
\]
It then makes sense to ask for when
the following is a coequalizer diagram in $\AFF$
\[
  \cU \times_X \cU \rightrightarrows \cU \to X  
\]
Or dually, we can ask when the following diagram in $\CRING$
is a equalizer
\[
  A \to \prod_{i = 1}^n A[1 / f_i] \rightrightarrows 
  \prod_{i , j = 1}^n A[1 / f_i f_j]  
\]
The idea is that if we view $\SPEC A$ as a physical system,
the basic opens $D(f_1) , \dots , D(f_n)$ cover $\SPEC A$
precisely when given any state $A \to K$ with measurements in a field $K$,
at least one of the measuring devices $f_i$ give a non-zero measurement. 
This leads to the following criterion.
\begin{prop}[Criterion for partition of unity using states]
  
  Let $f_1 , \dots , f_n \in A$ be elements of a ring.
  Then the following are equivalent : 
  \begin{enumerate}
    \item for all fields $K$ and maps of affine schemes $x : \SPEC K \to \SPEC A$,
    there exists some $f_i$ such that $f_i(x) \in K^\times$.
    \item the elements $f_1 , \dots, f_n$ gives a \emph{partition of unity},
    meaning $(f_1 , \dots , f_n) = (1)$.
  \end{enumerate}
\end{prop}
\begin{proof}
  (1 implies 2) This is the standard application of Zorn's lemma.
  Assuming for a contradiction that $(f_1 , \dots , f_n)$ does not 
  generate the unit ideal,
  one can use Zorn's lemma to give a maximal ideal $\f{m}$ of $A$
  containing $(f_1 , \dots , f_n)$.
  Using $K := A / \f{m}$ gives a contradiction.
  (2 implies 1) Suppose there exists a linear combination 
  $\sum_{i = 1}^n \la_i f_i = 1$ inside $A$.
  Let $K$ be a field and $x : \SPEC K \to \SPEC A$.
  Then we cannot have $f_i(x) = 0$ for all $i$ because
  that would imply $1 = 0$ inside $K$.
  Thus some $f_i(x)$ must be non-zero and hence a unit in $K$.
\end{proof}
\begin{dfn}[Basic Zariski covers]
  
  We call $\SPEC B \to \SPEC A$ a \emph{basic Zariski covering}
  when there exists $f_1 , \dots , f_n \in A$ such that
  \[
    B \simeq \prod_{i = 1}^n A[1 / f_i]  
  \]
  as $A$-algebras and any (and thus both)
  of the conclusions of the previous proposition holds.
\end{dfn}
We now show that basic Zariski coverings indeed give effective coequalizers
we discussed at the start.
\begin{prop}[Fundamental lemma of basic Zariski covers]
  
  Let $f_1 , \dots , f_n \in A$ and $B$ be the $A$-algebra
  $\prod_{i = 1}^n A[1/f_i]$.
  Then $(f_1 , \dots , f_n) = (1)$ iff
  for all $A$-modules $M$
  \[
  M \to M \otimes_A B \rightrightarrows 
  M \otimes_A B \otimes_A B
  \]
  is a equalizer diagram in $A\MOD$.
\end{prop}
\begin{proof}
  ($\limplies$) We want $A / (f_1 ,\dots , f_n) = 0$.
  Applying $M = A / (f_1 ,\dots , f_n)$ to the equalizer diagram,
  it suffices to show $(A / (f_1 ,\dots , f_n))[1 / f_i] = 0$ for each $i$.
  This follows since $0 = f_i$ would be a unit.

  ($\implies$) Note that we have isomorphisms of $B$ modules
  and $B \otimes_A B$ modules
  \[
    M \otimes_A B \simeq \prod_{i = 1}^n M[1 / f_i] \,\,\,\,\,\,\,\,\,\,\,
    M \otimes_A B \otimes_A B
    \simeq M \otimes_A \brkt{
      \prod_{i , j = 1}^n A [1 / f_i f_j]  
    }
    \simeq \prod_{i , j = 1}^n M[1 / f_i f_j]  
  \]
  so we equivalently asking for the following diagram to be an equalizer diagram :
  \[
    M \to \prod_{i = 1}^n M[1 / f_i] \rightrightarrows \prod_{i , j = 1}^n M[1 / f_i f_j]  
  \]
  One can prove this by direct computation. (Hint : see footnote.\footnote{
    Prove that $(f_1 , \dots , f_n) = (1)$
    actually implies $(f_1^N , \dots , f_n^N) = (1)$ 
    for any $N > 0$.
  })
  There is an alternative proof which generalises to \emph{fpqc coverings}.
  The idea is : 
  \begin{enumerate}
    \item $B$ is a \emph{faithfully flat} $A$-algebra.
    This means $\_ \otimes_A B$ is left exact and 
    for $M \in A\MOD$, $M \otimes_A B \simeq 0$ implies $M = 0$.
    \item $B$ faithfully flat over $A$ implies
    that to show left exactness sequences in $A\MOD$,
    it suffices to show left exactness after $\_ \otimes_A B$ in $B\MOD$.
    \item After $\_ \otimes_A B$ the equalizer fork
    becomes a split equalizer diagram
    and hence is left exact.
  \end{enumerate}
\end{proof}

In chapter 3 when we see modules over a ring $A$
as generalising vector bundles on $\SPEC A$,
we will be able to see the above proposition as saying
$f_1 , \dots , f_n$ gives a partition of unity
iff generalised vector bundles on $\SPEC A$
are determined by their restrictions to
$\cU := \coprod_{i = 1}^n D(f_i)$.

\end{document}