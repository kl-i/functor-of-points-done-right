\documentclass[./main.tex]{subfiles}
\begin{document}

Before we discuss an example of ``non-affine phenomenon'',
which will lead us to the definition of schemes,
let us give the promised discussion on
\emph{uniqueness} of objects equipped with universal properties.
This will nicely lead us to consider
an enlargement of $\AFF$ in which schemes will eventually reside.
The reader familiar with category theory can skip this section.

To recap, we have seen the following examples of
objects with universal properties :
\begin{enumerate}
  \item The affine lines $\bA^1$ has the universal property that
  morphisms into it correspond to elements of coordinate rings.
  In other words,
  we have an isomorphism \[
    \AFF(S , \bA^1)  \simeq \cO(S)
  \]
  functorial in $S \in \AFF$.
  \item Given a closed embedding $i : Z \to X$ in $\AFF$
  with ideal $I_Z$,
  a morphism $S \to X$ factors through $i$ iff
  $I_Z$ pullsback to zero on $S$ and the
  factoring is unique when it exists.
  This can be phrased as saying
  composition with $i$ gives a bijection
  \[
    \AFF (S , Z) \simeq \set{
      x \in \AFF(S , X) \st \forall\,f \in I_Z , f(x) = 0
    } 
  \]
  functorial in $S \in \AFF_k$.
  This is a nice rephrasing because the latter set
  is a subset of $\AFF(S , X)$
  which aligns with the idea that $i$ realises
  $Z$ as a ``subspace'' of $X$.
  \item Let $X \in \AFF$ and $f \in \cO(X)$.
  Then the universal property of $j : D(f) \to X$ can 
  phrased as saying the composition with $j$ gives a bijection
  \[
    \AFF(S , D(f)) \simeq \set{
      x \in \AFF(S , X) \st f(x) \in \cO(S)^\times
    }
  \]
  functorial in $S \in \AFF$.
  Again, the latter is a subset of $\AFF(S , X)$ which is nice
  because one thinks of $j$ as realising $D(f)$ as a ``subspace'' of $X$.
  \item Let $X , Y$ be affine schemes over an affine scheme $S$.
  Given a fiber product $W$ of $X , Y$ over $S$
  with projections $a : W \to X , b : W \to Y$,
  note that composition by $a$ and $b$ gives 
  \[
    \AFF(T , W) \to \AFF(T , X)  
  \]
  \[
    \AFF(T , W) \to \AFF(T , Y)  
  \]
  functorial in $T \in \AFF$.
  Furthermore, the assumption that $W \to X \to S$ equals
  $W \to Y \to S$ implies that
  we have a map of sets \[
    \AFF(T , W) \to \AFF(T , X) \times_{\AFF(T , S)} \AFF(T , Y)
  \]
  functorial in $T \in \AFF$.
  The universal property of $W$ as a fiber product
  then precisely says that the above map
  functorial in $T$ is a bijection.
\end{enumerate}

The finite disjoint union and basic Zariski coverings
are of a dual nature since their universal property is
not about mapping \emph{into} them but rather
mapping \emph{out}.
We will discuss these in the next section.

Back to the four examples above,
the common pattern is that
they are all of the form of \[
  \AFF(\_ , \cX) \simeq F(\_)  
\]
where \begin{itemize}
  \item $F$ is a functor $\AFF^\OP \to \SET$
  describing the desired property
  \item $\cX$ is an affine scheme and $\AFF(\_ , \cX)$
  is the functor $\AFF^\OP \to \SET, S \mapsto \AFF(S , \cX)$.
  \item the above isomorphism exhibits $\cX$
  as being the \emph{universal} affine scheme
  with the desired property specificed by $F$.
\end{itemize}
Our goal is to show that in the above situation,
the affine scheme $\cX$ is unique is some appropriate sense.
Now that we have enough examples,
we make the formal development.

\begin{dfn}
  
  The category of \emph{presheaves (on $\AFF$)} is
  defined as the category of functors $\AFF^\OP \to \SET$.

  For each $S \in \AFF$,
  we have the presheaf \[
    \underline{S} : \AFF^\OP \to \SET ,
    T \mapsto S(T)
  \]
  This defines a functor called the \emph{Yoneda embedding}
  \[
    \AFF \to \PSH \AFF , S \mapsto \underline{S}    
  \]
\end{dfn}
Hopefully through the four examples we have seen now,
one can see that intuitively a presheaf $X$ is supposed
to be ``something affine schemes map into''.
In more details,
\begin{itemize}
  \item For each $S \in \AFF$, one should think of
  the set $X(S)$ as ``the set of morphisms $S \to X$''.
  \item Given $t : T \to S$ in $\AFF$,
  we should be able to ``restrict morphisms $S \to X$
  to get morphisms $T \to X$''.
  In other words, we should be given a map of sets 
  \[
    X(S) \to X(T) 
  \]
  which we will denote with $x \mapsto x t$.
  \item If $t = \id_S : S \to S$,
  then we should have \[
    X(S) \to X(S) , x \mapsto x \id_S = x
  \]
  \item Functoriality of $X$ expresses
  the idea that iterated restriction
  should be equal to restricting along the composition.
  \item A morphism of presheaves $\varphi : X \to Y$ should be
  the data of turning maps $S \to X$ from $S \in \AFF$
  into $S \to Y$.
  In other words, for every $S \in \AFF$ a map \[
    \varphi : X(S) \to Y(S)  
  \]
  Naturality expresses the idea that
  given $t : T \to S$ in $\AFF$,
  composing $x : S \to X$ by $\varphi$ then restricting along $t$
  should be equal to first restricting along $T$ and then
  composing with $\varphi$.
  \[
    (\varphi x) t  = \varphi (x t) 
  \]
  \item Each affine $S$ is clearly something
  that affines can map to.
  This is what $\underline{S}$ records.
\end{itemize}
With the above intuition,
one obvious question to ask is whether we have an isomorphism
\[
  (\PSH \AFF)(\underline{S} , X) \simeq X(S)
\]
for a presheaf $X$ and affine $S$.
Given a morphism $\varphi : \underline{S} \to X$
which we see as a map from $S \to X$ as presheaves,
we should be able to get a map $S \to X$ as a point in $X(S)$.
The obvious thing to do is to ``compose $\varphi$ with the identity of $S$''.
This is precisely \emph{Yoneda's lemma}.
\begin{prop}[Yoneda's lemma]
  
  We have a bijection \[
    (\PSH \AFF)(\underline{S} , X) \simeq X(S) ,
    \varphi \mapsto \varphi \id_S
  \]
  functorial in $X \in \PSH \AFF$ and $S \in \AFF$.
  We will henceforth confuse the two sets.
\end{prop}
\begin{proof}
  (Functoriality in $S$)
  Given $t : T \to S$,
  we have \begin{cd}
    {\varphi \underline{t}} &&& {(\varphi \, \underline{t})(\id_T) = \varphi (t)
    = (\varphi \id_S) t} \\
    & {\mathrm{PSh}(\underline{T} , X)} & {X(T)} \\
    & {\mathrm{PSh}(\underline{S} , X)} & {X(S)} \\
    \varphi &&& {\varphi \id_S}
    \arrow[from=3-2, to=3-3]
    \arrow[from=3-2, to=2-2]
    \arrow[from=3-3, to=2-3]
    \arrow[from=2-2, to=2-3]
    \arrow[from=4-1, to=1-1]
    \arrow[from=4-1, to=4-4]
    \arrow[from=4-4, to=1-4]
    \arrow[from=1-1, to=1-4]
  \end{cd}

  (Functoriality in $X$)
  Given $\varphi : X \to Y$, we have 
  \begin{cd}
    x &&& {x \id_S} \\
    & {\mathrm{PSh}(\underline{S} , X)} & {X(S)} \\
    & {\mathrm{PSh}(\underline{S} , Y)} & {Y(S)} \\
    {\varphi x} &&& {(\varphi x) \id_S = \varphi (x \id_S)}
    \arrow[from=3-2, to=3-3]
    \arrow[from=2-2, to=2-3]
    \arrow[from=2-2, to=3-2]
    \arrow[from=2-3, to=3-3]
    \arrow[from=1-1, to=4-1]
    \arrow[from=4-1, to=4-4]
    \arrow[from=1-1, to=1-4]
    \arrow[from=1-4, to=4-4]
  \end{cd}

  (Inverse) Given $x \in X(S)$, we construct 
  $\underline{x} : \underline{S} \to X$ as follows : 
  for each $t \in \underline{S}(T) = S(T)$ define
  $\underline{x} t := x t$ by functoriality of $X$.
  Naturality of $\underline{x}$ : 
  For $r : U \to T$ in $\AFF$, we have for every $t \in \underline{S}(T) = S(T)$
  \[
    (\underline{x} t) r = (x t) r = x (t r) = \underline{x}(t r)
  \]
  where the middle equality is functoriality of $X$.
  Clearly, $\underline{x} \id_S = x$.
  And conversely given $\varphi : \underline{S} \to X$,
  we have that for all points $t \in \underline{S}(T) = S(T)$,
  \[
    \varphi t = \varphi (\id_S t) = (\varphi \id_S) t 
    = \underline{\varphi \id_S} t
  \]
\end{proof}

Yoneda's lemma implies that : 
\begin{prop}
  
  The Yoneda embedding $\AFF \to \PSH \AFF$
  is fully faithful.
  We henceforth confuse $S$ with $\underline{S}$ for $S \in \AFF$
  and view $\AFF$ as a full subcategory of $\PSH \AFF$.
\end{prop}

Yoneda's lemma also justifies the idea that
points of a presheaf $X$ are precisely maps from affines in $X$.
A presheaf should be determined by its points
and thus the following theorem surfaces.

\begin{prop}[Density theorem A.K.A. function extensionality for presheaves]
  
  Let $X , Y \in \PSH \AFF$
  Then  \begin{align*}
    (\PSH \AFF)(X , Y) 
    &= \set{ (\varphi x) \prod_{S \in \AFF} \prod_{x \in X(S)} Y(S)
    \st \text{ for all $t : T \to S$ in $\AFF$ we have }
    (\varphi x) t = \varphi (x t)
    } \\
    &\simeq \set{ (\varphi x) \prod_{x : S \to X} \PSH(S , Y)
    \st \text{ for all $t : T \to S$ in $\AFF$ we have }
    (\varphi x) t = \varphi (x t)}
  \end{align*}
  where $S \to X$ ranges over all morphisms from all affines into $X$.
\end{prop}
This expresses $X$ as the colimit of
the diagram $\AFF / X$ in $\PSH \AFF$.
\begin{proof}
  The isomorphism is given by Yoneda's lemma.
\end{proof}
To talk about uniqueness of objects equipped with universal properties,
we need the notion of a final object in a category.
% It is clear that in our discussion of presheaves so far
% we have used nothing about the category $\AFF$.
% It can really be replaced by any category.
\begin{prop}[Final objects and their uniqueness]
  
  Let $\cC$ be a category and $A$ an object of $\cC$.
  $A$ is called a final object when 
  every object in $\cC$ has a \emph{unique} morphism to $A$.
  Suppose $A, B$ are both final objects over $\cC$.
  Then there exists a unique isomorphism $A \simeq B$.
  
\end{prop}
\begin{proof}
  By finality of $A$ there exists a unqiue map $a : B \to A$.
  Similarly there exists a unique map $b : A \to B$.
  The composition $a b : A \to A$ and the identity of $A$
  both give maps from $A$ to itself so we must have
  $a b = \id_A$.
  Similarly $b a = \id_B$.
\end{proof}
We are ready.
\begin{prop}[Uniquness of objects defined by universal properties]
  
  Let $X \in \PSH$, $S \in \AFF$ and $x : S \to X$.
  The following are equivalent : 
  \begin{enumerate}
    \item $x$ is an isomorphism of presheaves
    \item $x : S \to X$ is final in the category $\AFF / X$
  \end{enumerate}
  When the above is true we say that
  $x$ exhibits $S$ as a representative of $X$
  It follows that representatives $(S,x)$ are unique up to unique isomorphism
  in the sense that given another representative $(\tilde{S} , \tilde{x})$
  there exists a unique isomorphism $S \simeq \tilde{S}$ such that
  we have a commuting triangle \begin{cd}
    & S \\
    {\tilde{S}} & X
    \arrow["{\tilde{x}}"', from=2-1, to=2-2]
    \arrow["x", from=1-2, to=2-2]
    \arrow["\sim", from=2-1, to=1-2]
  \end{cd}
\end{prop}
\begin{proof}
  (1 to 2) Clear.
  (2 to 1) We need to construct an inverse $s : X \to S$ to $x$.
  We do it by the density theorem of presheaves.
  For a point $t : T \to X$ where $T \in \AFF$,
  by finality of $(S , x)$
  there exists a unique map $s t : T \to S$ such that $x (s t) = t$.
  We need to show that given 
  $u : U \to T$ in $\AFF$ and $t : T \to X$ we have \[
    s (t u) = (s t) u
  \]
  By uniqueness of the finality of $(S , x)$,
  it suffices to show $x (s (t u)) = x((s t) u)$.
  We have \begin{align*}
    x(s (t u)) 
    &= t u &\text{    by def of $s (t u)$} \\
    &= (x (s t)) u &\text{    by def of $s t$} \\
    &= x((s t) u) &\text{    by naturality of $x$}
  \end{align*}
  This defines a morphism $s : X \to S$
  which by definition satisfies $x s = \id_X$.
  It remains to check $s x = \id_S$.
  By uniqueness of finality of $(S,x)$ it suffices to check that
  $x (s x) = x$.
  This is true by definition of $s x$.

\end{proof}
That concludes the discussion on uniqueness of objects defined by
universal properties.
For the rest of this section,
we give some features of the presheaf category $\PSH \AFF$.
First, it interacts well with relative point of view.

\begin{prop}
  For $S \in \PSH \AFF$,
  the Yoneda embedding $\AFF \to \PSH \AFF$
  gives a fully faithful functor
  \[
    \AFF / S \to (\PSH \AFF) / S 
  \]
  This extends to an equivalence \[
    \PSH(\AFF / S) \simeq (\PSH \AFF) / S
  \]
  In particular,
  for any algebra $k$,
  $(\PSH \AFF) / \SPEC k \simeq \PSH(\AFF_k)$.
\end{prop}
\begin{proof}
  For $X \in \PSH(\AFF / S)$,
  use the density theorem to write it as the colimit
  of affines over $S$.
  Taking this diagram and take the colimit in $\PSH \AFF$.
  This defines a presheaf equipped with a morphism to $S$.
  This procedure is functorial in $X$.
  For a quasi-inverse functor,
  simply do the reverse.
  Take a presheaf $X$ with a morphism to $S$,
  write $X$ as the colimit of affines using the density theorem.
  These affines map to $S$ via $X \to S$
  and so define objects in $\AFF / S$.
  Finally take the colimit in $\PSH(\AFF / S)$.
\end{proof}

As another show case of the power of the Yoneda embedding.
Let us prove the following fact which we will need next section.
The yoga is that once you know the proof in the category of sets,
you know the proof in any presheaf category and hence
any category.
\begin{prop}[Criterion for factoring through monomorphisms]
  The following are true : 
  \begin{enumerate}
    \item Let $f : X \to Y$ be a map of sets and
    $V \subs Y$ a subset of $Y$.
    Then $f$ factors through $V$ iff there exists
    a map $X \to V$ which exhibits $X \simeq X \times_Y V$.
    \item The same in $\AFF$,
    replacing subset with monomorphism.
  \end{enumerate}
\end{prop}
\begin{proof}
  (1) is an easy exercise.
  (2) We have already seen that
  fiber product of presheaves is computed
  ``pointwise'' : 
  for presheaves $X , Y$ over a presheaf $S$
  we have isomorphism functorial in affine $T$
  \[
    (X \times_S Y)(T) \simeq X(T) \times_{S(T)} Y(T)
  \]
  Similarly, a morphism of presheaves
  is a monomorphism iff it is pointwise a monomorphism.
  It follows that (1) implies (2).
\end{proof}

\end{document}