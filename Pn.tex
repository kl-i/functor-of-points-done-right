\documentclass[./main.tex]{subfiles}
\begin{document}

We are going to give the definition of
$\bP^n$ as the ``quotient of $\bA^{n+1}\setminus\set{0}$
by the scaling action of $\bG_m$''.
We begin with the intuition of quotients of group actions
in sets.

\begin{prop}
  
  Let $G$ be a group acting on a set $X$.
  Let $S$ be a another set.
  There is a bijection between \begin{enumerate}
    \item the set of maps $S \to X / G$
    \item the set of $G$-equivariant maps $S \times G \to X$
    up to the equivalence relation of \begin{cd}
      {S \times G} & X \\
      S & {S \times G}
      \arrow[from=1-1, to=2-1]
      \arrow[from=2-1, to=2-2]
      \arrow["x", from=1-1, to=1-2]
      \arrow["y"', from=2-2, to=1-2]
      \arrow["{\exists \, g}"', from=1-1, to=2-2]
    \end{cd}
  \end{enumerate}
\end{prop}
\begin{proof}
  Exercise.
\end{proof}

The above indicates how to define the functor $\bP^n$.
The issue is that for an affine $S$,
maps $S \times G \to X$ do not glue Zariski locally on $S$.

\end{document}