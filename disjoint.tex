\documentclass[./main.tex]{subfiles}
\begin{document}

In the previous section,
we saw that for $k$ algebraically closed characteristic not two,
we have \[
  k[t] / (t^2 - \la) \simeq k[t] / (t - a) \times k[t] / (t + a)  
\]
for $\la \in k^\times$ and some $a \in k$ with $a^2 = \la$.
The $k$-algebras $k[t] / (t \pm a)$ individually are isomorphic to $k$
and we saw that the $k$-points of the affine scheme associated
to the above ring had exactly two elements.
This is an example of \emph{finite disjoint unions of affine schemes}.
Once again, the notion of finite disjoint union in $\AFF$ 
can be understood by 
\begin{enumerate}
  \item considering finite disjoint union in 
  the category of sets
  \item extracting the universal property in the example of sets
  \item using the universal property to define finite disjoint
  union of affine schemes.
\end{enumerate}
Since we have done this a few times now,
we skip straight to the answer.

\begin{dfn}
  
  Let $S_1 , \dots , S_n$ be a finite collection of affine schemes.
  Then a \emph{disjoint union of $S_1 , \dots , S_n$} is defined as
  an affine scheme $S$ together with morphisms
  $\io_k : S_k \to S$ such that
  for all affines schemes $T$ together with
  morphism $\tau_k : S_k \to T$ there
  exists a unique morphism $S \to T$ making the following
  diagram commute : 
  \begin{cd}
    {S_1} \\
    \vdots & S & T \\
    {S_n}
    \arrow["{\iota_1}"', from=1-1, to=2-2]
    \arrow["{\iota_n}", from=3-1, to=2-2]
    \arrow["{\tau_1}", from=1-1, to=2-3, bend left = 20]
    \arrow["{\tau_n}"', from=3-1, to=2-3, bend right = 20]
    \arrow["{\exists !}"{description}, dashed, from=2-2, to=2-3]
  \end{cd}

\end{dfn}

\begin{prop}[Existence of finite disjoint union]
  
  Let $A_1 , \dots , A_n$ be a finite collection of rings.
  Let $A := \prod_{l = 1}^n A_l$.
  Then a disjoint union of $\SPEC A_1 , \dots , \SPEC A_n$ is
  given by $\SPEC A$
  where the maps $\SPEC A_i \to \SPEC A$
  come from the projections $A \to A_k$.
  Furthermore, the morphisms $\SPEC A_k \to \SPEC A$
  are both closed embeddings and basic open embeddings.

\end{prop}
Note that this makes sense intuitively : 
a function on $\SPEC A \sqcup \SPEC B$ should be
a pair of functions, one on $\SPEC A$ and one on $\SPEC B$,
i.e. an element of $A \times B$.
\begin{proof}
  For the existence, down to the universal property of the product of rings.

  The fact that $\SPEC A_k \to \SPEC A$
  is a closed embedding comes down to
  the fact that the projection $A \to A_k$ is surjective.

  Let $e_k \in A$ denote the
  element that is $1$ in the $k$-th component and zero everywhere else.
  Then the kernel of the projection to $A_k$ is precisely
  the elements $f$ such that $f e_k = 0$.
  It follows that for any ring map $A \to B$ which inverts $e_k$,
  the kernel of the projection $A \to A_k$ is annihilated
  so $A \to B$ factors through $A \to A_k$
  which proves that $\SPEC A_k \to \SPEC A$ is a basic open 
  of $\SPEC A$ for $e_k$.
\end{proof}

We also have a criterion.

\begin{prop}
  
  Let $S \in \AFF$.
  Then $S$ is a disjoint union of two affine schemes
  iff there exists an idempotent $e \in \cO(S)$.
\end{prop}
Intuitively, a idempotent $e$ is
the constant function with value 1 on a closed and open subspace
and $1 - e$ is the corresponding constant function with value 1
on the complement.
\begin{proof}
  Exercise. Hint : See footnote.\footnote{
    Use the chinese remainder theorem.
  }
\end{proof}

One may be wondering why we are sticking to finite disjoint unions
when the proof of the existence of disjoint unions
does not seem to use finiteness.
The following exercise demonstrates infinite product of rings
``computes the wrong thing''.

\begin{ex}
  Define the affine scheme $S := \SPEC \bF_2^\bN$ over $\bF_2$.
  Show that $S$ has a $\bF_2$-point which does not
  come from any of the projections $\bF_2^\bN \to \bF_2$.
\end{ex}

\end{document}