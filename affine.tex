\documentclass[./main.tex]{subfiles}
\begin{document}

Slogan : 
\emph{
  Commutative algebras as mathematical formalisation of
physical systems
}
The above idea is from \cite{Nes21}.
  
The idea of scientific empiricisms is that 
a \emph{physical system} nothing more than what we
can \emph{observe}. Observations are made with \emph{measuring devices}.
Measuring devices give for each \emph{state} a number.
Numbers are elements of a \emph{field}.
Given that numbers are elements of a field,
one sees that measuring devices can be added, 
subtracted, scaled, multiplied, but \textbf{not} divided
because they may give zero for certain states.
We thus arrive at the concept of a commutative ring with unity.
All rings hencforth are assumed to be cummutative with unity from here on.
  
We think of a ring $A$ as defining
a physical system by the measuring devices we can use on it.
When we want to think to $A$ as a physical system,
we refer to it as $\SPEC A$.
% This is nice because this is one idea away from schemes:
% schemes add the idea of allowing measuring devices
% which can only give values on certain states,
% i.e. \emph{local} measuring devices.
% One also adds the idea that local measuring devices
% can be \emph{patched together} along
% states which they share equal values on.

A single state in $\SPEC A$ is defined as
  a morphism of $k$-algebras $A \to k$.
  Intuitively, the role of a single state is give
  for each measuring device a number.

Examples of algebras and their physical interpretation : 
  $k[x_1 , \dots , x_n]$ represents affine $n$-space,
  $k$ represents the system where there is only one state.
  If one thinks of all states as positions,
  then it makes sense to confuse the word ``physical system''
  with the word ``space''.
  This is intuitively why one might call $\SPEC A$ a space.
  
A \emph{map of spaces} $\SPEC B \to \SPEC A$ 
  should be determined by how measuring devices on $\SPEC A$
  give rise to measuring devices on $\SPEC B$,
  i.e. a morphism of $k$-algebras $A \to B$.
  We take this as the definition.

One can also think of a map $\SPEC B \to \SPEC A$
as a family of states in $\SPEC A$ parameterised by $\SPEC B$.
E.g. $\bA^1 \to \SPEC A$ is a line of states in $\SPEC A$.

Let us make formal definitions.

\begin{dfn}
  
  Throughout, we fix a field $k$.
  Define the \emph{category $\AFF_k$ of affine schemes over $k$}
  as the opposite category of the category of
  commutative algebras over $k$.

  For a $k$-algebra $A$,
  we use $\SPEC A$ to denote the correponding object in $\AFF_k$.
  We refer to $\SPEC A$ as the \emph{affine scheme associated to $A$}.
  Conversely, given $S \in \AFF_k$,
  we write $\cO(S)$ for the corresponding $k$-algebra.
  We refer to $\cO(S)$ as the \emph{coordinate ring of $S$}.
  We refer to elements $f \in \cO(S)$ as
  \emph{functions on $S$}.

  For a morphism $\varphi : \SPEC B \to \SPEC A$ in $\AFF_k$,
  the corresponding map $A \to B$ will be denote \[
    \varphi^* : A \to B , f \mapsto f \varphi  
  \]
  We refer to $\varphi^*$ as 
  \emph{pulling back functions on $\SPEC A$ along $\SPEC B$}.
\end{dfn}

Remark for readers concerned about having a base field $k$ : 
the theory of schemes importantly does \emph{not}
need a base field $k$.
Historically, one of the reasons schemes
came about is because of the desire to apply
geometric arguments to problems in number theory.
(See Weil conjectures.)
In such applications, it is important to work with
\emph{all} commutative rings,
or in other words $\bZ$-algebras.
In the ``physics'' motivation we gave above,
we fixed a base field $k$ only for narrative.

\begin{dfn}
  
  Let $X$ be an affine scheme over $k$.
  A \emph{$k$-point of $X$} is defined as
  a morphism of affine schemes $\SPEC k \to X$.
  We write $X(k)$ for the set of $k$-points of $X$.

  More generally, for affine scheme $S$
  an \emph{$S$-point of $X$} is defined as
  a morphism of affine schemes $S \to X$.
  We write $X(S)$ for the set of $S$-points of $X$.
\end{dfn}

\begin{eg}[$k$-points do not determine affine schemes]
  
  Consider the morphism of affine schemes over $k$
  \[
    0 : \SPEC k \to \SPEC k[t] / (t^2)
  \] corresponding to
  the algebra morphism $k[t] / (t^2) \to k$ given by
  setting $t = 0$.
  Note that tautologically, $0$ is a $k$-point of $\SPEC k[t]/(t^2)$.
  This is in fact the unique $k$-point because
  $t^2 = 0$ in $k[t]/(t^2)$ so
  any $k$-algebra map $k[t]/(t^2) \to k$ must send $t \mapsto 0$.

  There is only one $k$-point of $\SPEC k$
  because the only $k$-algebra map $k \to k$ is the identity.
  Composing with $0$ gives the $k$-point $0$ of $\SPEC k[t] / (t^2)$.
  So the morphism $0$ induces a bijection on $k$-points : 
  \[
    \bullet = (\SPEC k)(k) \map{\sim}{} (\SPEC k[t]/(t^2))(k) = \set{0}
  \]
  However $0$ is not an isomorphism of affine schemes over $k$.
  Indeed, $k[t]/(t^2) \to k$ is not an isomorphism of $k$-algebras.

\end{eg}

\begin{eg}[$k$-points can be empty]
  
  Let $k = \bQ$.
  Consider $\nothing = \SPEC 0$ where $0$ is the zero ring.
  Then $\nothing$ has no $k$-points because
  $1 \neq 0$ in any field.

  Consider $S = \SPEC \bQ[t] / (t^2 + 1)$.
  Then $S$ has no $k$-points because $\bQ$ does not contain
  an element $i$ such that $i^2 + 1 = 0$.

  So $T, S$ both have empty sets of $k$-points!
  Of course, the two are not isomorphic as affine schemes over $k$
  because $\bQ[t] / (t^2 + 1) \simeq \bQ(i)$ as $\bQ$-algebras
  and in particular not the zero ring.
\end{eg}

\begin{dfn}
  
  Affine $n$-space over $k$ is defined as 
  the affine scheme $\bA^n_k$ corresponding to
  the $k$-algebra $k[x_1 , \dots , x_n]$.
\end{dfn}

In the theory of smooth manifolds,
we have a bijection between smooth functions and
manifold morphisms to $\bR$ \[
  C^\infty(M) \simeq \mathrm{Mfd}(M , \bR)
\]
which is functorial in manifolds $M$.
The terminology is that $\bR$ \emph{classifies}
smooth functions or
that $\bR$ is a \emph{moduli space} for smooth functions.
We have an analogous statement in algebraic geometry.

\begin{prop}[Universal property of the affine line]
  
  There exists a bijection
  \[
    \cO(S) \simeq \AFF_k(S , \bA^1_k)
  \]
  functorial in $S \in \AFF_k$.
  Given $f \in \cO(S)$,
  we often say that \emph{by the universal property of $\bA^1_k$, 
  $f$ gives a map $S \to \bA^1_k$}.
  We will usually use $f$ again to denote the morphism $S \to \bA^1_k$.
\end{prop}
In our setup of algebraic geometry,
the ring of functions on a space
comes first before the space
so this statement will be more or less a tautology.
\begin{proof}
  We describe the two directions : 
\begin{enumerate}
  \item Given $f \in \cO(S)$,
  define $k[t] \to \cO(S)$ by $t \mapsto f$.
  This defines a morphism
  $S \to \bA^1_k$ in $\AFF_k$.
  \item Given $\varphi : S \to \bA^1_k$ in $\AFF_k$,
  we obtain an element $\varphi^*(t) = t \varphi \in \cO(S)$.
\end{enumerate}
One can show the above are inverses and
functorial in $S$.
\end{proof}
Under this identitification,
for $\varphi : T \to S$ in $\AFF_k$,
the $k$-algebra map $\varphi^*$ really does
correspond to precomposition of morphisms $S \to \bA^1_k$
with $\varphi$.

\begin{ex}[$\SPEC k$ is the final object of $\AFF_k$]
  
  Show that every affine scheme over $k$
  has a unique map to $\SPEC k$.
  In the language of category theory,
  we say that $\SPEC k$ is a \emph{final object} of $\AFF_k$.
  This is analogous to how the singleton set
  is the final object in the category of sets.
  In other words,
  one should think of $\SPEC k$ as
  ``the singleton space'' in the category $\AFF_k$.
\end{ex}

\end{document}