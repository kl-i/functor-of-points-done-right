\documentclass[./main.tex]{subfiles}
\begin{document}

\begin{dfn}[Quasi-coherent sheaves on a functor]
  
  Let $X \in \PSH \AFF$.
  Then a quasi-coherent sheaf on $X$ 
  consists of the following data : 
  \begin{itemize}
    \item for each $x : \SPEC A \to X$, 
    an $A$-module $\cF_x$.
    \item (transition map) 
    for each $f : \SPEC B \to \SPEC A$ and $x : \SPEC A \to X$,
    a morphism of $A$-modules 
    \[
      \cF_x \to \cF_{x f}
    \]
    that is identity when $f$ is.
    We call the map $\cF_x \to \cF_{xf}$ 
    the \emph{transition map associated to $f$}.
    \item (transitivity) We require the maps given in the previous point 
    to satisfy that for any commuting triangle on the left,
    \begin{cd}
      {\SPEC C} && \rightsquigarrow & {\cF_{xfg}} \\
      {\SPEC B} & {\SPEC A} && {\cF_{xf}} & {\cF_x}
      \arrow["g"', from=1-1, to=2-1]
      \arrow["f"', from=2-1, to=2-2]
      \arrow[from=1-1, to=2-2]
      \arrow[from=2-5, to=2-4]
      \arrow[from=2-4, to=1-4]
      \arrow[from=2-5, to=1-4]
    \end{cd}
    we get a commuting triangle on the right of $A$-modules.
    \item (quasi-coherence)
    The transition map associated to $f : \SPEC B \to \SPEC A$
    for any point $x : \SPEC A \to X$
    induces an isomorphism $\cF_x \otimes_A B \simeq \cF_{x f}$.
  \end{itemize}
  Let $\cF , \cG$ be two quasi-coherent sheaves on $X$.
  Then a morphism $\varphi : \cF \to \cG$ consists of the data : 
  \begin{itemize}
    \item for each $x : \SPEC A \to X$,
    a morphism of $A$-modules 
    \[
      \varphi_x : \cF_x \to \cG_x  
    \]
    \item for each $f : \SPEC B \to \SPEC A$ and $x : \SPEC A \to X$,
    we have the commutative diagram in $A$-modules 
    \begin{cd}
      {\mathcal{F}_{xf}} & {\mathcal{G}_{xf}} \\
      {\mathcal{F}_x} & {\mathcal{G}_x}
      \arrow[from=2-1, to=1-1]
      \arrow[from=2-2, to=1-2]
      \arrow["{\varphi_x}"', from=2-1, to=2-2]
      \arrow["{\varphi_{xf}}", from=1-1, to=1-2]
    \end{cd}
    where the vertical morphisms are
    the transition maps of $\cF$ and $\cG$
    associated to $f$.
  \end{itemize}
  We write $\QCOH X$ for the category of quasi-coherent sheaves on $X$.
\end{dfn}

A priori, a quasi-coherent sheaf consists of
infinite amount of data.
We nonetheless have the following for affine schemes.

\begin{prop}
  
  Let $X = \SPEC R$.
  We define a functor $R\MOD \to \QCOH X$.
  \begin{enumerate}
    \item (Objects) Let $M \in R\MOD$.
    \begin{itemize}
      \item (fibers) 
      For every algebra map $R \to A$,
      we have an $A$-module $A \otimes_R M$.
      \item (transition maps) For $R \to A \to B$,
      the adjunction $B \otimes_A \_ \dashv \text{ forget } : 
      A\MOD \leftrightarrow B\MOD$ gives the morphism
      \[
        A \otimes_R M \to B\otimes_R M ,
        a \otimes x \mapsto \tilde{a} \otimes x
      \]
      where $\tilde{a}$ is the image of $a$ in $B$.
      If $A \to B$ is the identity of $A$
      then the above map is indeed the identity.
      \item (transitivity) This is clear.
      \item (quasi-coherence)
      The map $A \otimes_R M \to B\otimes_R M$
      indeed induces
      \[
        B \otimes_A (A \otimes_R M) \map{\sim}{} B\otimes_R M
      \]
    \end{itemize}
    \item (Morphisms) Let $\al : M \to N$ be a morphism
    of $R$-modules.
    Then functoriality of $A \otimes_R \_$
    across all $R \to A$ defines
    a morphism between the quasi-coherent sheaves associated with
    $M$ and $N$.
    Functoriality is readily checked.
  \end{enumerate}
  We have a functor $\QCOH X \to R\MOD$ given by
  taking fiber at $\id : \SPEC R \to \SPEC R$.
  Then this gives an equivalence \[
    R\MOD \simeq \QCOH X  
  \]
\end{prop}
The proof has no geometric content and is included
purely for pedantry.
Indeed, this definition of quasi-coherent sheaves
is designed so that this proposition holds more-or-less by definition.
\begin{proof}
  It is clear that the composition
  $R\MOD \to \QCOH X \to R\MOD$ is on-the-dot
  the identity functor.
  We show the other composition
  $\QCOH X \to R\MOD \to \QCOH X$ is isomorphic to the identity functor.
  Let $\cF \in \QCOH X$, $M = \cF_{\id_R}$ and $\tilde{M}$
  the image of $M$ under $R\MOD \to \QCOH X$.
  Then for each algebra map $x : \SPEC A \to \SPEC R$,
  quasi-coherence of $\cF$ implies that
  the equality $M = \cF_{\id_R}$ extends to
  an isomorphism
  \[
    \varphi_x^\cF : \tilde{M}_x = A \otimes_R M \map{\sim}{} \cF_x
  \]
  Now we show that for $f : \SPEC B \to \SPEC A$,
  the above square in the following diagram commutes :
  \begin{cd}
    {\tilde{M}_{xf}} & {\mathcal{F}_{xf}} \\
    {\tilde{M}_x} & {\mathcal{F}_x} \\
    M & {\mathcal{F}_{\id_R}}
    \arrow[from=2-1, to=1-1]
    \arrow[from=2-2, to=1-2]
    \arrow["{\varphi_x^\cF}"', from=2-1, to=2-2]
    \arrow["{\varphi_{xf}^\cF}", from=1-1, to=1-2]
    \arrow[from=3-1, to=2-1]
    \arrow["{=}"', from=3-1, to=3-2]
    \arrow[from=3-2, to=2-2]
  \end{cd}
  By the universal property of $\tilde{M}_x = A \otimes_R M$,
  it suffices to show that from $M$,
  ``up-up-right = up-right-up''.
  Indeed \begin{align*}
    \text{ ``up-up-right'' }
    &= \text{ ``right-up-up'' }\,\,\,\text{by def of $\varphi_{xf}$} \\
    &= \text{ ``up-right-up'' }\,\,\,\text{by def of $\varphi_x$}
  \end{align*}
  We have thus defined an isomorphism morphism 
  $\varphi^\cF : \tilde{M} \simeq \cF$.
  Now let $\al : \cF \to \cG$ in $\QCOH X$
  and $\varphi^\cG : \tilde{N} \simeq \cG$ be the corresponding
  reconstruction morphism for $\cG$.
  Let $\tilde{\al} : \tilde{M} \to \tilde{N}$ be the
  reconstructed morphism from $\al$.
  We wish to check commutativity of
  \begin{cd}
    {\tilde{M}} & {\mathcal{F}} \\
    {\tilde{N}} & {\mathcal{G}}
    \arrow["{\tilde{\alpha}}"', from=1-1, to=2-1]
    \arrow["{\varphi^\mathcal{G}}"', from=2-1, to=2-2]
    \arrow["\alpha", from=1-2, to=2-2]
    \arrow["{\varphi^\mathcal{F}}", from=1-1, to=1-2]
  \end{cd}
  We consider how we defined $\varphi_x^\cF, \varphi_x^\cG$ in the first place : 
  \begin{cd}
    M & {\mathcal{F}_{\id_R}} \\
    N & {\mathcal{G}_{\id_R}} & {\tilde{M}_x} & {\mathcal{F}_x} \\
    && {\tilde{N}_x} & {\mathcal{G}_x}
    \arrow["{\tilde{\alpha}_x}"', from=2-3, to=3-3]
    \arrow["{\varphi^\mathcal{G}_x}"', from=3-3, to=3-4]
    \arrow["{\alpha_x}", from=2-4, to=3-4]
    \arrow["{\varphi^\mathcal{F}_x}", from=2-3, to=2-4]
    \arrow["{=}", from=2-1, to=2-2]
    \arrow["{=}", from=1-1, to=1-2]
    \arrow["{\alpha_{\id_R}}"', from=1-1, to=2-1]
    \arrow["{\alpha_{\id_R}}", from=1-2, to=2-2]
    \arrow[from=2-1, to=3-3]
    \arrow[from=1-1, to=2-3]
    \arrow[from=2-2, to=3-4]
    \arrow[from=1-2, to=2-4]
  \end{cd}
  By the universal property of $\tilde{M}_x = A\otimes_R M$,
  it suffices to show that starting from $M$
  we have ``diagonal-right-down = diagonal-down-right''.
  Indeed \begin{align*}
    \text{ ``diag-right-down'' }
    &= \text{ ``right-diag-down' }\,\,\,\text{by def of $\varphi_{x}^\cF$} \\
    &= \text{ ``right-down-diag'' }\,\,\,\text{by def of $\al$} \\
    &= \text{ ``down-right-diag'' }\,\,\, \text{clear}\\
    &= \text{ ``down-diag-right'' }\,\,\,\text{by def of $\varphi_{x}^\cG$} \\
    &= \text{ ``diag-down-right'' }\,\,\,\text{by def of $\al_{x}$} \\
  \end{align*}
\end{proof}

\end{document}