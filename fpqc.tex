\documentclass[./main.tex]{subfiles}
\begin{document}

Descent of modules formalizes the idea that
under certain conditions,
one can view $p : \SPEC B \to \SPEC A$ as being a ``surjection''
in the sense that
$A$-modules can be thought of as $B$-modules
equipped identifications of fibers along fibers of $p$.
We begin with motivation from sets again.
\begin{prop}
  
  Let $p : \cU \to X$ be a surjection of sets.
  Define functors 
  \begin{cd}
    {\mathrm{Set} / X} & {\mathrm{Set} /\cU}
    \arrow["{p^*}", from=1-1, to=1-2]
    \arrow["{p_!}"', shift right=5, from=1-2, to=1-1]
    \arrow["{p_*}"', shift left=5, from=1-2, to=1-1]
  \end{cd}
  \begin{enumerate}
    \item $p^*$ sends $E \to X$ to $E \times_X\cU$
    \item $p_!$ sends $F \to\cU$ to $F \to X$
    \item $p_*$ sends $F \to\cU$ to the $X$-family
    $x \mapsto \prod_{u \in p^{-1}(x)} F_u$.
  \end{enumerate}
  Consider the following diagram : 
  \begin{cd}
    {\mathcal{U} \times_X \mathcal{U} \times_X \mathcal{U}} & {\mathcal{U} \times_X \mathcal{U}} & {\mathcal{U}}
    \arrow["{p_0}"{description}, shift left=2, from=1-2, to=1-3]
    \arrow["{p_{12}}"{description}, from=1-1, to=1-2]
    \arrow["{p_{02}}"{description}, shift right=4, from=1-1, to=1-2]
    \arrow["{p_{01}}"{description}, shift left=4, from=1-1, to=1-2]
    \arrow["{p_1}"{description}, shift right=2, from=1-2, to=1-3]
  \end{cd}
  where we have the projections \begin{itemize}
    \item $p_i : \cU \times_X \cU \to \cU, (x_0 , x_1) \mapsto x_i$
    \item $p_{ij} : \cU \times_X \cU \times_X \cU \to \cU \times_X \cU, 
    (x_0 , x_1, x_2) \mapsto (x_i , x_j)$
  \end{itemize}
  Note that pullback along these projections correspond to
  restricting to double and triple intersections.
  Define $(\SET / \cU)^{\cU \times_X \cU}$ as the category where :
  \begin{itemize}
    \item An object of $(\SET / \cU)^{\cU \times_X \cU}$ is 
    a family $\cF \in \SET / \cU$
    equipped with a \emph{transition map} 
    \[
      \phi : p_0^* \cF \cong p_1^* \cF,
      (u_0 , u_1 , v \in F_{u_0}) \mapsto 
      (u_0 , u_1 , \phi(u_0 , u_1 , v) \in F_{u_1})
    \]
    in $\SET / \cU \times_X \cU$ such that
    \begin{enumerate}
      \item (reflexivity) $\phi = \id$ when restricted to the diagonal 
      $\De : \cU \to \cU \times_X \cU$.
      i.e. 
      \[
        \phi(u_0 , u_0 , v) = v  
      \]
      \item (transitivity) the following diagram commute : 
      \begin{cd}
        & {p_{01}^*p_1^* \mathcal{F}} \\
        {p_{01}^*p_0^*\mathcal{F}} && {p_{12}^*p_0^*\mathcal{F}} \\
        {p_{02}^*p_0^*\mathcal{F}} && {p_{12}^*p_1^*\mathcal{F}} \\
        & {p_{02}^*p_1^* \mathcal{F}}
        \arrow["{p_{01}^*(\phi)}", from=2-1, to=1-2]
        \arrow["{p_{12}^*(\phi)}", from=2-3, to=3-3]
        \arrow["{p_{02}^*(\phi)}"', from=3-1, to=4-2]
        \arrow["\sim"', from=2-1, to=3-1]
        \arrow["\sim", from=1-2, to=2-3]
        \arrow["\sim", from=3-3, to=4-2]
      \end{cd}
      % \item for $(U_i , U_j , U_k) \in \cU \times \cU \times \cU$,
      % we have ``$\phi_{ik} = \phi_{jk} \phi_{ij}$''.
      % More rigorously, the following diagram commutes :
      % \begin{cd}[sep = small]
      %   {\mathcal{F}(U_i) \otimes_{\mathcal{O}(U_i)} \mathcal{O}(U_{ij}) \otimes_{\mathcal{O}(U_{ij})} \mathcal{O}(U_{ijk})} & {\mathcal{F}(U_i) \otimes_{\mathcal{O}(U_i)} \mathcal{O}(U_{ijk})} \\
      %   {\mathcal{F}(U_j) \otimes_{\mathcal{O}(U_j)} \mathcal{O}(U_{ij}) \otimes_{\mathcal{O}(U_{ij})} \mathcal{O}(U_{ijk})} & {\mathcal{F}(U_i) \otimes_{\mathcal{O}(U_i)} \mathcal{O}(U_{ik}) \otimes_{\mathcal{O}(U_{ik})} \mathcal{O}(U_{ijk})} \\
      %   {\mathcal{F}(U_j) \otimes_{\mathcal{O}(U_j)} \mathcal{O}(U_{ijk})} \\
      %   {\mathcal{F}(U_j) \otimes_{\mathcal{O}(U_j)} \mathcal{O}(U_{jk}) \otimes_{\mathcal{O}(U_{jk})} \mathcal{O}(U_{ijk})} & {\mathcal{F}(U_k) \otimes_{\mathcal{O}(U_k)} \mathcal{O}(U_{ik}) \otimes_{\mathcal{O}(U_{ik})} \mathcal{O}(U_{ijk})} \\
      %   {\mathcal{F}(U_k) \otimes_{\mathcal{O}(U_k)} \mathcal{O}(U_{jk}) \otimes_{\mathcal{O}(U_{jk})} \mathcal{O}(U_{ijk})} & {\mathcal{F}(U_k) \otimes_{\mathcal{O}(U_k)} \mathcal{O}(U_{ijk})}
      %   \arrow["\sim", from=1-1, to=1-2]
      %   \arrow["\sim"', from=5-1, to=5-2]
      %   \arrow["\sim", from=4-2, to=5-2]
      %   \arrow["\sim"', from=2-2, to=1-2]
      %   \arrow["{\phi_{ik} \otimes \mathrm{id}}", from=2-2, to=4-2]
      %   \arrow["{\phi_{jk} \otimes \mathrm{id}}"', from=4-1, to=5-1]
      %   \arrow["\sim", from=4-1, to=3-1]
      %   \arrow["\sim"', from=2-1, to=3-1]
      %   \arrow["{\phi_{ij} \otimes \mathrm{id}}"', from=1-1, to=2-1]
      % \end{cd}
      % https://q.uiver.app/?q=WzAsOSxbMCwwLCJcXG1hdGhjYWx7Rn0oVV9pKSBcXG90aW1lc197XFxtYXRoY2Fse099KFVfaSl9IFxcbWF0aGNhbHtPfShVX3tpan0pIFxcb3RpbWVzX3tcXG1hdGhjYWx7T30oVV97aWp9KX0gXFxtYXRoY2Fse099KFVfe2lqa30pIl0sWzEsMCwiXFxtYXRoY2Fse0Z9KFVfaSkgXFxvdGltZXNfe1xcbWF0aGNhbHtPfShVX2kpfSBcXG1hdGhjYWx7T30oVV97aWprfSkiXSxbMCw0LCJcXG1hdGhjYWx7Rn0oVV9rKSBcXG90aW1lc197XFxtYXRoY2Fse099KFVfayl9IFxcbWF0aGNhbHtPfShVX3tqa30pIFxcb3RpbWVzX3tcXG1hdGhjYWx7T30oVV97amt9KX0gXFxtYXRoY2Fse099KFVfe2lqa30pIl0sWzEsNCwiXFxtYXRoY2Fse0Z9KFVfaykgXFxvdGltZXNfe1xcbWF0aGNhbHtPfShVX2spfSBcXG1hdGhjYWx7T30oVV97aWprfSkiXSxbMSwzLCJcXG1hdGhjYWx7Rn0oVV9rKSBcXG90aW1lc197XFxtYXRoY2Fse099KFVfayl9IFxcbWF0aGNhbHtPfShVX3tpa30pIFxcb3RpbWVzX3tcXG1hdGhjYWx7T30oVV97aWt9KX0gXFxtYXRoY2Fse099KFVfe2lqa30pIl0sWzEsMSwiXFxtYXRoY2Fse0Z9KFVfaSkgXFxvdGltZXNfe1xcbWF0aGNhbHtPfShVX2kpfSBcXG1hdGhjYWx7T30oVV97aWt9KSBcXG90aW1lc197XFxtYXRoY2Fse099KFVfe2lrfSl9IFxcbWF0aGNhbHtPfShVX3tpamt9KSJdLFswLDEsIlxcbWF0aGNhbHtGfShVX2opIFxcb3RpbWVzX3tcXG1hdGhjYWx7T30oVV9qKX0gXFxtYXRoY2Fse099KFVfe2lqfSkgXFxvdGltZXNfe1xcbWF0aGNhbHtPfShVX3tpan0pfSBcXG1hdGhjYWx7T30oVV97aWprfSkiXSxbMCwyLCJcXG1hdGhjYWx7Rn0oVV9qKSBcXG90aW1lc197XFxtYXRoY2Fse099KFVfail9IFxcbWF0aGNhbHtPfShVX3tpamt9KSJdLFswLDMsIlxcbWF0aGNhbHtGfShVX2opIFxcb3RpbWVzX3tcXG1hdGhjYWx7T30oVV9qKX0gXFxtYXRoY2Fse099KFVfe2prfSkgXFxvdGltZXNfe1xcbWF0aGNhbHtPfShVX3tqa30pfSBcXG1hdGhjYWx7T30oVV97aWprfSkiXSxbMCwxLCJcXHNpbSJdLFsyLDMsIlxcc2ltIiwyXSxbNCwzLCJcXHNpbSJdLFs1LDEsIlxcc2ltIiwyXSxbNSw0LCJcXHBoaV97aWt9IFxcb3RpbWVzIFxcbWF0aHJte2lkfSJdLFs4LDIsIlxccGhpX3tqa30gXFxvdGltZXMgXFxtYXRocm17aWR9IiwyXSxbOCw3LCJcXHNpbSJdLFs2LDcsIlxcc2ltIiwyXSxbMCw2LCJcXHBoaV97aWp9IFxcb3RpbWVzIFxcbWF0aHJte2lkfSIsMl1d
      i.e. $\phi(u_1 , u_2 , \phi(u_0 , u_1 , v)) = \phi(u_0 , u_2 , v)$.
    \end{enumerate}
    \item a morphism $\eta : (\cF , \phi) \to (\cG , \psi)$ is
    a morphism $\eta : \cF \to \cG$ in $\SET / \cU$ such that
    the following commutes : 
    \begin{cd}
      {p_0^* \cF} & {p_1^* \cF} \\
      {p_0^* \cG} & {p_1^* \cG}
      \arrow["{p_0^*(\eta)}"', from=1-1, to=2-1]
      \arrow["{p_1^*(\eta)}", from=1-2, to=2-2]
      \arrow["\psi"', from=2-1, to=2-2]
      \arrow["\phi", from=1-1, to=1-2]
    \end{cd}
  \end{itemize}


  Then we have an equivalence between three categories : 
  \[
    \SET / X \simeq \MOD_T \simeq (\SET / \cU)^{\cU \times_X \cU}  
  \]
  where $\MOD_T$ is the category
  of algebras over the monad $T := p^*p_!$
  under which the following adjunctions coincide : 
  \begin{itemize}
    \item $p_! \dashv p^*$ for $\SET / X$
    \item $\text{ free } \dashv \text{ forget }$ for $\MOD_T$
    \item $L \dashv \text{ forget }$ for $(\SET / \cU)^{\cU \times_X \cU}$
    where $L$ is explicit but I am too lazy right now to describe.
  \end{itemize}
\end{prop}
A key example of a surjection $p : \cU \to X$
is when $\cU = \coprod_{U \in I} U$ where
$I$ is a collection of subsets of $X$ which covers $X$.
In this case $\cU \times_X \cU \simeq \coprod_{U , V \in I} U \cap V$
tracks the pairwise intersections.
\begin{proof}
  Let us make the comparison $\MOD_T \simeq (\SET / \cU)^{\cU \times_X \cU}$
  first.
  Note that \begin{cd}
    {\mathcal{U}} & {\mathcal{U} \times_X \mathcal{U}} & {F \times_X \mathcal{U}} \\
    X & {\mathcal{U}} & F
    \arrow[from=1-1, to=2-1]
    \arrow[from=2-2, to=2-1]
    \arrow[from=2-3, to=2-2]
    \arrow[from=1-3, to=2-3]
    \arrow["{p_1}"', from=1-2, to=1-1]
    \arrow["{p_0}", from=1-2, to=2-2]
    \arrow[from=1-3, to=1-2]
    \arrow["\lrcorner"{anchor=center, pos=0.125, rotate=-90}, draw=none, from=1-2, to=2-1]
    \arrow["\lrcorner"{anchor=center, pos=0.125, rotate=-90}, draw=none, from=1-3, to=2-2]
  \end{cd}
  Concretely, $F \times_X \cU$ consists of
  $(v \in F_{u_0}, u_1)$ where $p(u_0) = p(u_1)$
  and the projection $F \times_X \cU \to \cU$ is $(v , u_0 , u_1) \mapsto u_1$.

  Now let's work out the multiplication $\mu : T^2 \to T$
  of the monad.
  We have 
  \[
    T^2 F = (F \times_X \cU) \times_X \cU
  \]
  The projection $(F \times_X \cU) \times_X \cU$ is
  $(v \in F_{u_0} , u_1 , u_2) \mapsto u_2$.
  So the multiplication $T^2 F \mapsto T F$ is 
  \[
    (v \in F_{u_0} , u_1 , u_2) \mapsto (v \in F_{u_0} , u_2)
  \]
  We can also compute the unit $F \to TF$ as 
  \[
    v \in F_{u_0} \mapsto (v \in F_{u_0} , u_0)  
  \]
  and verify the unit laws : 
  \begin{cd}
    {F \times_X \mathcal{U}} & {(F \times_X \mathcal{U}) \times_X \mathcal{U}} \\
    {(F \times_X \mathcal{U}) \times_X \mathcal{U}} & {F \times_X \mathcal{U}}
    \arrow[from=1-2, to=2-2]
    \arrow[from=1-1, to=2-1]
    \arrow[from=2-1, to=2-2]
    \arrow[from=1-1, to=1-2]
  \end{cd}
  \begin{itemize}
    \item ``unit then multiply'' is 
    $(v \in F_{u_0} , u_1) \mapsto (v \in F_{u_0} , u_0 , u_1)
    \mapsto (v \in F_{u_0} , u_1)$
    \item ``multiply then unit'' is
    $(v \in F_{u_0} , u_1) \mapsto (v \in F_{u_0} , u_1 , u_1)
    \mapsto (v \in F_{u_0} , u_1)$.
  \end{itemize}
  Associativity is the fact
  that given $(v \in F_{u_0} , u_1 , u_2, u_3)$,
  first forgeting $u_2$ then $u_1$ is the same as
  first forgetting $u_1$ then $u_2$.

  We are ready for modules over $T$.
  A map $\phi : F \times_X \cU \to F$ over $\cU$
  sends $(v \in F_{u_0} , u_1) \mapsto \phi(v , u_1) \in F_{u_1}$.
  In other words, $\phi$ transports $v \in F_{u_0}$
  to the fiber $F_{u_1}$ given that $u_0 = u_1$ in $X$.
  This gives $F$ a $T$-module structure when we have 
  \begin{itemize}
    \item (associativity) 
    $\phi(\phi(v \in F_{u_0} , u_1), u_2) = \phi(v \in F_{u_0} , u_2)$
    \item (unity) $\phi(v \in F_{u_0} , u_0) = v$.
  \end{itemize}
  Here is the tautology : 
  $\phi$ is equivalent to giving a map
  $\widetilde{\phi} : F_{u_0} \to F_{u_1}$ for every pair
  $p(u_0) = p(u_1)$.
  In fancy terms, we have $p^* p_! \simeq (p_0)_! p_1^*$ which implies
  \[
    (\SET / \cU)(p^* p_! F , F) \simeq (\SET / \cU \times_X \cU)
    (p_0^* F , p_1^* F)  
  \]
  The point now is that 
  \begin{itemize}
    \item associativity for $\phi$ corresponds to 
    transitivity of $\widetilde{\phi}$
    \item unity for $\phi$ corresponds to reflexivity of $\widetilde{\phi}$.
  \end{itemize}
  What about symmetry of $\widetilde{\phi}$?
  This is implied because 
  \[
    \phi(\phi(v \in F_{u_0} , u_1) , u_0)
    = \phi(v \in F_{u_0} , u_0) = v
  \]
  This proves the equivalence $\MOD_T \simeq (\SET / \cU)^{\cU \times_X \cU}$.
  One can see that the forgetful functor $\MOD_T \to \SET / \cU$
  corresponds to the forgetful functor 
  $(\SET / \cU)^{\cU \times_X \cU} \to \SET / \cU$.

  The comparison $\SET / X \simeq \MOD_T$
  comes from the monadicity theorem.
  Let $\mathrm{Glue} : \MOD_T \to \SET / X$ denote the
  inverse constructed in the proof of the monadicity theorem.
  Indeed, all the categories involved are locally small and 
  we have a triple of adjoints
  $p_! \dashv p^* \dashv p_*$ so $p^*$ preserves small colimits.
  The fact that $p$ is surjective implies that
  $p^*$ is conservative.\footnote{
    In fact, in this situation $p$ being surjective is
  equivalent to conservativity of $p^*$.
  }
  Opening the proof of the monadicity theorem,
  the key part is that
  for any $T$-module $\phi : TF \to F$,
  we have the (split) coequalizer in $\MOD_T$ : 
  \begin{cd}
    {T^2 F} & {T F} & F
    \arrow["\phi", from=1-2, to=1-3]
    \arrow["{\mu_F}", shift left=2, from=1-1, to=1-2]
    \arrow["{T(\phi)}"', shift right=2, from=1-1, to=1-2]
  \end{cd}
  Concretely in our situation, 
  \begin{itemize}
    \item $\mu_F : (v \in F_{u_0} , u_1 , u_2) \mapsto (v \in F_{u_0} , u_2)$
    \item $T(\phi) : (v \in F_{u_0} , u_1 , u_2) \mapsto 
    (\phi(v \in F_{u_0} , u_1) \in F_{u_1} , u_2)$
  \end{itemize}
  The functor $\mathrm{Glue}$ is
  defined by forming the coequalizer diagram in $\SET / X$
  \begin{cd}
    {p_!p^*p_!F} & {p_! F} & {\mathrm{Glue}(F , \phi)}
    \arrow[from=1-2, to=1-3]
    \arrow["{\nu_F}", shift left=2, from=1-1, to=1-2]
    \arrow["{p_!(\phi)}"', shift right=2, from=1-1, to=1-2]
  \end{cd}
  where \begin{itemize}
    \item $\nu_V : (v \in F_{u_0} , u_1, x) \mapsto (v , x)$
    \item $p_!(\phi) : (v \in F_{u_0} , u_1, x) \mapsto (\phi(v , u_1) , x)$
  \end{itemize}
  In other words, for each $x \in X$, 
  the fiber $\mathrm{Glue}(F , \phi)_x$ is obtained by 
  identifying all $F_{u}$ with $p(u) = x$ using the transition
  map $\phi$.
\end{proof}

We now apply the above idea to
standard Zariski covers $(f_1 , \dots , f_n) = A$.
\[
  p : \coprod_{i = 1}^{n} D(f_i) \to \SPEC A  
\]
Here the adjunction is 
\begin{cd}
  {(A\MOD)^\mathrm{op}} & {(B\MOD)^\mathrm{op}}
	\arrow["{B \otimes_A \_}"', shift right=2, from=1-1, to=1-2]
	\arrow["{\text{forget}}"', shift right=2, from=1-2, to=1-1]
	\arrow["\bot"{description}, draw=none, from=1-2, to=1-1]
\end{cd}
where $B = \prod_{i = 1}^n A[1 / f_i]$.

\begin{dfn}[Descent data]
  
  Let $A \to B$ be an algebra map.
  Define $(B\MOD)^{B \otimes_A B}$ as the category with
  \begin{itemize}
    \item objects $(N , \phi)$ where
    $N \in B\MOD$ and $\phi \in (B \otimes_A B)\MOD(B \otimes_A N , N \otimes_A B)$
    such that we have the \emph{cocycle condition},
    i.e. the following triangle commutes \begin{cd}
      & {B \otimes_A N \otimes_A B} \\
      {N \otimes_A B \otimes_A B} \\
      & {B \otimes_A B \otimes_A N}
      \arrow["{\phi_{01}}", from=2-1, to=1-2]
      \arrow["{\phi_{12}}", from=1-2, to=3-2]
      \arrow["{\phi_{02}}"', from=2-1, to=3-2]
    \end{cd}
    where \begin{itemize}
      \item $\phi_{01} = \phi \otimes \id_B$
      \item $\phi_{12} = \id_B \otimes \phi$
      \item $\phi_{02} =$ ``swap 0-th and 2nd factor using $\phi$'',
      which is $n \otimes f \otimes g \mapsto \sum_{i} a_i \otimes f \otimes n_i$
      where $\phi(n \otimes g) = \sum_{i} a_i \otimes n_i$.
    \end{itemize}
    \item a morphism $(N , \phi) \to (N_1 , \phi_1)$ is a
    morphism 
  \end{itemize}
\end{dfn}

\begin{prop}[Descent for standard Zariski covers]
  

\end{prop}
  
\end{document}