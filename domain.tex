\documentclass[./main.tex]{subfiles}
\begin{document}

\begin{prop}
  
  Let $X$ be a scheme.
  Then the following are equivalent : 
  \begin{enumerate}
    \item $X$ is nonempty and every nonempty affine open of $X$
    is an integral domain.
    \item There exists an affine open cover $\cU$ of $X$
    such that each $U \in \cU$ is an integral domain and
    for $U , V \in\cU$ we have $U \cap V \neq \nothing$.
    \item $\abs{X}$ is irreducible and $X$ is reduced.
  \end{enumerate}
\end{prop}

\begin{dfn}
  
  A scheme $X$ is called a \emph{domain} when
  it satisfies any (and thus all) of the conditions in the previous proposition.
\end{dfn}

\begin{prop}
  
  Let $X = \SPEC A$ where $A$ is an integral domain.
  Consider $\varnothing \neq U \subs X$ an open. 
  Then $A \to \cO(U)$ is injective.

  Thus in general, when $X$ is a domain, 
  restriction of local functions along affine opens is injective.
\end{prop}
\begin{proof}
  Let $f \in A$ and assume it is zero in $\cO(U)$.
  Then there exists $g \in A$ with 
  $\varnothing \neq D(g) \subs U$.
  By assumption, $f$ must also be zero when restricted to $D(g)$,
  i.e. it is in the kernel of $A \to A[1 / g]$.
  This implies the existence of $n \geq 0$ with $g^n f = 0$.
  The condition $\varnothing \neq D(g)$ implies $0 \neq g$.
  Hence by $A$ being integral domain,
  $f = 0$.
\end{proof}

\begin{dfn}
  
  Let $X$ be a domain.
  Define the function field of $X$ to be 
  \[
    K(X) := \COLIM_{\nothing \neq U \text{ affine open } \subs X} \cO(U)  
  \]
\end{dfn}

Note that for a nonempty affine open $U$,
the map $\cO(U) \to K(X)$ is injective
because all transition maps of the filtered diagram are injective.
One can thus safely think about $K(X)$ as the
``union'' of $\cO(U)$'s.

\begin{prop}
  
  Let $X$ be a domain.
  Then $K(X)$ is a field.
\end{prop}
\begin{proof}
  Exercise.
\end{proof}

\end{document}