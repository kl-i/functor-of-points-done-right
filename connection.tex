\documentclass[./main.tex]{subfiles}
\begin{document}

Let's review the three definitions of connections on
vector bundles in differential geometry.
Let $\pi : E \to B$ be a rank $m$ vector bundle over an $n$-dimensional manifold $B$.

\begin{dfn}
	
  The \emph{vertical bundle of $E$} is defined as the kernel of
	vector bundles on $E$ : 
	\[
    0 \to V E \to TE \map{D\pi}{} TB \times_B E \to 0
	\]
\end{dfn}
Let's see what this sequence looks like locally on $E$.
Locally on $B$,
we have an isomorphism of vector bundles over spaces
\begin{cd}
	E & {\mathbb{R}^n \times \mathbb{R}^m} & {(x , a)} \\
	B & {\mathbb{R}^n} & x
	\arrow[from=1-1, to=2-1]
	\arrow["\sim", from=2-1, to=2-2]
	\arrow["x"', from=2-1, to=2-2]
	\arrow[from=1-2, to=2-2]
	\arrow["\sim"', from=1-1, to=1-2]
	\arrow["{(x , a)}", from=1-1, to=1-2]
	\arrow[maps to, from=1-3, to=2-3]
\end{cd}
Then the above SES looks like at each $p \in E$ with $b = \pi(p)$
\[
	0 \to \left\langle \frac{\partial}{\partial a^j}_p \right\rangle \to 
	\left\< \frac{\partial}{\partial a^j}_p ,  \frac{\partial}{\partial x^k}_p\right\>
	\to \left\< \frac{\partial}{\partial x^k}_b\right\> \to 0
\]
\begin{dfn}[Family of Horizontal subspaces]
	
	A family of horizontal subspace is a splitting of
	the vertical bundle short exact sequence. 
\end{dfn}
From linear algebra,
giving a splitting $0 \to V \to W \to W / V \to 0$
is the same as giving a retraction of $V \to W$.
Using this, let's describe a choice of family of horizontal subspaces in terms of
concrete data.
A retraction of $0 \to \left\langle \frac{\partial}{\partial a^j}_p \right\rangle \to 
\left\< \frac{\partial}{\partial a^j}_p ,  \frac{\partial}{\partial x^k}_p\right\>$
is equivalent to giving cotangent vectors $\theta^1_p , \dots , \theta^m_p$
in $T^*_p E$ such that \[
	\theta^i_p \frac{\partial}{\partial a^j}_p = 
	\delta^i_j \frac{\partial}{\partial a^j}_p
\]
This is equivalent to the condition
\[
  \theta^j = (d a^j)_p + A^j_k(p) (d x^k)_p	
\]
\begin{dfn}[Linear family of horizontal spaces]
	
	Let $r : TE \to VE$ be a family of horizontal subspaces.
	For $\la \in \bR$ let $\la : E \to E$ be the fiberwise scaling by $\la$ map.
	This in particular induces $d \la : TE \to TE$.
	Then $r$ is called \emph{linear} when for all $e \in E$ and scalars $\la$
	we have 
	\[
		r_{\la e} (d \la)_e = r_e 
	\]
\end{dfn}

% First, the idea is that for $B = \PT$ so $E$ is just a vector space,
% then $TE \simeq E \times T_0 E$ so the SES is
% \[
% 	0 \to T_0 E \to E \times T_0 E \to E \to 0
% \]
% Of course there is an obvious $E \simeq T_0 E$ via
% $v \mapsto \frac{\partial}{\partial v}_0$.
% This works in families.
% \begin{prop}
	
% 	Consider 
% 	\[
% 	  E \times_B E \to T E	
% 	\]
% 	by $p , v \mapsto \brkt{ p , \frac{\partial}{\partial v}_{p}}$.
% 	This map of vector bundles over $E$
% 	is independent of the choice of local coordinates
% 	and induces an isomorphism $E\times_B E \simeq VE$.
% \end{prop}
% \begin{proof}
% 	By dimensions, it suffices to show the map is injective on fibers.
% 	Let $f$ be a smooth function.
%   Using vector bundle coordinates $(x , a)$ and let
% 	$\tilde{f}$ be the corresponding smooth function on $\bR^{n + m}$,
% 	\[
% 		\frac{\partial}{\partial v}_{p} f
% 		= \COLIM_{t \to 0} \frac{f(p + t v) - f(p)}{t} 
% 		= \COLIM_{t \to 0} \frac{\tilde{f}(x(p) , a(p) + t a(v)) - 
% 			\tilde{f}(x(p) , a(p))}{t} 
% 	\]
% 	\[
% 	  = \frac{\partial}{\partial a(v)}_{(x(p) , a(p))} \tilde{f}
% 		= a^j(v) \frac{\partial}{\partial a^j}_{(x(p) , a(p))} \tilde{f}
% 		= a^j(v) \frac{\partial}{\partial a^j}_{p} f	
% 	\]
% 	So $\partial/\partial v$ is zero iff $a^j(v) = 0$ for all $j$
% 	iff $v = 0$ inside the fiber $E$ over $\pi(v)$.
% \end{proof}



For some reason, differential geometers like working with vector bundles
by their total spaces whilst algebraic geometers like working with them 
through their quasi-coherent sheaf of sections.

Let $X = \SPEC A$ where $A$ is a commutative algebra over a field $k$.
Let $G$ be an algebraic group over $k$.
We assume we already know what a principal $G$-bundle $\pi : P \to X$ means.
Considering the sequence of spaces $P \to X \to \SPEC k$, 
we obtain the exact sequence
\[
  0 \to T_{P / X} \to T_{P} \to \pi^* T_X
\]
Let us compute $T_{P / X}$.
For this, I found it easiest to consider it as a space over $P$
and compute its points.
Let $y : U \to P$ be a general point of $P$ where $U$ is an affine.
Let $\tilde{U} := U \times_k \SPEC k[\epsilon]$.
Maps from $y$ to $T_{P / X}$ are lifts $\tilde{y}$ satisfying 
\begin{cd}
  U & P \\
	{\tilde{U}} & X
	\arrow["y", from=1-1, to=1-2]
	\arrow["{x = \pi(y)}"', from=2-1, to=2-2]
	\arrow["\subseteq"', from=1-1, to=2-1]
	\arrow["\pi", from=1-2, to=2-2]
	\arrow["{\tilde{y}}"{description}, dashed, from=2-1, to=1-2]
\end{cd}
Note that $\tilde{U}$ has a retraction to $U$
which gives rise to the zero tangent vector at $y$.
By abuse of notation, we also use $y$ to denote the zero tangent vector
$\tilde{U} \to P$ at $y$.
Giving a general relative tangent vector at $y$ then amounts to
a lift 
\begin{cd}
  & P \\
	{\tilde{U}} & {P \times_X P}
	\arrow["{(y , \tilde{y})}"', dashed, from=2-1, to=2-2]
	\arrow["y", from=2-1, to=1-2]
	\arrow["{\mathrm{fst}}"', from=2-2, to=1-2]
\end{cd}
However by assumption $P \times_X P \simeq P \times G$,
so one must have $\tilde{y} = g y$ for some unique $g \in G(\tilde{U})
= (\LIE G)(U)$.
The above is all functorial in $y$ and hence we deduce 
that as spaces over $P$ we have
\[
  T_{P / X} \simeq P \times \LIE G
\]
In fact, the reverse map is the map producing
fundamental vector fields we previously saw.

Note that $P \to X$ is formally smooth.
Writing everything as quasi-coherent sheaves on $P$ we then have the SES
\[
  0 \to \cO_P \otimes \LIE G \to T_P \to \pi^* T_X \to 0
\]
We can descend this SES to $X$ by modding out $G$.
Formally, we have descent along $G$-bundles
\begin{cd}
  {(\QCOH P)^G } & {\QCOH X}
	\arrow["{\pi^*}"', shift right=2, from=1-2, to=1-1]
	\arrow["{\pi_*(\_^G)}"', shift right=2, from=1-1, to=1-2]
	\arrow["\sim"{description}, draw=none, from=1-1, to=1-2]
\end{cd}
So we obtain the SES in $\QCOH X$,
\[
  0 \to \pi_* (\cO_P \otimes \LIE G)^G \to \pi_*(T_P^G) \to T_X \to 0
\]
A \emph{connection on $P$} is a splitting of this SES in $\QCOH X$.
Computationally, one should give such data on $P$.
One way of doing this is to give a $G$-equivariant retraction
of $T_P$ to $\cO_P \otimes \LIE G$.
This amounts to giving a $G$-equivariant $(\LIE G)$-valued one form $\omega$ 
such that for any element $\de \in \LIE G$,
\[
  \om(X_\de) = \de
\]
\cite[Section 19.1]{Mic}

Exercise (which I have not done) : 
boil this down for $G = \GL_n$.

Suppose one is given a connection on $P$ as
a section $\nabla : T_X \to \pi_*(T_P^G)$.
The connection is called \emph{flat} if this is a morphism of Lie algebroids.

Relation of D-modules and differential equations : 
Let $(\cE , \nabla)$ be a vector bundle on a smooth curve $X$ and $\nabla$
a connection.
Then on an open $U$ of $X$ on which $\cE$ admits trivialising sections
$e_1 , \dots , e_n$, we have for any section $s$ on $U$,
\[
  \nabla s = \nabla(s^i e_i) = ds^i \otimes e_i + s^i \nabla e_i
  = ds^i \otimes e_i + s^i A^{_i^j} \otimes e_j
  = ds + As
\]
Therefore looking for horizontal sections $\nabla s = 0$
is equivalent to solving the $n$ 1-dimensional ordinary differential equation
$d s = - A s$ and then pasting solutions together to all of $X$.

\end{document}