\documentclass[./main.tex]{subfiles}
\begin{document}

\begin{dfn}
  
  Let $f : X \to Y$ in $\PSH \AFF$.
  Define \[
    f^* : \QCOH Y \to \QCOH X  
  \]
  by $\cF \mapsto (x \in X(A) \mapsto \cF_{fx})$.
  This is symmetric monoidal with respect to the tensor product of
  quasi-coherent sheaves.
\end{dfn}

\begin{prop}
  
  Let $f : \SPEC B \to \SPEC A$.
  Then under the identification $\QCOH \SPEC A = A\MOD$
  and $\QCOH \SPEC B = B\MOD$,
  we have 
  \[
    f^* M \simeq B \otimes_A M  
  \]
\end{prop}
\begin{proof}
  Exercise.
\end{proof}

\begin{prop}
  
  Let $f : X \to Y$ be a morphism between schemes.
  Then there exists an adjunction
  \[
    f^* \dashv f_* : \QCOH X \leftrightarrows \QCOH Y  
  \]
\end{prop}
\begin{proof}
  This is abstract non-sense and has no geometric content.
  The fact that $X , Y$ are schemes ensure that
  $\QCOH X$, $\QCOH Y$ are equivalent to small categories 
  so that one can apply the adjoint functor theorem to $f^*$.
  The reason that $f^*$ preserves small colimits is that
  they are computed fiberwise.
\end{proof}

We will show how to compute sections of pushforward on affine opens.
This requires a technical assumption of quasi-compact quasi-separated-ness
which is covered in the next section.
\begin{prop}[Base change for affine opens]
  
  Let $f : X \to Y$ be a qcqs morphism between schemes.
  Let $j : U \subs Y$ be an affine open and $\cF \in \QCOH X$.
  Then \[
    (f_* \cF)(U) \simeq \cF(f^{-1}(U))  
  \]
\end{prop}
Since $U$ is affine,
we are equivalently proving $j^* f_* \simeq (f_1)_* j_1^*$
which is an example of base change.
\begin{proof}
  Proved next section.
\end{proof}

\end{document}