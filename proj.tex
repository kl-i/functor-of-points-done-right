\documentclass[./main.tex]{subfiles}
\begin{document}

\begin{enumerate}
  \item We have colimits in $\SH_\ZAR \AFF$ so we
  can just quotient in there.
  But because sheafification, this is intractible to compute.
  How do we even show this is a scheme?
  (aside, trying to access this anyway leads to algebraic stacks.)
  \item group quotients in sets. Functor of points
  = ``family of orbits'' = equivariant map from torsor.
  \item Define Zariski torsors over a scheme. Show they give group quotient
  in Zariski sheaves.
  We take torsors as our definition of scheme quotients.
  \item Example : projective space is Zariski $\bG_m$-torsor.
  \item Example : line bundles equivalent to Zariski $\bG_m$-torsors.
  More generally, rank $n$ vector bundles
  equivalent to Zariski $\GL_n$-torsors.
  \item Proj construction as a method of producing $\bG_m$-torsors
  which are also projective.
  This includes closed subschemes of projective space.
  \item A full discussion of how to quotient by algebraic groups
  to get schemes is a delicate question.
  Indeed, one approach is that \emph{there is no general quotient
  in schemes without choices} and
  that we must enlarge our category of ``spaces glued by opposite of
  commutative rings'' again.
  This brings us to \emph{algebraic stacks}.

  In the above sense, proj is a hack for $G = \bG_m$.
  We will not prove any equivalences.
  This is really just a ``method''.
  \item $\bA^1$-action equivalent to $\bN$-grading.
  \item $\SPEC A \setminus (0 \cdot \SPEC A) \to 
  \brkt{\SPEC A \setminus (0 \cdot \SPEC A)} / \bG_m$
  is $\bG_m$-torsor.
  \item Gluing affine construction for global proj.
\end{enumerate}

\end{document}