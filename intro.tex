\documentclass[./main.tex]{subfiles}
\begin{document}

This documents my personal journey of understanding the foundations of
modern algebraic geometry completely from the functorial point of view.
At the risk of sounding controversial,
I believe this is the way involving \emph{least} amount of technology,
however requires developing the most intuition.

Problems with courses in modern algebraic geometry I've seen :
\begin{enumerate}
  \item don't get to schemes, leaving students unable to access 
  modern AG
  \item amount of theory of sheaves on topological spaces
  is demotivating.
  I want sheaves to appear as
  requiring coverings to be effective epimorphisms.
  \item With locally ringed spaces, it feels difficult to write down
  a scheme. 
  New students often mistakenly conflate 
  a scheme with its underlying topological space.
  The point that ``schemes = commutative algebra + descent''
  is clouded.
  \item takes too long to get to modern treatment of varieties
  \item With locally ringed spaces,
  quasi-coherent sheaves live in a larger category of topological sheaves.
  This obscures the idea that
  ``quasi-coherent sheaves = a module in any chart compatibly''
  and also takes longer to develop technically speaking.
  \item too many properties of morphisms. 
  Hard for students to get a handle on.
  \item sheaf cohomology : is there really a need to see
  abstract sheaf cohomology when seeing schemes for first time?
\end{enumerate}
When corrected motivated,
the functor of points approach solves (1), (2), (3), (4), (5).
Point (7) is debatable. 
Point (6) may be better if there was a specific purpose for
learning scheme theory?
E.g. if one focuses on curves?

Places where functors shine : 
\begin{enumerate}
  \item a scheme is really nothing more than 
  ``affine schemes quotient along opens''.
  \item definition of pullback of quasi-coherent sheaves
  \item the associated topological space of a scheme is \emph{secondary},
  so it avoids confusion.
  \item definition of tangent bundle
  \item formal schemes and ind-schemes are easy to introduce
  and compute with
  \item the generalization to algebraic spaces
  is conceptually explained as replacing quotients along
  open immersions to quotients along étale morphisms
  \item the generalization to stacks is conceptually explained
  as using groupoids to describe ``non-propositional equivalence relations''.
\end{enumerate}

Some questions for myself : 
\begin{enumerate}
  \item Maths where schemes are indispensible?
  \begin{itemize}
    \item Modular forms to Galois reps.
    Modular curves are usually described complex analytically.
    But the moduli description makes them over $\bQ$.
    Now étale cohomologies naturally carry action from absolute Galois group.
  \end{itemize}
\end{enumerate}

\end{document}